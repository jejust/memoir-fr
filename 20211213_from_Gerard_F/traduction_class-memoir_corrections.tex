\documentclass[12pt]{article}
\usepackage[T1]{fontenc}       
\usepackage[utf8]{inputenc}   
\usepackage[frenchb]{babel}
\usepackage[left=2cm,right=2cm,top=2cm,bottom=2cm]{geometry}
\usepackage{hyperref}
\usepackage[tikz]{bclogo}

\usepackage{parallel}

\newcommand{\note}[1]{
 \begin{bclogo}[couleur = white,noborder,logo=\bcattention]{Note}
%
#1
%
 \end{bclogo}
}

\newcommand{\trait}{
\hrule
\vspace{5mm}

}

%\newcommand{\Mname}{\texttt{memoir}}
\newcommand{\Pclass}[1]{\texttt{#1}}
\newcommand{\Ie}[1]{\texttt{#1}}
\newcommand{\ltx}{\LaTeX}
\newcommand{\ctt}{\textsc{ctt}}

\begin{document}

\section{Pr\'eface}  
%\switchEng
\begin{Parallel}[v]{0.475\textwidth}{0.475\textwidth}
 \ParallelLText{
    From personal experience and also from lurking on the \url{comp.text.tex} 
newsgroup the major problems with using \ltx\ are related to document
design. Some years ago most questions on \ctt\ were answered by
someone providing a piece of code that solved a particular problem, and
again and again. More recently these questions are answered along the
lines of `Use the ---------{} package', and again and again.
\\}
%\switchFrench\TS{1}

  \ParallelRText{
      D'apr\`es mon exp\'erience personnelle et celle des membres du groupe de discussion \url{comp.text.tex}, les principaux prob\`emes li\'es \`a l'utilisation de \ltx\ sont li\'es \`a la conception des documents. Il y a quelques ann\'ees, la plupart des questions pos\'ees sur \ctt\ recevaient une r\'eponse sous la forme d'un bout de code qui r\'esolvait un probl\`eme particulier, et ainsi de suite. Plus r\'ecemment, la r\'eponse aux questions ressemble davantage \`a \og{}Utilisez l'extension ---------{}\fg{}, et ainsi de suite.}
\end{Parallel}
\note{une coquille : dessemble --> ressemble}
\trait
%-------------------
%\switchEng
\begin{Parallel}[v]{0.475\textwidth}{0.475\textwidth}
 \ParallelLText{
    I have used many of the more common of these packages but my filing system
is not always well ordered and I tend to mislay the various user manuals,
even for the packages I have written. The \Pclass{memoir} class is an attempt
to integrate some of the more design-related packages with the LaTeX
\Pclass{book} class. I chose the \Pclass{book} class as the \Pclass{report} class
is virtually identical to \Pclass{book}, except that \Pclass{book} does
not have an \Ie{abstract} environment while \Pclass{report} does; however it is 
easy to fake an \Ie{abstract} if it is needed. With a little bit of tweaking,
\Pclass{book} class documents can be made to look just like \Pclass{article}
class documents, and the \Pclass{memoir} class is designed with tweaking very
much in mind.}
  \ParallelRText{
%\switchFrench\TS{1}
    J'ai beaucoup utilis\'e les paquets les plus courants, mais mon syst\`eme de classement n'est pas toujours bien ordonn\'e et j'ai tendance \`a  \'egarer les divers manuels d'utilisation, m\^eme pour les paquets que j'ai \'ecrits. La classe \Pclass{memoir} est une tentative d'int\'egration de certains des paquets les plus li\'es \`a la conception avec la classe \Pclass{book} de  \ltx. J'ai choisi la classe \Pclass{book} car la classe \Pclass{report} est virtuellement identique \`a \Pclass{book}, \`a l'exception du fait que \Pclass{book} n'a pas d'environnement \Ie{abstract} alors que \Pclass{report} en a un ; cependant, il est facile de simuler un \Ie{abstract} si n\'ecessaire. Avec un peu d'ajustement, les documents de la classe \Pclass{book} peuvent ressembler aux documents de la classe \Pclass{article}, et la classe \Pclass{memoir} a \'et\'e con\c{c}ue en tenant compte de cet ajustement.}
 \end{Parallel}
\note{j'ai traduit  mémoire en \Pclass{memoir} et mise en forme des mots \Pclass{book report article abstract}. Idem pour LaTeX transformé en \ltx\ (bien qu'il ne soit pas ainsi dans le texte anglais).}
\trait
%--------------------------
\newpage
%\switchEng
\begin{Parallel}[v]{0.475\textwidth}{0.475\textwidth}
 \ParallelLText{
    The \Pclass{memoir} class effectively incorporates the facilties that
are usually accessed by using external packages. In most cases the class
code is new code reimplementing package functionalities. The exceptions
tend to be where I have cut and pasted code from some of my packages.
I could not have written the \Pclass{memoir} class without the excellent 
work presented by the implementors of \ltx\ and its many packages.}
  \ParallelRText{
%\switchFrench\TS{1}
    La classe \Pclass{memoir} incorpore effectivement les facilit\'es auxquelles on acc\`ede habituellement en utilisant des paquets externes. Dans la plupart des cas, le code de la classe est un nouveau code r\'eimpl\'ementant les fonctionnalit\'es du paquet. Les exceptions sont les cas o\`u j'ai coup\'e et coll\'e du code de certains de mes paquets. Je n'aurais pas pu \'ecrire la classe \Pclass{memoir} sans l'excellent travail pr\'esent\'e par les impl\'ementeurs de \ltx\  et de ses nombreux paquets.}
\end{Parallel}
\note{mise en forme du mot \Pclass{memoir} et\ltx. \\
remplacement du mot paquetage par paquet (pourrait-on laisser  \emph{package} ?)}
\trait
%-------------------------------

\end{document}
