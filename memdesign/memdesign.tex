% memdesign.tex   Part I of memman.tex (the Memoir user manual) as a separate doc
%                Author: Peter Wilson
%                Copyright 2001, 2002, 2003, 2004, 2008, 2009 Peter R. Wilson
%
%%   Author: Peter Wilson  (herries dot press at earthlink dot net)
%%           Herries Press
%%   Copyright 2001--2009 Peter R. Wilson
%% 
%%   This work may be distributed and/or modified under the
%%   conditions of the LaTeX Project Public License, either
%%   version 1.3 of this license or (at your option) any
%%   later version.
%%   The latest version of the license is in
%%      http://www.latex-project.org/lppl.txt
%%   and version 1.3 or later is part of all distributions of
%%   LaTeX version 2003/06/01 or later.
%% 
%%   This work has the LPPL maintenance status "maintained".
%%   Maintainer: Lars Madsen (daleif at math dot au dot dk)
%% 
%%   
%
%%%%%%%%%%%%%%%%%%%%%%%%%%%%%%%%%%%%%%%%%%%%%%%%%%%%%%%%%%%%%%%%%%%%%%%%%%%%
%% If you do not have the FontSite 500 fonts use the draft option whereby %%
%% all the FontSite fonts will be replaced by the regular body font, thus %%
%% invalidating many of the typeface examples.                            %%
%%                                                                        %%
%% The free Web-O-Mints fonts is also required. Download from             %%
%% http://www.galapagosdesign.com                                         %%
%% LaTeX support is on CTAN in /fonts/webomints                           %%
%%%%%%%%%%%%%%%%%%%%%%%%%%%%%%%%%%%%%%%%%%%%%%%%%%%%%%%%%%%%%%%%%%%%%%%%%%%%

\documentclass[10pt,letterpaper,extrafontsizes]{memoir}
%\documentclass[draft,10pt,letterpaper,extrafontsizes]{memoir}
\listfiles
\usepackage{comment}
\usepackage{svn-multi}

\svnidlong
{$HeadURL: https://svn.nfit.au.dk/svn/memoir/trunk/doc/memdesign.tex $}
{$LastChangedDate: 2018-09-14 10:14:49 +0200 (Fri, 14 Sep 2018) $}
{$LastChangedRevision: 617 $}
{$LastChangedBy: daleif@math.au.dk $}

\frenchspacing






% For (non-printing) notes  \PWnote{date}{text}
\newcommand{\PWnote}[2]{} 
\PWnote{2009/04/29}{Added fonttable to the used packages}

% same
\newcommand{\LMnote}[2]{} 

%%%% kill bibliographystyle used in memsty.sty
\renewcommand*{\bibliographystyle}[1]{}
\usepackage{memsty}


\hypersetup{
  pdftitle={A Few Notes on Book Design},
  pdfauthor={Peter Wilson, maintained by Lars Madsen},
}





%%%%%%%%%%%%%%%%%%%%%%%%%%%%
\usepackage{titlepages}  % code of the example titlepages
\usepackage{fonttable}[2009/04/01]   % font tables
%%%%%%%%%%%%%%%%%%%%%%%%%%%%
%%\renewcommand*{\FSfont}[1]{}
%%%% Change section heading styles
%%%\memmansecheads

%%% use biblatex for the bibliography
\usepackage[style=authoryear,backend=bibtex]{biblatex}
%%% use \autocite instead of \cite to get parentheses round refs
\providecommand*{\onctan}{On CTAN at }
\let\origbibsetup\bibsetup
\renewcommand*{\bibsetup}{\origbibsetup\small\raggedyright}
\defbibnote{rectan}{{\small 
CTAN is the Comprehensive TeX Archive Network.
Information on how to access CTAN is available at
\url{http://www.tug.org}.}
\par\vspace{\onelineskip}}
\bibliography{memetc}



%%%% Use the built-in division styling
\headstyles{memman}
%%%% but swap section & subsection heads
\setsecheadstyle{\normalfont\large\bfseries\raggedright}
\setsubsecheadstyle{\normalfont\scshape\raggedright}

\newcommand*{\sqrd}[1]{#1\textsuperscript{2}}
\newcommand*{\gsm}{g/\sqrd{m}}
\newcommand*{\half}{\slashfrac{1}{2}}


%%%% Font selection
%% \textparagraph really only defined in TS1
\usepackage[TS1]{fontenc}
\renewcommand*{\encodingdefault}{T1}

\newcommand*{\thisfont}[5]{\fontencoding{T1}%
  \fontsize{#1}{#2}\fontfamily{#3}\fontseries{#4}\fontshape{#5}%
  \selectfont}
\newcommand*{\pickfont}[1]{\thisfont{12}{14}{#1}{m}{n}}
\newcommand*{\Pickfont}[1]{\thisfont{18}{14}{#1}{m}{n}}
\newcommand*{\Pickit}[1]{\thisfont{18}{14}{#1}{m}{it}}
\newcommand*{\Picksl}[1]{\thisfont{18}{14}{#1}{m}{sl}}
\newcommand*{\pickfontx}[1]{\thisfont{10}{12}{#1}{m}{n}}
  
\newcommand*{\caslon}{\pickfont{5ca}}
\newcommand*{\garamond}{\pickfont{5gm}}
\newcommand*{\della}{\pickfont{5de}}
\newcommand*{\bodoni}{\pickfont{5bd}}
\newcommand*{\Caslon}{\Pickfont{5ca}}
\newcommand*{\Garamond}{\Pickfont{5gm}}
\newcommand*{\Della}{\Pickfont{5de}}
\newcommand*{\Bodoni}{\Pickfont{5bd}}

\newcommand*{\baskervillex}{\pickfontx{5bv}}
\newcommand*{\bellx}{\pickfontx{5lb}}
\newcommand*{\bembox}{\pickfontx{5bo}}
\newcommand*{\bodonix}{\pickfontx{5bd}}
\newcommand*{\caslonx}{\pickfontx{5ca}}
\newcommand*{\centaurx}{\pickfontx{5jr}}
\newcommand*{\clarendonx}{\pickfontx{5cd}}
\newcommand*{\futurax}{\pickfontx{5fu}}
\newcommand*{\garamondx}{\pickfontx{5gm}}
\newcommand*{\gillsansx}{\pickfontx{5ch}}
\newcommand*{\optimax}{\pickfontx{5op}}

%%% \FSfont is used in titlepages (for FontSite fonts)
%%% \LXfont is for \ltx fonts, defined as:
%%% \newcommand*{\LXfont}[1]{\fontfamily{#1}\selectfont}
%%%%% If FontSite fonts are not available, set all to body font
\ifdraftdoc
  \typeout{You have used the draft option. Any FontSite 500 font}
  \typeout{will be replaced by the body font, thus invalidating}
  \typeout{many of the typface examples.}
  \renewcommand*{\pickfont}[1]{\thisfont{12}{14}{ppl}{m}{n}}
  \renewcommand*{\Pickfont}[1]{\thisfont{18}{14}{ppl}{m}{n}}
  \renewcommand*{\Pickit}[1]{\thisfont{18}{14}{ppl}{m}{it}}
  \renewcommand*{\Picksl}[1]{\thisfont{18}{14}{ppl}{m}{sl}}
  \renewcommand*{\pickfontx}[1]{\thisfont{10}{12}{ppl}{m}{n}}
  \renewcommand*{\FSfont}[1]{}
\fi

%%% for font examples
\newcommand*{\atoq}{abcefghijopstuvwyz}
\renewcommand*{\atoq}{ab\-ce\-fg\-hi\-jo\-ps\-tu\-vw\-yz}
\newcommand*{\ligs}{ff fi fl ffi ffl \&}
\newcommand*{\onetoo}{\ligs\ 1236 AVOQ}
\newcommand*{\Qligs}{Q \ligs}
\newcommand*{\Onetoo}{1236 AVOQ \&}

\newcommand*{\Shelley}{% Percy Bysshe Shelley (1792--1822) O World, O Life, O Time
\hspace*{2em} Out of the day and night \\
\hspace*{2em} A joy has taken flight --- \\
Fresh spring and summer and winter hoar \\
\hspace*{1em} Move my faint heart with grief, but with delight \\
\hspace*{2em} No more, O never more!}

\newcommand*{\Beddoes}{% Thomas Lovell Beddoes (1803--1849) Song
How many times do I love thee, dear? \\
\hspace*{1em} Tell me how many thoughts there be \\
\hspace*{4em} In the atmosphere \\
\hspace*{4em} Of a new-fall'n year, \\
Whose white and sable hours appear \\
\hspace*{1em} The latest flake of Eternity--- \\
So many times do I love thee, dear.}

\newcommand*{\Campion}{% Thomas Campion (1567--1620) When to Her Lute Corinna Sings
When to her lute Corinna sings, \\
Her voice revives the leaden strings, \\
And doth in highest notes appear \\
As any challenged echo clear; \\
But when she doth of mourning speak, \\
Ev'n with her sighs the strings do break.}

\newcommand*{\Housman}{% A. E. Housman (1859--1936) When I Was One-and-Twenty
When I was one-and twenty \\
\hspace*{1em} I heard a wise man say, \\
`Give crowns and pounds and guineas \\
\hspace*{1em} But not your heart away; \\
Give pearls away and rubies \\
\hspace*{1em} But keep your fancy free.' \\
But I was one-and-twenty, \\
\hspace*{1em} No use to talk to me.}

\newcommand*{\meol}{// }
\renewcommand*{\meol}{}
\newcommand*{\mypar}{{\fontencoding{TS1}\selectfont\textparagraph}\ }

\newcommand{\facetext}{Puella Rigensis ridebat \meol 
Quam tigris in tergo vehebat; \meol 
Externa profecta, \meol 
Interna revecta, \meol 
Risusque cum tigre manebat.  \mypar
Meum est propositum, \meol 
In taberna mori, \meol
Ut sint vina proxima, \meol
Morientis ori. \meol
Tunc cantabunt laetius \meol
Angelorum chori; \meol
`Sit Deus propitius \meol
Huic potatori!'  \mypar
Gaudeamus igitur, \meol
Juvenes dum sumus \meol
Post jucundum juventutem, \meol
Post molestam senectutem, \meol
Nos habebit humus.\par}


\makeatletter
\LMnote{2010/10/28}{the definition from memsty is better}
% \renewcommand{\doidxbookmark}[1]{{\def\@tempa{Symbols}\def\@tempb{#1}%
%   \centering\bfseries \ifx\@tempa\@tempb %
%   Analphabetics
%   \phantomsection
%   \pdfbookmark[1]{Analphabetics}{Analphabetics-idx}%
%   \else
%   #1%
%   \pdfbookmark[1]{#1}{#1-idx}%
%   \fi%
%   \vskip\onelineskip\par}}

%%% ToC down to subsections
\settocdepth{subsection}
%%% Numbering down to subsections as well
\setsecnumdepth{subsection}



\newcommand\U[2]{\textrm{#1}\,\textrm{#2}}


\makeatletter
\renewcommand*{\setupparasubsecs}{%
  \let\oldnumberline\numberline%
  \renewcommand*{\cftsubsectionfont}{\itshape}%
  \renewcommand*{\cftsubsectionpagefont}{\itshape}%
  \renewcommand{\l@subsection}[2]{%
    \ifnum\c@tocdepth > 1\relax%
    \def\numberline####1{\textit{####1}~}%
    \leftskip=\cftsubsectionindent%
    \rightskip=\@tocrmarg%
    %% \advance\rightskip 0pt plus \hsize% uncomment this for raggedright
    %% \advance\rightskip 0pt plus 2em    % uncomment this for semi-ragged
     \parfillskip=\fill%
    \rightskip=3.55em plus 3fil%
    \ifhmode,\enskip \else\noindent\fi%
    {\cftsubsectionfont ##1}~{\cftsubsectionpagefont##2}%
    \let\numberline\oldnumberline%
    \ignorespaces%
    \fi}%
}

\makeatother


%% end preamble
%%%%%%%%%%%%%%%%%%%%%%%%%%%%%%%%%%%%%%%%%%%%%%%%%%%%%%%
\begin{document}
\tightlists
%%%%\firmlists
\midsloppy
\raggedbottom
\chapterstyle{demo3}

%%%%%%%%%%%%%%%%%%%%%%%%%%%%%%%%%%%%%%%%%%%%%%%%%%%%%%%

\frontmatter
\pagestyle{empty}

% half-title page

\vspace*{\fill}
\begin{adjustwidth}{1in}{1in}
\begin{flushleft}
\HUGE\sffamily A Few
\end{flushleft}
\begin{center}
\HUGE\sffamily  Notes on
\end{center}
\begin{flushright}
\HUGE\sffamily  Book Design
\end{flushright}
\end{adjustwidth}
%\vspace*{0.55\textheight}
\vspace*{0.6\textheight}
\vspace*{\fill}
Last update: \svnfileyear-\svnfilemonth-\svnfileday
\cleardoublepage

% title page
\vspace*{0.5\fill}
\begin{center}
\LARGE\textsf{A Few}\par
\end{center}
\begin{center}
\HUGE\textsf{Notes}\par
\end{center}
\begin{center}
\LARGE\textsf{on}\par
\end{center}
\begin{center}
\HUGE\textsf{Book Design}\par
\end{center}

\vspace{2\onelineskip}
\begin{center}
\LARGE\textsf{Peter Wilson}\par
\end{center}
\vspace*{\fill}
\def\THP{T\kern-0.2em H\kern-0.4em P}%   OK for CMR
\def\THP{T\kern-0.15em H\kern-0.3em P}%   OK for Palatino
\newcommand*{\THPress}{The Herries Press}%
\begin{center}
\settowidth{\droptitle}{\textsf{\THPress}}%
\textrm{\normalsize \THP} \\
\textsf{\THPress} \\[0.2\baselineskip]
\includegraphics[width=\droptitle]{anvil2.mps}
\setlength{\droptitle}{0pt}%
\end{center}
\clearpage

% copyright page
\begingroup
\footnotesize
\setlength{\parindent}{0pt}
\setlength{\parskip}{\baselineskip}
%%\ttfamily
\textcopyright{} 2001 -- 2009 Peter R. Wilson \\
All rights reserved

The Herries Press, Normandy Park, WA.

Printed in the World 

The paper used in this publication may meet the minimum requirements
of the American National Standard for Information 
Sciences --- Permanence of Paper for Printed Library Materials, 
ANSI Z39.48--1984.

\begin{center}
19 18 17 16 15 14 13 12 11 10 09\hspace{2em}6 5 4 3 2 1
\end{center}
\begin{center}
\begin{tabular}{ll}
First edition:                        & August 2009 \\
\end{tabular}
\end{center}



\bigskip

Maintained by Lars Madsen



\endgroup






\clearpage
\mbox{}

\cleardoublepage

% ToC, etc
%%%\pagenumbering{roman}
\pagestyle{headings}
%%%%\pagestyle{Ruled}


\setupshorttoc

\tableofcontents



\cleardoublepage

\setupparasubsecs

\setupmaintoc


\begingroup

% important point here: We need \endlineshar=-1 here for the inline
% list of subsections. Why? Beacause we have subsection subsubsection
% subsection, and under hyperref running the l@subsubsection for
% subsubsection, which typesets nothing, ruins our \ignorespaces in
% our redefinition of \l@subsection (it cannot see and ignore the space after the
% \contentsline line for subsubsection). Easiest solution: use
% change \endlinechar
%
% Special thanks to David Carlisle in the tex.stackexchange.com chat
% for suggesting it


\endlinechar=-1


\tableofcontents

\endgroup



%\tracingoutput=0
%\tracingmacros=0

\setlength{\unitlength}{1pt}
\clearpage 
\listoffigures
\clearpage
\listoftables
\clearpage

\chapter{Preface}

    Some fifteen or so years ago I started developing code for typesetting
documents that would make it easy for designers to get the appearance
they had in mind.

% In 2001 this resulted in the \Pclass{memoir} class for 
%use with the \ltx\ typesetting system developed by Leslie Lamport based 
%on Donald Knuth's \tx\ system.

   While doing this I read a lot about book design and have tried my hand
at printing a variety of books and ephemera using hand set lead type 
and a hand operated Chandler \& Price 1904 Old Style 8 by 12 platen
press, pretty much as Gutenberg did some five and a half centuries ago.

   These notes are partly based on my own amateur
experience and feelings but the majority have been culled from the
professionals.

{\raggedleft{\scshape Peter Wilson} \\ Seattle, WA \\ July 2009\par}


\vskip 3cm

\noindent
This document is maintained by Lars Madsen,
daleif(at)math.au.dk. Please write him if you find any typos, errors or
other omissions.







%%%%%%%%%%%%%%%%%%%%%%%%%%%%%%%%%%%%%%%%%%%%%%%%%%%%%%%%%%%%%%%%%%%%%%%%%

\chapter{Introduction}

    These notes briefly cover some aspects of book design and typography, 
independently of the means of typesetting. Among
the several books on the subject listed in the \bibname{} I prefer
Bringhurst's \btitle{The Elements of Typographic Style}~\autocite{BRINGHURST99}.

    The notes originally formed the first part of a user manual for
the \Pclass{memoir} class for 
use with the \ltx\ typesetting system developed by Leslie 
Lamport~\autocite{LAMPORT94} based 
on Donald Knuth's \tx\ system~\autocite{TEXBOOK}. 
The manual was first published in 2001
and as the notes have grown in size and \Pclass{memoir}'s capabilities 
have been extended the manual also grew to approaching 700 pages~\autocite{MEMMAN}. 
At that point seemed advantageous to separate the design notes from the 
technicalities, hence this document.




%%%%%%%%%%%%%%%%%%%%%%%%%%%%%%%%%
\chapter{Terminology}
%%%%%%%%%%%%%%%%%%%%%%%%%%%%%%%%%%

    Like all professions and trades, typographers and printers have their
specialised vocabulary.

First there is the question of pages, leaves and sheets.  The trimmed
sheets of paper\index{paper} that make up a book are called
\emph{leaves}\index{leaf}, and I will call the untrimmed sheets the
\emph{stock}\index{stock} material.  A leaf has two sides, and a
\emph{page}\index{page} is one side of a leaf.  If you think of a book
being opened flat, then you can see two leaves. The front of the
righthand leaf, is called the \emph{recto}\index{recto} page of that
leaf, and the side of the lefthand leaf that you see is called the
\emph{verso}\index{verso} page of that leaf.  So, a leaf has a recto
and a verso page. Recto pages are the odd-numbered pages and verso
pages are even-numbered.

   Then there is the question of folios. The typographical term for
the number of a page is \emph{folio}\index{folio}.
This is not to be confused with
the same term as used in `Shakespeare's first folio' where the reference is
to the height and width of the book, nor to its use in the phrase
`\emph{folio} signature'\index{signature} where the term refers to the 
number of times a printed sheet is folded. 
Not every page in a book has a printed
folio, and there may be pages that do not have a folio at all. Pages with
folios, whether printed or not, form the \emph{pagination}\index{pagination} 
of the book. Pages
that are not counted in the pagination have no folios.

 I have not been able to find what I think is a good
definition for `type' as it seems to be used in different contexts with
different meanings. It appears to be a kind of generic word; for instance
there are type designers, type cutters, type setters, type foundries,...
For my purposes I propose that \emph{type}\index{type|seealso{typeface}} is 
one or more printable characters (or variations or extensions to this idea).  
Printers use the term \emph{sort}\index{sort} to refer to one piece of lead
type.

   A \emph{typeface}\index{typeface} is a set of one or more fonts, in one
or more sizes, designed as a stylistic whole. 

   A \emph{font}\index{font} is a set of characters. In the days of 
metal type and hot lead a font meant a complete alphabet and auxiliary
characters in a given size. More recently it is taken to mean a complete
set of characters regardless of size. A font of roman type normally
consists of CAPITAL LETTERS, \textsc{small capitals}, lowercase letters,
numbers, punctuation marks, ligatures (such as `fi' and `ffi'), and a
few special symbols like \&.

   A \emph{font family}\index{font!family} is a set of fonts designed to
work harmoniously together, such as a pair of roman and italic fonts.

   The size of a font\index{font} is expressed in points\index{point} 
(72.27 points equals 1 inch
equals 25.4 millimetre). The size is a rough indication of the height
of the tallest character, but different fonts with the same size may have
very different actual heights. Traditionally font sizes were referred to
by names (see \tref{tab:fontsizes}) but nowadays just the number of points 
is used.


\begin{table}
\centering
\caption{Traditional font size designations} \label{tab:fontsizes}
\begin{tabular}{cl@{\hspace{2em}}cl} \toprule
Points & Name & Points & Name \\ \midrule
%%3      & Excelsior \\
\phantom{0}3      & Excelsior &
11     &  Small Pica \\
\phantom{0}3\rlap{\slashfrac{1}{2}} & Brilliant &
12     & Pica \\
\phantom{0}4      & Diamond &
14     & English \\
\phantom{0}5      & Pearl &
18     & Great Primer \\
\phantom{0}5\rlap{\slashfrac{1}{2}} & Agate &
24     & Double (or Two Line) Pica \\
\phantom{0}6      & Nonpareil &
28     & Double (or Two Line) English \\
\phantom{0}6\rlap{\slashfrac{1}{2}} & Mignonette &
36     & Double (or Two Line) Great Primer \\
\phantom{0}7      & Minion &
48     & French Canon (or Four Line Pica) \\
\phantom{0}8      & Brevier &
60     & Five Line Pica \\
\phantom{0}9      & Bourgeois &
72     & Six line Pica \\
10     & Long Primer &
%%16     & Columbian \\
%%20     & Paragon \\
%%22     & Double Small Pica \\
%%32     & Four Line Brevier \\
%%40     & Double Paragon \\
%%44     & Meridian \\
96     & Eight Line Pica \\ \bottomrule
\end{tabular}
\end{table}



    The typographers' and printers' term for the vertical space between
the lines of normal text is \emph{leading}\index{leading}, which is also
usually expressed in points and is usually larger than the font size.
A convention for describing the font and leading is to give the font size 
and leading separated by a slash; for instance $10/12$ for a
\U{10}{pt} font set with a \U{12}{pt} leading, or $12/14$ for a \U{12}{pt} font set with a
\U{14}{pt} leading.

    The normal length of a line of text is often called the 
\emph{measure}\index{measure} and is normally specified in terms of
picas\index{pica} where 1 pica equals 12 points (\U{1}{pc} = \U{12}{pt}).

    Documents may be described as being typeset with a particular font
with a particular size and a particular leading on a particular measure;
this is normally given in a shorthand form. 
A \U{10}{pt} font with \U{11}{pt} leading on a \U{20}{pc} measure is described as
\abyb{10/11}{20}, and \abyb{14/16}{22} describes a \U{14}{pt} font
with \U{16}{pt} leading set on a a \U{22}{pc} measure.

\section{Units of measurement}

    Typographers and printers use a mixed system of units, some of which
we met above. The fundamental unit is the point; \tref{tab:units} lists 
the most common units employed.

\begin{table}
\centering
\caption{Printers units} \label{tab:units}
\begin{tabular}{l r @{\,=\,} l   } \toprule
  Name (abbreviation) & \multicolumn{2}{c}{Value}
  \\ 
  \midrule
  point (pt)\index{point}\index{pt}          &   \multicolumn{2}{c}{}         
  \\
  pica (pc)\index{pica}\index{pc}           & \U{1}{pc} & \U{12}{pt} 
  \\
  inch (in)\index{inch}\index{in}           & \U{1}{in} & 72.\U{27}{pt} 
  \\
  centimetre (cm)\index{centimetre}\index{cm}     & 2.\U{54}{cm} & \U{1}{in} 
  \\
  millimetre (mm)\index{millimetre}\index{mm}     & \U{10}{mm} & \U{1}{cm} 
  \\ 
  big point (bp)\index{big point}\index{bp}      & \U{72}{bp} & 72.\U{27}{pt} 
  \\
  didot point (dd)\index{didot point}\index{dd}    & \U{1157}{dd} & \U{1238}{pt} 
  \\
  cicero (cc)\index{cicero}\index{cc}         & \U{1}{cc} & \U{12}{dd} 
  \\
  \bottomrule
\end{tabular}
\end{table}

    Points\index{point} and picas\index{pica} 
are the traditional printers units used in English-speaking countries. 
The didot point\index{didot point} and cicero\index{cicero} are the 
corresponding units used in continental Europe. In Japan `kyus'\index{kyus}
(a quarter of a millimetre) may be used as the unit of measurement.
Inches\index{inch} and centimetres\index{centimetre} are the units that we
are all, or should be, familiar with.

    The point system was invented by Pierre Fournier le jeune in 1737 with
a length of \U{0.349}{mm}. Later in the same century Fran\c{c}ois-Ambroise Didot
introduced his point system with a length of \U{0.3759}{mm}. This is the value
still used in Europe. Much later, in 1886, the American Type Founders
Association settled on \U{0.013837}{in} as the standard size for the point, and
the British followed in 1898. Conveniently for those who are not entirely
metric in their thinking this means that 
six picas are approximately equal to one inch.

    The big point\index{big point} 
is somewhat of an anomaly in that it is a recent
invention. It tends to be used
in page markup languages, like \pscript\footnote{\pscript{} is a 
registered trademark of Adobe Systems Incorporated.\label{fn:ps}},
in order to make calculations quicker and easier.

    The above units are all constant in value. There are also some units
whose value depends on the particular font\index{font} being used. 
The \textit{em}\index{em}
is the nominal height of the current font; it is used as a width measure.
An \textit{en}\index{en} is half an em.
The \textit{ex}\index{ex} is
nominally the height of the letter `x' in the current font. You may also
come across the term \textit{quad}\index{quad}, often as in a phrase
like `starts with a quad space'. It is a length defined in terms of
ems; a quad is \U{1}{em}.


\cleardoublepage
%\pagenumbering{arabic}

% body
\mainmatter

%%%%%%\part{Art and Theory} \label{part:art} 
%%%%%%%%%%%%%%%%%%%%%%%%%%%%%%%%%%%%%%%%%%%%%


\PWnote{2009/02/02}{Added Historical background chapter}
\chapter{Historical background} \label{chap:history}

\section{Galloping through the millenia}

    The earliest known writing dates back to the Sumerians around 3300\textsc{bc}
who used pointed sticks or reeds to impress marks into wet clay tablets
that were subsequently dried. The result is what we call 
Cuneiform\index{cuneiform}.\footnote{From the Latin \emph{cuneus} meaning wedge.}
For the next several thousand years all texts were produced, one way or 
another, individually by hand.

    The earliest printed book known is a 9th century Chinese woodblock
printing of the \emph{Diamond Sutra}. In this technique the complete text
for a page is carved on a wooden block which is then used to impress
the ink onto the paper. Once the woodblocks were available many copies 
of the text could be produced very quickly.

    The Chinese were perhaps the first to print using moveable 
type\index{type} where 
the individual characters were engraved on wood blocks so they could be 
reused for different texts. In his \emph{Writings Beside the Meng Creek}
the Song Dynasty essayist Shen Kuo\index{Shen Kuo} (1031--1095) 
described how Bi Sheng\index{Bi Sheng} during the reign of 
Chingli (1041--1048) printed
from moveable type that he made from baked clay, which was rather fragile.
Somewhat later Wang Zhen\index{Wang Zhen} (c. 1290--1333) improved the 
process by using
wooden type. These never became particularly popular methods because 
of the thousands of different characters that a printing house might need.
By 1230 the Chinese used moveable metal type for printing. None of this was 
known outside Asia.

    In the West books and manuscripts were hand written by scribes, 
although some
small items, like playing cards or depictions of saints, were printed
from woodblocks. Then Johannes\index{Gutenberg, Johannes} Gutenberg 
(c. 1398--1468) of Mainz invented printing using moveable type 
around about 1440--1450.\footnote{Others have been put forward as the 
inventor, notably a Dutchman named Coster, but the preponderance of 
opinion favours Gutenberg.}
 He had to experiment to determine the formula for
a suitable ink and also to develop a good metal alloy\index{typemetal} 
for the type itself.
He came up with lead to which he added antimony for strength and hardness and
tin for toughness.\footnote{This is still the basis for metal type today;
Monotype casting machines use lead with 15--24\% antimony and 6--12\% tin.}

    In order be successful in the market Gutenberg had to produce books
that equaled those produced by the scribes, except that they did not
have to be decorated so lavishly. The scribes used many ligatures,
contractions, and other techniques in order to have justified text with
no raggedy edges. To compete with them his font for the famous 42-line 
Bible, published around 1455, consisted of some 290 characters though all
the text was in Latin which requires a basic character set of only forty 
letters --- twenty lowercase letters and twenty caps --- plus
some punctuation marks.

    The 42-line Bible is set in two columns of 42 lines each. It is believed
that about 135 copies were printed on paper and 40 on vellum. The page size was
12 by 16\slashfrac{1}{2} inches and it is estimated that more than five 
thousand calfskins were required for the vellum copies.

    The new technology spread rapidly. In 1465 
Konrad\index{Sweynham, Konrad} Sweynheym and Arnold\index{Pannartz, Arnold}
Pannartz set up a printing shop in the monastery at Subiaco, east of Rome.
There was printing in K\"{o}ln in 1466 and in 1468 in Augsberg and Rome itself.
The first Venetian printer was Johann\index{Speyer, Johann van} van Speyer 
who started work in 1469. A year later printing was established at the 
Sorbonne and Nicolas\index{Jenson, Nicolas} Jenson
had his press in Venice. Printing was introduced into Spain at Valencia 
in 1474. William\index{Caxton, William} Caxton started printing in England 
in 1476 setting up a press at The Sign of the Red Pale in Westminster, 
near the Abbey; Theoderic\index{Rood, Theoderic} Rood was printing in 
Oxford between 1478 and 1485 and John\index{Sieberch, John} Sieberch at 
Cambridge in 1520.

    The German printers kept with the initial gothic
style of Gutenberg's type and Caxton used a cursive bastarda gothic. The
Italians and other Europeans, though, moved to a roman type, based on the
humanist bookhands, for their
work. In 1471 Jenson produced the first full set of Greek type, which still
remains one of the best. Aldus\index{Manutius, Aldus} Manutius, printing in 
Venice, introduced the italic type in 1500.

   The early printers were jack of all trades. They had to make their presses,
design, make and cast their type\index{type}, 
and print and sell the results. As time went
on typemaking and printing became separate crafts. It became possible to 
purchase the materials and equipment for printing but the printer was still
the book designer.

   For four centuries setting the type for printing was done by hand until 
the introduction of Ottmar Mergenthaler's\index{Mergenthaler, Ottmar} 
Linotype\index{Linotype} machine in 1886. The operator
sat at a keyboard, typing the text line by line and the machine produced a 
corresponding solid line of type. The disadvantage was when an error needed
correcting at least one whole new line of type was needed, or two or more if
the correction spilled over the end of the line, or even more if it continued
onto the following page. The competing 
Monotype\index{Monotype} machine, invented by Tolbert\index{Lanston, Tolbert} 
Lanston, was first available in 1896. This was operated via a keyboard which
produced a punched paper tape which was fed to the caster which produced
lines of type composed of individual pieces. Correcting typos was easier
because individual characters could be added or replaced. On the other hand,
Linotype output was easier to handle if complete sections had to be moved
around, for example for `quick' printing such as a daily newspaper. 

    Alan Bartram~\autocite{BARTRAM01} shows examples of book designs from between
1470 and 1948, not all of which he considers to be good. Examples of printed
pages from the 15th to the 20th century are in the TUG 2007 San Diego Meeting
keynote presentation~\autocite{TUGKEYNOTE07}.


\section{Making type}

\typesubidx{manufacture|(}
    This is a very brief description of how lead type is made. For
a good overview see~\autocite{CHAPPELL99} and Fred Smeijers~\autocite{SMEIJERS96}
provides a detailed description of punchcutting.
 
Making type has been an inherently manual process. Having got a design
for a font, for each character, a punchcutter makes a punch starting
with a square steel bar about 2\slashfrac{1}{4}~inches (\U{6}{cm}) long with
an end face large enough to encompass the character. Using files and
gravers, and perhaps some specialized tools like a counterpunch, he
carves out the character in relief on one end of the bar. The
character is oriented so that it is backwards with respect to its
appearance when printed.  To check the shape, the end of the punch is
put into the flame of an alcohol lamp which coats it with lampblack,
and it is then pressed against a chalky paper to leave a black image
of the character. Once the shape is correct the punch is hardened and
annealed.

   The next stage is to create the matrix for the character. The punch is
hammered into a softer material, usually copper, or sometimes brass which
is harder but lasts longer. At this point the character is in the same
orientation as printed but is a negative impression in the matrix.

   The matrix is then put into a casting box and molten typemetal poured in.
Once it has hardened and removed from the mould the new piece of type is 
dressed to the same length as all the other pieces for the font. Many, many 
pieces of type can be cast from one matrix, and if the punch is retained new
matrices can be made. Typically one buys the lead type from a typecasting
company, and a typecasting company would purchase matrices from the type
design company. Of course, in the early days these were all the same 
organisation and only as the centuries passed did they tend to become
separated.

   The Linotype\index{Linotype} and Monotype\index{Monotype} machines 
require the matrices but cast the type only when needed. After use the type 
from these machines is melted down and reused time and time again.
\typesubidx{manufacture|)}

\section{Book types}

\PWnote{2009/04/25}{Added longish section on book types}
\PWnote{2009/04/25}{Used endnotes in book types section}

    Roughly speaking, there are two kinds of printing type\index{type};
one, called 
in general \emph{book type}\index{book type}, 
is what is used for setting longer pieces 
of text such as a poem or a book, or other material meant for continuous
reading. The other, called \emph{display type}\index{display type}, is used 
for pretty much everything else, such as company names, posters,
advertisements, ephemera and sometimes even book titles, all of which are 
short pieces of text, often intended to catch your eye. There are a multitude
of display types, some of them almost illegible. Here I want to say a little
bit about book types.

\begin{table}
\centering
\caption{Broad typeface categories}\label{tab:typecat}
\begin{tabular}{lllll} \toprule
Typeface & & Lawson & Bringhurst & Vox \\ \midrule
\centaurx Centaur\facesubseeidx{Centaur}  & & 
\centaurx Venetian\typesubidx{Venetian} & 
\centaurx Renaissance\typesubidx{Renaissance}  & 
\centaurx Humanist\typesubidx{Humanist} \\
\bembox Bembo\facesubseeidx{Bembo}    & & 
\bembox Aldine/French\typesubidx{Aldine/French} & 
\bembox Renaissance\typesubidx{Renaissance}  & 
\bembox Garald\typesubidx{Garald} \\
%%%Sabon    & & Aldine/French OS & Renaissance  & Garald \\
%%%Garamond & & Aldine/French OS & Renaissance  & Garald \\
\garamondx Garamond\facesubseeidx{Garamond} & & 
\garamondx Aldine/French\typesubidx{Aldine/French} & 
\garamondx Baroque\typesubidx{Baroque}      & 
\garamondx Garald\typesubidx{Garald} \\
\caslonx Caslon\facesubseeidx{Caslon}   & & 
\caslonx Dutch/English\typesubidx{Dutch/English} & 
\caslonx Baroque\typesubidx{Baroque}      & 
\caslonx Garald\typesubidx{Garald} \\
\baskervillex Baskerville\facesubseeidx{Baskerville} & & 
\baskervillex Transitional\typesubidx{Transitional} & 
\baskervillex Neoclassical\typesubidx{Neoclassical}  & 
\baskervillex Transitional\typesubidx{Transitional} \\
\bellx Bell\facesubseeidx{Bell}     & & 
\bellx Transitional\typesubidx{Transitional} & 
\bellx Rationalist\typesubidx{Rationalist}   & 
\bellx Transitional\typesubidx{Transitional} \\
\bodonix Bodoni\facesubseeidx{Bodoni}   & & 
\bodonix Modern\typesubidx{Modern} & 
\bodonix Romantic\typesubidx{Romantic}  & 
\bodonix Didone\typesubidx{} \\
\clarendonx Clarendon\facesubseeidx{Clarendon} & & 
\clarendonx Square Serif\typesubidx{Square Serif} & 
\clarendonx Realist\typesubidx{Realist} & 
\clarendonx Mechanistic\typesubidx{Mechanistic} \\
\futurax Futura\facesubseeidx{Futura}    & & 
\futurax Sans-serif\typesubidx{Sans-serif} & 
\futurax Geometric Modernist\typesubidx{Geometric Modernist} & 
\futurax Lineal Geometric\typesubidx{Lineal Geometric} \\
\optimax Optima\facesubseeidx{Optima}    & & 
\optimax Sans-serif\typesubidx{Sans-serif} & 
\optimax Neoclassical\typesubidx{Neoclassical} & 
\optimax Lineal Humanist\typesubidx{Lineal Humanist} \\
\gillsansx Gill Sans\facesubseeidx{Gill Sans} & & 
\gillsansx Sans-serif\typesubidx{Sans-serif} & 
\gillsansx Geometric Humanist\typesubidx{Geometric Humanist} & 
\gillsansx Lineal Humanist\typesubidx{Lineal Humanist} \\
\bottomrule
\end{tabular}
\end{table}


    There are several ways of categorizing typefaces, three of which are
shown in \tref{tab:typecat}. The listed schemes are
\begin{description}
\item[Lawson] from 
Lawson \& Agner~\autocite{LAWSONAGNER90} who proposed  
\emph{a Rational System} based on the historical sequence.
\item[Bringhurst] who categorizes according to the artistic and architectural
period that a typeface can be said to represent~\autocite{BRINGHURST99}.
\item[Vox] devised a system that has been adopted as a British Standard 
(BS 2961: 1967). This tried to be language-neutral and get away from
the more traditional descriptions such as gothic, antique, grotesque, and 
modern which have different, and somtimes opposite, meanings in different
languages~\autocite{MCLEAN80}.
\end{description}
Later I will expand on the Lawson \& Agner system and show some types 
corresponding to some of their categories. I have limited the examples to 
those types which are included in a modern \ltx\ 
distribution\pagenote[modern \ltx\ distribution]{An example of a modern 
distribution is TeXlive 2008. There are many more fonts that could
be used but I wanted \ltx ers to be able to run the manual's source through
\ltx\ themselves without having to install any extra fonts. 
\emph{The \ltx\ Font Catalogue} (\url{http://www.tug.dk/FontCatalogue}) has 
examples of 181 different fonts and these do not include many of the
more specialized ones, such as for setting chess or for archaic languages
(e.g., cuneiform, egpytian hieroglyphs, etc.), or scripts for Asian languages.}, 
which unfortunately does not include types corresponding to all the 
categories.


\subsection{Type-related terminology}

    First, though, some typographical terms related to types, and illustrated 
in \fref{fig:typeterms}\pagenote[illustrated in \fref{fig:typeterms}]{The \ltx\
fontfamily names for the typefaces used in the illustration are: 
\begin{center}
\begin{tabular}{llcll}
Antiqua Turin\facesubseeidx{Antiqua Turin} & \pfontfam{antt} & &
Avant Garde\facesubseeidx{Avant Garde} & \pfontfam{pag} \\
Bera Serif\facesubseeidx{Bera Serif} & \pfontfam{fve} & &
Bookman\facesubseeidx{Bookman} & \pfontfam{pbk} \\
GFS Bodoni\facesubseeidx{GFS Bodoni} & \pfontfam{bodoni} & &
GFS Didot\facesubseeidx{GFS Didot} & \pfontfam{udidot} \\
Times Roman\facesubseeidx{Times Roman} & \pfontfam{ptm} & &
Utopia\facesubseeidx{Utopia} & \pfontfam{put} \\
\end{tabular}
\end{center}}. 

\begin{figure}
\centering
\begin{tabular}{cccc} \toprule
Bracketed serif & Unbracketed serif & Square serif & Sans serif \\
\usethisfont{25}{30}{pbk}{m}{n} H & 
\usethisfont{25}{30}{antt}{m}{n} H &
\usethisfont{25}{30}{fve}{m}{n} H &
\usethisfont{25}{30}{pag}{m}{n} H \\
\itshape Bookman\facesubseeidx{Bookman} & 
\itshape Antiqua Turin\facesubseeidx{Antiqua Turin} & 
\itshape Bera Serif\facesubseeidx{Bera Serif} & 
\itshape Avant Garde\facesubseeidx{Avant Garde} \\ \midrule
Inclined axis & Vertical axis & Gradual contrast & Abrupt contrast \\
\usethisfont{25}{30}{antt}{m}{n} O &
\usethisfont{25}{30}{ptm}{m}{n} O &
\usethisfont{25}{30}{pbk}{m}{n} N U &
\usethisfont{25}{30}{udidot}{m}{n} N U \\
\itshape Antiqua Turin\facesubseeidx{Antiqua Turin} & 
\itshape Times Roman\facesubseeidx{Times Roman} & 
\itshape Bookman\facesubseeidx{Bookman} &
\itshape GFS Didot\facesubseeidx{GFS Didot} \\ \midrule
small counter & large counter & small counter & large counter \\
\usethisfont{25}{30}{pbk}{m}{n} e &
\usethisfont{25}{30}{fve}{m}{n} e &
\usethisfont{25}{30}{pbk}{m}{n} a &
\usethisfont{25}{30}{antt}{m}{n} a \\
\itshape Bookman\facesubseeidx{Bookman} & 
\itshape Bera Serif\facesubseeidx{Bera Serif} & 
\itshape Bookman\facesubseeidx{Bookman} & 
\itshape Antiqua Turin\facesubseeidx{Antiqua Turin} \\ \midrule
separate & ligatured & separate & ligatured \\
\termfont{put}{m}{n} f{}i f{}l & 
\termfont{put}{m}{n} fi fl &
\termfont{bodoni}{m}{n} a{}e o{}e &
\termfont{bodoni}{m}{n} \ae{} \oe{} \\
\itshape Utopia\facesubseeidx{Utopia} & 
\itshape Utopia\facesubseeidx{Utopia} & 
\itshape GFS Bodoni\facesubseeidx{GFS Bodoni} & 
\itshape GFS Bodoni\facesubseeidx{GFS Bodoni} \\ \bottomrule
\end{tabular}
\caption{Examples of some typographical type-related terms} \label{fig:typeterms}
\end{figure}


\begin{labelled}{itlabel}
\item[Serif:] The\index{serif} cross stroke that finishes the stems or arms of letters.
\item[Bracketed serif:] A\index{bracketed serif} serif that transitions gradually into 
     the stem it is attached to.
\item[Unbracketed serif:] A\index{unbracketed serif} serif with a sharp break between 
      it and the stem.
\item[Square serif:] A\index{square serif} rectangular serif with squared ends.
\item[Sans serif:] Without\index{Sans-serif} serifs.
\item[Axis:] The\index{axis} direction of the hypothetical line joining the thinnest 
             parts of a letter like `O'. It is related to the angle that
             a broad nibbed pen would be held in order to replicate the
             inner and outer contours.\footnote{The axis and angle are 
             perpendicular to each other.}
\item[Contrast (also called shading):] The\index{contrast}\index{shading} 
              difference between the thick and thin strokes.
\item[Counter:] The\index{counter (of a letter)} white space enclosed by a letter, 
                whether open or closed.
                Sometimes used to refer to the closed part of letters such
                as `a' or `e', which may also be referred to as the 
               \textit{eye}\index{eye}.
\item[Ligature:] The\index{ligature} conjoining of two (or more) letters, 
                 usually with a change of shape.
\end{labelled}


For more detailed descriptions and further terms
you may wish to consult other sources, such as 
\autocite{BRINGHURST99,LAWSONAGNER90,MEGGS00}. If you are interested in the
subtle, and the not so subtle, differences between typefaces then Karen 
Cheng's \btitle{Designing Type}~\autocite{CHENG05} has a great deal to offer.

    The names of typefaces can be confusing; different suppliers have a
tendency to give different names to the same underlying typeface. For
example Goudy's\index{Goudy, Frederic} University of California Old 
Style\facesubseeidx{University of California Old Style} can also be found
as Californian\facesubseeidx{Californian}, 
University Old Style\facesubseeidx{University Old Style}, 
Berkely Old Style\facesubseeidx{Berkely Old Style}, 
and possibly under other names as well, all more or less adhering to the 
original design.\pagenote[more or less adhering to the original design]{The
final printed character depends not only on the geometric shape but also on
the printing technology used, the ink, and the paper. With letterpress printing
where the inked metal type is impressed into the paper, the ink tends to spread
just a little bit; all other things being equal the spread depends on the 
type and amount of ink, the hardness of the paper, the surface finish of the 
paper, the amount of pressure applied, in some cases the dampness of the
paper, and so on. To get a similar looking result using offset lithography
where the ink stays where it is put, 
the geometric shape must be changed to simulate the ink spread of the 
letterpress process, but then the question arises as to which of the many
letterpress impressions is the one to be simulated? Different designers and
different manufacturers have different ideas about this.}


\subsection{Blackletter}

    The first type was \emph{Gothic}\typesubidx{Gothic}, or 
\emph{Blackletter}\typesubidx{Blackletter}, used by 
Gutenberg\index{Gutenberg, Johannes} which was based on
the kind of script that the scribes were using at that time (c. 1455). 
It remained in
fashion in Germany until towards the end of the last century, and is still
often used for the names of newspapers. Elsewhere, starting in Italy, 
it was replaced by the \emph{Roman}\typesubidx{Roman} type. 

    There are several kinds of blackletter type. The first is 
\emph{Textura}\typesubidx{Textura} where the characters are squarely 
drawn without any curves
and are the kind that Gutenberg\index{Gutenberg, Johannes} used. 
In the scribal tradition from which these came the idea was that the words
created a uniform texture along each line and down each page. To modern eyes it
is difficult to tell one letter from another. Two modern versions are
Goudy Text\facesubseeidx{Goudy Text} and 
Cloister Black\facesubseeidx{Cloister Black}.

    Another grouping is \emph{Rotunda}\typesubidx{Rotunda} where the letters are 
more rounded than Textura and are easier to read. A modern example is 
Goudy Thirty\facesubseeidx{Goudy Thirty}.

    The last subdivision is \emph{Bastarda}\typesubidx{Bastarda} which has been
the common type used in Germany for many a year. The most common form
is \emph{Fraktur}\typesubidx{Fraktur}, first cut in the sixteenth century, which
is a lighter and more open version of Textura\typesubidx{Textura} and so easier
to read. Many newspapers use a Fraktur type for their headline. An
example of a \emph{Fraktur}\typesubidx{Fraktur}
\pagenote[An example of a \emph{Fraktur}]{The typeface was originally created
by Yannis Haralambous\index{Haralambous, Yannis} and is accessed in \ltx\ 
as the \pfontfam{yfrak} fontfamily.}
is shown in \fref{fig:fraktur}.%

\begin{figure}
\centering
{\centering\fontfamily{yfrak}\selectfont
  Blackletter --- Fraktur \\
  \UCalphabet \\
  \LCalphabet \\
  \fox\par}
\caption{An example of the Fraktur style of Blackletter types} \label{fig:fraktur}
\end{figure}

\subsection{Oldstyle}

\subsubsection{Venetian}

    Early roman types, based on the humanist scribal hand, were cut by 
Sweynheym \& Pannartz in the Rome area (c. 1467). In Venice Nicolas
Jenson\index{Jenson, Nicolas} cut what is considered to be the first, 
and one of the best, romans
(c. 1471). His types have been widely reproduced and copied and the style
is known as \emph{Venetian}\typesubidx{Venetian}. Some modern day Venetians
include Cloister\facesubseeidx{Cloister}, Eusebius\facesubseeidx{Eusebius} (originally
called Nicolas Jenson) and  Venezia\facesubseeidx{Venezia}; 
Bruce Rogers'\index{Rogers, Bruce} Centaur\facesubseeidx{Centaur} is an elegant 
modernized Venetian.

    The characteristics of the Venetian types include uneven or slightly
concave serifs, there is minimal contrast between the thick and thin 
strokes, and an inclined axis. The crossbar of the lowercase `e' is slanted 
upwards. On some
capitals, principally `N' and `M', there are slab serifs that extend
across the tops of the vertical strokes.
  
    William Morris\index{Morris, William} chose Jenson's\index{Jenson, Nicolas} 
type as the model
for his Golden Type\facesubseeidx{Golden Type}, cut by Edward 
Prince\index{Prince, Edward} around 1890. This started the revival of the
Venetian types. 

    The first type generally available was Morris Benton's\index{Benton, Morris}
Cloister Oldstyle\facesubseeidx{Cloister Oldstyle}. Other modern Venetians include 
Goudy's\index{Goudy, Frederic} Kennerly\facesubseeidx{Kennerly}, 
Deepdene\facesubseeidx{Deepdene} and Californian\facesubseeidx{Californian}, which is now
called Berkely Old Style\facesubseeidx{Berkely Old Style}. One, perhaps the best,
is Bruce Rogers'\index{Rogers, Bruce} Centaur\facesubseeidx{Centaur} which he created
in 1914.


\subsubsection{Aldine/French}
\typesubidx{Aldine/French}\typesubidx{Aldine}\typesubidx{French}

    Another of the many printers in Venice, 
Aldus Manutius\index{Manutius, Aldus}, wanted a type
that was less related to the pen-drawn scribal characters. Aldus employed
Francesco Griffo da Bologna\index{Griffo, Francesco} to cut two types
for him.  The first, cut in 1497, was for an edition of \emph{De Aetna} 
by the humanist scholar Pietro Bembo\index{Bembo, Pietro} --- the modern
version of this is called Bembo\facesubseeidx{Bembo}. Griffo also cut another
variation on Jenson's\index{Jenson, Nicolas} roman and which soon superseded it 
in popularity.
It was first used in the famous \emph{Hypnerotomachia Poliphili} by
Francesco Colonna\index{Colonna, Francesco} which Aldus published in 1499.
A modern version is available called Poliphilus\facesubseeidx{Poliphilus}.

The Aldine roman soon spread across Europe. One of the first type
cutters to use it as a model was Claude Garamond\index{Garamond,
  Claude} in Paris (c. 1540), and his types had a wide distribution,
for example being used in Antwerp by Christopher
Plantin\index{Plantin, Christopher}. The main basis for modern
versions is a version of Garamond's types cut by the French printer
Jean Jannon\index{Jannon, Jean} about 1621.

    Characteristics of these types are wide concave serifs, particularly
on the capitals, which are narrower than the Venetians and may be not as high
as the lowercase ascenders. The crossbar of the lowercase `e' is horizontal, 
as opposed to the slanted crossbar of the Venetians. There is an inclined axis
and a medium contrast between the thick and thin strokes.

    Some modern Aldine/French Oldstyle types are 
Bembo\facesubseeidx{Bembo}, 
Estienne\facesubseeidx{Estienne}, 
Garamond\facesubseeidx{Garamond}, 
Geraldus\facesubseeidx{Geraldus}, 
Granjon\facesubseeidx{Granjon},
Palatino\facesubseeidx{Palatino},  
Poliphilus\facesubseeidx{Poliphilus}, 
and Sabon\facesubseeidx{Sabon}.
An example of Palatino\facesubseeidx{Palatino}%
\pagenote[An example of Palatino]{In \ltx\ the Palatino\index{Palatino} 
font is accessed as the \pfontfam{ppl} fontfamily.}, 
which was created by Hermann Zapf\index{Zapf, Hermann}
in 1950, is shown in \fref{fig:palatino}.

\begin{figure}
\centering
{\centering\fontfamily{ppl}\selectfont
  Oldstyle Aldine/French --- Palatino \\
  \UCalphabet \\
  \LCalphabet \\
  \fox\par}
\caption{An example of an Oldstyle Aldine/French type: Palatino} 
   \label{fig:palatino}
\end{figure}


\subsubsection{Dutch/English}
\typesubidx{Dutch/English}\typesubidx{Dutch}\typesubidx{English}

    During the sixteenth century the French types were popular throughout
Europe but then the pendulum swung towards types from the Low Countries.
The Dutch were principally traders and their printing style became 
increasingly known. They produced types that were more practical for
commercial printing. The contrast between thick and thin strokes increased
and the serifs straightened.

The English type cutter William Caslon\index{Caslon, William} (1692--1766) 
cut a famous face of this kind that has been used ever since throughout the 
world; in America the first printed version of the 
\btitle{Declaration of Independence}
was set with Caslon type.

    Modern Dutch/English types include
Caslon\facesubseeidx{Caslon} (of course),
and Janson\facesubseeidx{Janson}.

\subsection{Transitional}
\typesubidx{Transitional}

    Transitional types are those based on the Oldstyle\typesubidx{Oldstyle} 
types but with features of the style called Modern\typesubidx{Modern}. 

    By the end of the sixteenth century the quality of printing in Italy 
and France had fallen off from when Claude Garamond\index{Garamond, Claude} 
was working. In 1692
King Louis \textsc{xiv} ordered a new set of types for the Royal Printing
House. In a lengthy report the Academy of Sciences recommended a roman type
constructed on mathematical principles. 
Lucien Grandjean\index{Grandjean, Lucien} who cut the
new Romain du Roi\facesubseeidx{Romain du Roi} allowed his type cutter's eye to 
sometimes overrule the 
academicians to the betterment of the result.

    Grandjean's type was copied by many others and effectively replaced 
Oldstyle in Europe. Pierre Simon Fournier\index{Fournier, Pierre Simon} 
(1712--1768) started his typecutting 
business in 1737, cutting over eighty types in twenty-four years. These were
based on Garamond's types but influenced by Grandjean's work. The result was
the first intimations of the Transitional types.

    The changes on the continent had had little impact in England, but John
Baskerville\index{Baskerville, John}, about 1750, set up a printing shop 
in Birmingham and created the type that bears his name. Some consider this
to be the real beginning of the Transitionals. Baskerville's work was
disliked in England but was siezed on with alacrity on the Continent.

    Some modern day Transitional types are 
Baskerville\facesubseeidx{Baskerville}, 
Bell\facesubseeidx{Bell}, 
Fournier\facesubseeidx{Fournier},
Georgian\facesubseeidx{Georgian},
and URW Antiqua\facesubseeidx{URW Antiqua}\pagenote[and URW Antiqua]{The URW
Antiqua\index{URW Antiqua} font in \ltx\ is known as the \pfontfam{uaq} fontfamily.} which is shown in \fref{fig:antiqua}.

\begin{figure}
\centering
{\centering\fontfamily{uaq}\selectfont
  Transitional --- URW Antiqua \\
  \UCalphabet \\
  \LCalphabet \\
  \fox\par}
\caption{An example of a Transitional type: URW Antiqua} 
   \label{fig:antiqua}
\end{figure}

\begin{figure}
\centering
{\centering\fontfamily{pnc}\selectfont
  Transitional (newspaper) --- New Century Schoolbook \\
  \UCalphabet \\
  \LCalphabet \\
  \fox\par}
\caption{An example of a Transitional newspaper type: New Century Schoolbook} 
   \label{fig:newcent}
\end{figure}

    Much later, around the end of the nineteenth century, another kind of 
Transitional type was introduced, designed for legibility for newspapers 
when printed on high speed presses. The counter spaces were open, the 
serifs were even and strongly bracketed and with a high x-height.
Examples are Century\facesubseeidx{Century} designed by 
 Linn Boyd Benton\index{Benton, Linn Boyd} in 1895 for the \emph{Century}
magazine, and 
Cheltenham\facesubseeidx{Cheltenham} by Betram Goodhue\index{Goodhue, Bertram} 
in 1896, which has become one of the printers' standard types. Stanley
Morison's\index{Morison, Stanley} Times Roman\facesubseeidx{Times Roman}, 
which he
designed for \emph{The Times} of London fits into the Transitional
classification. The general characteristics include vertical, or nearly
vertical axis, more pronounced contrast compared with the Oldstyle faces, but
nowhere nearly as pronounced as the later Didot types. Some have finely, or
unbracketed, serifs.

    A version of one of the Century 
series\pagenote[version of one of the Century series]{In \ltx\ the New 
Century Schoolbook\facesubseeidx{New Century Schoolbook} font
is known as the \pfontfam{pnc} fontfamily.} of typefaces, 
New Century\facesubseeidx{New Century Schoolbook} Schoolbook, designed
by Morris Benton\index{Benton, Morris},
is shown in \fref{fig:newcent}.

\subsection{Modern}
\typesubidx{Modern}

    Modern in this case means with respect to Transitional\typesubidx{Transitional} 
and applies to a style of type introduced in the eighteenth century.

    Grandjean's Romain du Roi\facesubseeidx{Romain du Roi} had started a trend 
in which the contrast 
between thick and thin strokes gradually increased. Following Baskerville's
type, Giambattista Bodoni\index{Bodoni, Giambattista} in Italy and the 
Didot\index{Didot} foundry in France increased the contrast 
to extreme limits with the thin strokes degenerating into hairlines.

\begin{figure}
\centering
{\centering\fontfamily{udidot}\selectfont
  Modern --- GFS Didot \\
  \UCalphabet \\
  \LCalphabet \\
  \fox\par}
\caption{An example of a Modern type: GFS Didot} 
   \label{fig:didot}
\end{figure}

A modern type, GFS Didot\facesubseeidx{GFS Didot}\pagenote[GFS Didot]{The
\ltx\ fontfamily for GFS Didot\index{GFS Didot} is \pfontfam{udidot}.},
after the style of 
Didot\index{Didot} is illustrated in \fref{fig:didot}.

    The general characteristics are vertical axis, exaggerated contrast, 
and flat, unbracketed, serifs.
    
\subsection{Square Serif}
\typesubidx{Square Serif}

    The Victorian printers found that they needed new type forms that
would work better than the traditional romans when used with the new 
and faster breeds of printing presses, and particularly for use in
commercial printing and advertising. 

\begin{figure}
\centering
{\centering\fontfamily{fve}\selectfont
  Square Serif --- Bera Serif \\
  \UCalphabet \\
  \LCalphabet \\
  \fox\par}
\caption{An example of a Square Serif type: Bera Serif} 
   \label{fig:beraserif}
\end{figure}

    The types went out of fashion during the first half of the twentieth
century but have since become more popular with the 
Clarendon\facesubseeidx{Clarendon}
type. An example of Square Serif\pagenote[An example of Square Serif]{The 
Bera Serif's\index{Bera Serif} fontfamily name in \ltx\ is \pfontfam{fve}.}
is shown in \fref{fig:beraserif}.


\subsection{Sans-serif}
\typesubidx{Sans-serif}

    Sans-serif types were first created around 1830. In England they were
called Grotesques\index{Grotesques} and in America Gothics\index{Gothics}. 
Around 1920 there appeared artistic
schools such as Expressionism, Constructivism and Cubism. These had a
marked effect on typographic styles and the Sans-serif types experienced
a great burst of popularity, seeming to express `modern' ideas. They tend to
be geometric in form as opposed to the curvaceous romans.

\begin{figure}
\centering
{\centering\fontfamily{fvs}\selectfont
  Sans-serif --- Bera Sans \\
  \UCalphabet \\
  \LCalphabet \\
  \fox\par}
\caption{An example of a Sans-serif type: Bera Sans} 
   \label{fig:berasans}
\end{figure}

    Examples of modern Sans-serifs are 
Helvetica\facesubseeidx{Helvetica}, 
Futura\facesubseeidx{Futura},
and, famously, Gill Sans\facesubseeidx{Gill Sans}. 
Yet another sans\pagenote[Yet another sans]{\pfontfam{fvs} is the \ltx\
fontfamily name name for the Bera Sans\index{Bera Sans} font.}, 
Bera Sans\facesubseeidx{Bera Sans}, is shown in \fref{fig:berasans}.


\subsection{Script/Cursive}
\typesubidx{Script/Cursive}\typesubidx{Script}\typesubidx{Cursive}

    This is a very broad category but 
essentially the forms are closer
to handwriting rather than printing. Some are based on letter forms created
by drawing with a brush while others are based on forms written using
a pen. In general they have an informal
presence but some of the latter kind are used in formal settings such 
as wedding invitations.

\begin{figure}
\centering
{\centering%\fontfamily{pbsi}\selectfont
  \usethisfont{10}{12}{pbsi}{xl}{n}
  Brush  --- Brush Script \\
  \UCalphabet \\
  \LCalphabet \\
  \fox\par}
\caption{An example of a Script/Cursive Brush type: Brush Script} 
   \label{fig:brush}
\end{figure}

\begin{figure}
\centering
{\centering\fontfamily{pzc}\selectfont
  Calligraphic --- Zapf Chancery \\
  \UCalphabet \\
  \LCalphabet \\
  \fox\par}
\caption{An example of a Script/Cursive Calligraphic type: Zapf Chancery} 
   \label{fig:chancery}
\end{figure}

    Figure~\ref{fig:brush} shows a brush-based script 
unimaginatively\pagenote[unimaginatively]{The Brush Script\index{Brush Script}
font is accessible as the \pfontfam{pbsi} fontfamily.} called 
Brush Script\facesubseeidx{Brush Script} while a
calligraphic script, Hermann Zapf's\index{Zapf, Hermann} fine
Zapf Chancery\facesubseeidx{Zapf Chancery}\pagenote[a calligraphic script]{Zapf
Chancery\index{Zapf Chancery} is called the \pfontfam{pzc} 
fontfamily in \ltx.},
is in \fref{fig:chancery}.

\subsection{Display/Decorative}
\typesubidx{Display/Decorative}\typesubidx{Display}\typesubidx{Decorative}

    This is another very broad category 
but all the members are designed
to catch the eye. Display types tend to be used for display purposes and
are not meant to be too difficult to read. The decoratives are smaller in scale
but can be extremely detailed, such as alphabets based on human figures 
apparently performing calisthenic exercises. One of the many fonts
in this category, Cyklop\facesubseeidx{Cyklop}, is shown in 
\fref{fig:cyklop}.\pagenote[shown in \fref{fig:cyklop}]{The Cyklop\index{Cyklop}
typeface has been given the \pfontfam{cyklop} fontfamily name for access with \ltx.}

    As far as bookwork goes the more restrained of these types may occasionally
be useful for book or chapter titles.

\begin{figure}
\centering
{\centering\fontfamily{cyklop}\selectfont
  Display/Decorative --- Cyklop \\
  \UCalphabet \\
  \LCalphabet \\
  \fox\par}
\caption{An example of a Display/Decorative type: Cyklop} 
   \label{fig:cyklop}
\end{figure}

\section{Setting type}

\PWnote{2009/04/03}{Added bits and pieces about letterpress.}

   Until the last hundred years or so, type has been hand set. Today there
are still a few printers who still set type by hand, called 
now letterpress\index{letterpress} printing, and on the odd occasion I am 
one of them. Again, this is a brief description of the process but 
Chappell~\autocite{CHAPPELL99}
provides much more detail if you are interested. If you have a desire to
set up your own small print shop, perhaps in your garage or shed in the garden,
then John Ryder\index{Ryder, John} has lots of pertinent advice~\autocite{RYDER}. 
He directed
the design and production of many books for The Bodley Head, and, starting in
1930, produced much interesting work on an Adana\index{hand press!Adana} 
quarto press in his home.
The Briar Press\index{Briar Press} is another very useful resource, 
available via the web\label{briar}\footnote{\url{http://www.briarpress.org}} 
--- and in their words: `Proudly
introducing the bleeding-edge world of personalized desktop publishing
circa 1820'!

    The type is kept in type-, or job-, cases\index{typecase}\index{job case|see{typecase}}. 
These are shallow wooden 
partitioned trays and traditionally there were two of them for each font --- 
a lower one closer to the typesetter for the minuscule characters and the 
second one, arranged above the first as an `upper case', for the majuscules
(capitals); hence the terms lowercase and uppercase characters. The
characters are not arranged in alphabetical order but follow a system
that is meant to reduce the amount of movement required from the typesetter.
The shop where I print uses a `California job case' for type, 
illustrated in \fref{fig:cacase}, which combines both the lowercase and 
uppercase into a single case. The lowercase letters are arranged in a 
seemingly semi-random order while the uppercase, which are used much less
frequently, are in alphabetical order. The exceptions here are `U' and `J' 
which are latecomers to the alphabet only being generally accepted in the 16th
and 17th centuries, respectively --- printing has strong traditions.


\begin{figure}
\centerfloat
\setlength{\unitlength}{0.8pc}
\setlength{\unitlength}{0.7pc}
\setlength{\unitlength}{0.75pc}
\noindent \begin{picture}(50,30)
%  framing
  \put(0,0){\framebox(50,30){}}
  \put(1,1){\framebox(48,28){}}
% first 1/3
  \put(1,1){\framebox(2.0,4){q}}
  \put(1,5){\framebox(2.0,4){x}}
  \put(1,9){\framebox(2.0,4){z}}
  \put(1,13){\framebox(2.0,4){!}}
  \put(1,17){\framebox(2.0,4){?}}
  \put(1,21){\framebox(2.0,4){j}}
  \put(1,25){\framebox(2.0,4){ffi}}

  \put(3,1){\framebox(2,8){v}}
  \put(3,9){\framebox(2,8){l}}
  \put(3,17){\framebox(2,8){b}}
  \put(3,25){\framebox(2,4){fl}}

  \put(5,1){\framebox(4,8){u}}
  \put(5,9){\framebox(4,8){m}}
  \put(5,17){\framebox(4,8){c}}
  \put(5,25){\framebox(2,4){\shortstack{5 \\ to \\ an \\ em}}}
  \put(7,25){\framebox(2,4){\shortstack{4 \\ to \\ an \\ em}}}

  \put(9,1){\framebox(4,8){t}}
  \put(9,9){\framebox(4,8){n}}
  \put(9,17){\framebox(4,8){d}}
  \put(9,25){\framebox(2,4){'}}
  \put(11,25){\framebox(2,4){k}}

  \put(13,1){\framebox(4,8){\shortstack{3 to \\ an em}}}
  \put(13,9){\framebox(4,8){h}}
  \put(13,17){\framebox(4,12){e}}
%  second 1/3
  \put(18,1){\framebox(4,8){a}}
  \put(18,9){\framebox(4,8){o}}
  \put(18,17){\framebox(4,8){i}}
  \put(18,25){\framebox(2,4){1}}
  \put(20,25){\framebox(2,4){2}}

  \put(22,1){\framebox(4,8){r}}
  \put(22,9){\framebox(2,8){y}}
  \put(24,9){\framebox(2,8){p}}
  \put(22,17){\framebox(4,8){s}}
  \put(22,25){\framebox(2,4){3}}
  \put(24,25){\framebox(2,4){4}}

  \put(26,1){\framebox(2,4){.}}
  \put(26,5){\framebox(2,4){;}}
  \put(26,9){\framebox(2,8){w}}
  \put(26,17){\framebox(2,8){f}}
  \put(26,25){\framebox(2,4){5}}

  \put(28,1){\framebox(2,4){-}}
  \put(28,5){\framebox(2,4){:}}
  \put(28,9){\framebox(2,8){,}}
  \put(28,17){\framebox(2,8){g}}
  \put(28,25){\framebox(2,4){6}}

  \put(30,1){\framebox(4,8){quads}}
  \put(30,9){\framebox(2,8){ens}}
  \put(30,17){\framebox(2,4){fi}}
  \put(30,21){\framebox(2,4){ff}}
  \put(30,25){\framebox(2,4){7}}
%  \put(37,1){\framebox(5,8){quads}}

  \put(32,9){\framebox(2,8){ems}}
  \put(32,17){\framebox(2,4){0}}
  \put(32,21){\framebox(2,4){9}}
  \put(32,25){\framebox(2,4){8}}
%   third 1/3
  \put(35,1){\framebox(2,4){X}}
  \put(35,5){\framebox(2,6.67){P}}
  \put(35,11.67){\framebox(2,6.67){H}}
  \put(35,18.33){\framebox(2,6.66){A}}
  \put(35,25){\framebox(2,4){\$}}

  \put(37,1){\framebox(2,4){Y}}
  \put(37,5){\framebox(2,6.67){Q}}
  \put(37,11.67){\framebox(2,6.67){I}}
  \put(37,18.33){\framebox(2,6.66){B}}
  \put(37,25){\framebox(2,4){--}}

  \put(39,1){\framebox(2,4){Z}}
  \put(39,5){\framebox(2,6.67){R}}
  \put(39,11.67){\framebox(2,6.67){K}}
  \put(39,18.33){\framebox(2,6.66){C}}
  \put(39,25){\framebox(2,4){---}}

  \put(41,1){\framebox(2,4){J}}
  \put(41,5){\framebox(2,6.67){S}}
  \put(41,11.67){\framebox(2,6.67){L}}
  \put(41,18.33){\framebox(2,6.66){D}}
  \put(41,25){\framebox(2,4){(}}

  \put(43,1){\framebox(2,4){U}}
  \put(43,5){\framebox(2,6.67){T}}
  \put(43,11.67){\framebox(2,6.67){M}}
  \put(43,18.33){\framebox(2,6.66){E}}
  \put(43,25){\framebox(2,4){)}}

  \put(45,1){\framebox(2,4){\&}}
  \put(45,5){\framebox(2,6.67){V}}
  \put(45,11.67){\framebox(2,6.67){N}}
  \put(45,18.33){\framebox(2,6.66){F}}
  \put(45,25){\framebox(2,4){[}}

  \put(47,1){\framebox(2,4){ffl}}
  \put(47,5){\framebox(2,6.67){W}}
  \put(47,11.67){\framebox(2,6.67){O}}
  \put(47,18.33){\framebox(2,6.66){G}}
  \put(47,25){\framebox(2,4){]}}
\end{picture}
\caption{The California job case layout} \label{fig:cacase}
\end{figure}

    A line of type is set, or composed, in a hand-held 
composing\index{composing stick} stick, which
has an adjustable stop which is set to the required line length. Since the
letters read in reverse, right to left, they are assembled upside down, 
allowing working from left to right by the compositor. The characters for
a word are put into the stick, then a space, the next word, a space and on
until the line is almost full when it can be justified by inserting small extra 
spaces between the words. A lead may be put separating this line from the next,
which is then built up as before. When several lines have been assembled they
are slid from the composing stick onto the composing table, which is
a large, smooth and flat marble slab.


\enlargethispage{1em}

    When sufficient type has been assembled for printing one sheet of paper
it is put into a chase\index{chase} which is a rectangular cast-iron frame, 
rather like a picture frame. The chase is placed round the type
on the composing table and rectangular
blocks of wood, called furniture, are placed between the type and the chase
to hold the type in position, then expandable metal wedges, called 
quoins\index{quoin},
are inserted to lock the type within the chase. This is essential as the
chase and contents will be lifted up and transferred to the press itself ---
there is nothing like the joy of picking up and sorting out several hundred
small pieces of type that have scattered themselves all over the floor, and 
then putting them all back in the chase in the correct order! The assembled
and locked up type, furniture, and chase are called a forme\index{forme}.




\PWnote{2009/03/29}{Revised description of composing type and added paras 
        about kinds of hand presses}
    There are two basic types of hand\index{hand press} press. In the 
flat-bed\index{hand press!flat-bed} type, as 
used from Gutenberg's day onwards, the forme is fixed on a horizontal bed
which is mounted on horizontal rails, the type is inked (usually by hand), 
a sheet of paper is positioned over the forme, and the bed
slid under the platen --- a large flat plate --- which is then pressed
down hard by a screw mechanism to force the paper against the type. The platen 
is raised, the bed slid out and the printed sheet removed, ready for the next
sheet to be positioned. Originally the presses were made of oak but
nowadays are steel and cast iron. Typical of
the flat-bed presses are the \index{hand press!Albion} Albion in England 
and the Washington\index{hand press!Washington} in America.

    The other type is a platen\index{hand press!platen} press 
exemplified by the Excelsior\index{hand press!Excelsior} in England 
and the Chandler \& Price\index{hand press!Chandler \& Price}, which 
is the one I use, in America. Here the bed --- a rectangular steel plate ---
is vertical and fixed and the forme is locked onto the bed. Above the bed
is a circular disc on which the ink is spread. As the press operates rollers
come down over the disc picking up a thin film of ink, 
then over the forme --- thus inking the type ---
and back up again; while the rollers are inking the type the disc is
rotated a few degrees so that the rollers will run over a different section 
of the disc each time thus improving the uniformity of the ink pickup. 
A sheet of paper is clipped to the platen which, as the 
rollers go over the disc, swings up to press the paper against the inked type
and then down again so the sheet can be removed and the next one inserted.

    You can see pictures of the presses mentioned above, and many other as 
well, on the 
Briar Press\index{Briar Press} website (see \pref{briar}).

    At the end, the forme is put back on the composing table and any ink 
is cleaned off the type, which is then taken from the chase 
and `distributed' back into
the correct places in the typecases(s)\index{typecase}. The furniture and
leads are also put back into the proper places in their respective
storage areas.

    If the text is to be printed in multiple colours, the type for the first
colour is set with spaces left for the second coloured type, and then printed.
The type for the second colour is set in the spaces left for it, and the first
set of type removed and replaced by spaces. The press is cleaned and the first
colour ink replaced by the second colour ink. The original partially printed
pages are then printed with the next colour. If all is well the differently
coloured printed characters will all be aligned. Understandably, it is rare 
that more than two colours are used.

 
\section{Today}

    Today the great majority of printed works are produced with 
offset\index{offset printing} printer presses, the first of which was
invented by Ira Washington\index{Rubel, Ira Washington} Rubel in 
1903.\footnote{Rubel was an American; with a name like that he certainly
couldn't be English.}

    In the offset lithography process the original work image is put 
onto metal foils which 
are wrapped around a cylinder on the press; ink is picked up by the foil 
and is transferred to a `blanket' or `offset' cylinder that is in contact
with the first one. In turn, the ink from the offset cylinder is 
transferred to the paper which is pressed against it.

    Nowadays the original image is created on a computer and the foils
produced automatically. The system is excellent for coloured work --- one foil
is produced for each colour, following the CMYK\index{CMYK}
(Cyan, Magenta, Yellow, Key (black)) subtractive colour system 
(televisions and computer monitiors use the RGB\index{RGB} (Red, Green, Blue) 
additive colour system). In the lowest capital cost situation only a
single station press is used which can print a single colour. To get full
colour the paper must be run through the press four times in all, once for
each colour (and black) and if both sides are to be printed, then another
four times for the second side. Moving up
the scale there are two stand presses that can do two colours in one run,
four stand presses that will do all four in one run.\footnote{The press shop
where I occasionally go to do traditional hand set printing has all of these.} 
In most
smaller printing shops single sheets of paper are used which have to be put 
through the pres(ses) at least twice, once for each side. Newspapers are
also printed using offset lithography. In their case, though, web offset
printers are used where the paper is fed in continuously from a large roll and
is printed on both sides on its journey through the press.

    Up and coming are digital presses operating directly from the 
computer.\footnote{Think sophisticated inkjet or colour laser printers.}
These are approaching the cost and quality tradeoffs of offset printers
and are increasingly being used for on-demand small quantity printing. 

   Apart from the time and effort involved the principal difference between
traditional letterpress\index{letterpress} printing and the modern methods 
is the way the ink is put onto the paper. In the traditional method the type
is inked and then pressed \emph{into} the paper (think typewriters), while 
in modern methods the ink is effectively laid \emph{onto} the paper (think
inkjet printers). You can sometimes tell if something has 
been printed by letterpress methods by running a finger lightly over the page;
if the lines are `bumpy' then it has been printed letterpress. Printers using
letterpress actually face a dichotomy. In order to obtain the finest detail
from the type then it should be pressed as lightly as possible into the paper,
but to indicate clearly that it has been letterpress printed, then it is 
advisable to impress hard enough to leave permanent indentations, no matter
how slight. As is common, tradeoffs seem inevitable.



\section{Setting maths}

    I had always wondered how maths was typeset before TeX was available and
I eventually found an answer in an article by David Wishart~\autocite{WISHART03}.
The following is based upon his descriptions.

    As described above, before the advent of typesetting machines compositors 
picked pieces of type from a double typecase\index{typecase}, 
typically just containing the 
upper and lower case roman characters, the numerals and punctuation marks, 
assembled some lines in a composing stick\index{composing stick}, 
and transferred them to a chase\index{chase}, which when full was locked 
up and then 
put into the press to print onto the paper. When setting text this is
straightforward as each `line' in the composing stick is a line of text.
This is not the case with maths, such as this formula

\begin{equation} \label{eq:typeset}
P_{N_1 + m} = \frac{C}{N_{1} + m} \binom{N_{2} - N_{1}}{m}
              \alpha^{m}\beta^{(N_{2}-N_{1}) - m}
\end{equation}

    For setting math the compositor will have two double 
typecases\index{typecase} (roman
and italic), a case of mathematical sorts containing signs, superscripts
and subscripts, etc., and a case of unaccented Greek characters.
    In order to set maths such as shown as formula~\ref{eq:typeset},
in an assumed \U{11}{pt} font with \U{24}{pt} spacing, 
the widest elements, properly spaced and centered within the measure, 
are set in the composing\index{composing stick} stick as
\begin{displaymath}
P_{N_1 + m} = N_{1} + m \raisebox{-0.6\baselineskip}{$\displaystyle\binom{N_{2} - N_{1}}{\phantom{m}}$}
              \alpha^{m}\beta^{(N_{2}-N_{1}) - m}
\end{displaymath}

    The terms to the right of the $=$ sign are then removed and stored where
hopefully they won't be disturbed. A lead of \U{6.5}{pt} is then inserted above and below 
the first terms.
The $N_{1} +m$ term is taken from the storage, and a piece of \U{2}{pt} rule is cut
to the exact length of the term and the $C$ centered above it. Then the opening
parenthesis is added, so the contents of the composing stick look like:
\begin{displaymath}
P_{N_1 + m} = \frac{C}{N_{1} + m} \bigg(
\end{displaymath}

   Moving on, the $N_{2}-N_{1}$ term is put in the composing stick and a \U{2}{pt}
lead put below it with the $m$ centered underneath, then the closing parenthesis
is added, so the formula now appears as:
\begin{displaymath}
P_{N_1 + m} = \frac{C}{N_{1} + m} \binom{N_{2} - N_{1}}{m}
\end{displaymath}
 
    Finally the Greek terms are added, with \U{6.5}{pt} leads above and below, 
resulting in:
\begin{displaymath}
P_{N_1 + m} = \frac{C}{N_{1} + m} \binom{N_{2} - N_{1}}{m}
              \alpha^{m}\beta^{(N_{2}-N_{1}) - m}
\end{displaymath}

    Even if an automatic caster, such as a Linotype or Monotype, was being 
used the process was certainly not automatic. With a 
Monotype\index{Monotype} caster the 
operator would produce
\providecommand*{\tmri}{\mathrm{i}}
\providecommand*{\tmrx}{\mathrm{x}}
\begin{displaymath}
P_{N}\tmri_{+m} = N_{1} + m\tmrx N_{2}-N_{1}\tmrx \alpha^{m}\beta^{(N_{\tmri}-N_{\tmri}) - m} \; C \; m
\end{displaymath}

This would then go to the `marker-up' who would turn it into
\begin{displaymath}
P_{N_1 + m} = \frac{C}{N_{1} + m} \binom{N_{2} - N_{1}}{m}
              \alpha^{m}\beta^{(N_{2}-N_{1}) - m}
\end{displaymath}

After 1958 it was slightly easier when Monotype had introduced the 
`4-line' system but there was still a lot of handwork required. 
It is, however, much easier using LaTeX where 
formula~\ref{eq:typeset} above was specified as:
\begin{lcode}
\begin{equation}
P_{N_1 + m} = \frac{C}{N_{1} + m} \binom{N_{2} - N_{1}}{m}
              \alpha^{m}\beta^{(N_{2}-N_{1}) - m}
\end{equation}
\end{lcode}

%\end{document}



%%%%%%%%%%%%%%%%%%%%%%%%%%%%%%%%%%%%%%%%%%%%% 

\chapter{The Parts of a Book}

    This chapter describes the various parts of a book, the 
ordering of the parts, and the typical page numbering scheme used
in books. 

\section{\prFrontmatter}


    There are three major divisions in a book: 
the \pixfrontmatter\ or preliminaries\index{preliminaries}, 
the \pixmainmatter\ or text, 
and the \pixbackmatter\ or references. 
The main differences as
far as appearance goes is that in the \pixfrontmatter\ the folios\index{folio} are 
expressed as roman numerals and sectional divisions are not numbered. The 
folios\index{folio} are expressed as arabic numerals in the \pixmainmatter\ 
and \pixbackmatter. Sectional
divisions are numbered in the \pixmainmatter\ but not in the \pixbackmatter.

    The \pixfrontmatter\ consists of such elements as the title
of the book, a table of contents\ixtoc, and similar items. All pages are
paginated\index{pagination} --- that is they are counted --- but the first 
few pages in the \pixfrontmatter, the title pages and such, do not usually have 
folios\index{folio}. 
The remainder of the pages in the \pixfrontmatter\ do have folios\index{folio} 
which are usually expressed as roman numerals. Not all
books have all the elements described below.

    The first page is a recto 
\emph{half-title}\index{half-title page}\index{title page!half-title},
or \emph{bastard title}\index{bastard title page}\index{title page!bastard},
page with no folio\index{folio}. 
The page is very simple and displays just the main title of the book --- 
no subtitle, author, or other information. One purported purpose of this
page is to protect the main title page.

    The first verso page, the back of the half-title page, may contain the 
series title, if the book is one in a series, a list of contributors, 
a frontispiece, or may be blank. The series title may instead be put on the 
half-title page or on the copyright page.

   The \emph{title page}\index{title page} is recto and contains the full 
title of the work, the names of the author(s) or editor(s), and often at the
bottom of the page the name of the publisher, together with the publisher's 
logo if it has one.

    The title page(s) may be laid out in a simple manner or can have various
fol-de-rols, depending on the impression the designer wants to give. In
any event the style of this page should give an indication of the style
used in the main body of the work.

    The verso of the title page is the copyright page\index{copyright page}.
This contains the copyright notice, the publishing/printing history, 
the country where printed, ISBN and/or CIP information. The page is usually 
typeset in a smaller font\index{font!change} than the normal text.

    Following the copyright page may come a dedication or an epigraph\index{epigraph}, 
on a recto page, with the following verso page blank.

    This essentially completes the unfolioed pages.

    The headings\index{heading} and textual forms for the paginated 
pages should be the same as those for the \pixmainmatter, except that 
headings\index{heading} are usually unnumbered.

    The first folioed page,
usually with roman numerals (e.g., this is folio iii),
is recto with the Table of Contents (\toc). If the book contains 
figures\index{figure} (illustrations\index{illustration}) 
and/or tables\index{table}, the List of Figures (\lof) and/or List of Tables 
(\lot) come after the \toc, with no blank pages separating them. The \toc\ 
should contain an entry for each following major element. If there is a \lot, 
say, this should be listed in the \toc. The main chapters\index{chapter} must 
be listed, of course, and so should elements like a preface\index{preface}, 
bibliography\index{bibliography} or an index\index{index}.

    There may be a foreword\index{foreword} after the listings, with no blank
separator. A foreword is usually written by someone other than the author, 
preferably an eminent person whose name will help increase the sales potential,
and is signed by the writer. The writer's
signature is often typeset in small caps after the end of the piece.

   A preface\index{preface} is normally written by the author, in which he
includes reasons why he wrote the work in the first place, and perhaps to 
provide some more personal comments than would be justified in the body. 
A preface starts on the page immediately following a foreword, or the lists.

   If any acknowledgements are required that have not already appeared in the
preface, these may come next in sequence.

   Following may be an introduction if this is not part of the main text. 
The last elements in the front material may be a list of abbreviations, list
of symbols, a chronology of events, a family tree, or other information of
a like sort depending on the particular work.

    Table~\ref{tab:front} summarises the potential elements in the 
\pixfrontmatter.

\begin{table}
\centering
\caption{\prFrontmatter}\label{tab:front}
\begin{tabular}{llcc} \toprule
Element                      & Page  & Folio     & Leaf \\ \midrule
Half-title page              & recto & no        & 1 \\
Frontispiece, etc., or blank & verso & no        & 1 \\
Title page                   & recto & no        & 2 \\
Copyright page               & verso & no        & 2 \\
Dedication                   & recto & no        & 3 \\
Blank                        & verso & no        & 3 \\
Table of Contents\ixtoc            & recto & yes       & 3 or 4 \\
List of Figures\ixlof     & recto or verso & yes       & 3 or 4 \\
List of Tables\ixlot      & recto or verso & yes       & etc. \\
Foreword            & recto or verso & yes       & etc. \\
Preface             & recto or verso & yes       & etc. \\
Acknowledgements    & recto or verso & yes       & etc. \\
Introduction        & recto or verso & yes       & etc. \\
Abbreviations, etc  & recto or verso & yes       & etc. \\
\bottomrule
\end{tabular}
\end{table}


    Note that the titles Foreword, Preface and Introduction are somewhat
interchangeable. In some books the title Introduction may be used for what
is described here as the preface, and similar changes may be made among the 
other terms and titles in other books. 

\subsection{Copyright page}

    Most people are familiar with titles, \toc, prefaces, etc., but like
me are probably
less familiar with the contents of the copyright page\index{copyright page|(}. 
In any event this is
usually laid out by the publishing house, but some authors may like to be,
or are forced into being, their own publisher.

    The main point of the copyright page is to display the 
copyright\index{copyright} notice.
The Berne Convention does not require that published works carry a copyright
notice in order to secure copyright protection but most play it on the safe
side and include a copyright\index{copyright} notice.
This usually comes in three parts: the word \textit{Copyright} or more usually
the symbol \textcopyright, 
the year of publication, 
and the name of the copyright owner.
The copyright symbol matches the requirements of the Universal Copyright
Convention to which the USA, the majority of European and many Asian
countries belong.
The phrase `All rights reserved' is often added to ensure protection under the
Buenos Aires Convention, to which most of the Americas belong. A typical
copyright notice may look like: \\
{\footnotesize \textcopyright{} 2035 by Frederick Jones. All rights reserved.}

    Somewhere on the page, but often near the copyright notice, is the name 
and location(s) of the publisher.

    Also on the copyright page is the publishing history, denoting the edition
or editions\footnote{A second edition should be more valuable than a first
edition as there are many fewer of them.} and their dates, 
and often where the book has been printed. One thing that has puzzled me in
the past is the mysterious row of numbers you often see, looking like: \\
\centerline{\footnotesize\texttt{02 01 00 99 98 97\hspace{2em}10 9 8 7 6 5}}
The set on the left, reading from right to left, are the last two digits
of years starting with the original year of publication.
The set on the right, and again reading from right to left, represents the
potential number of new impressions (print runs). The lowest number in each 
group indicates the edition date and the current impression. So, the example
indicates the fifth impression of a book first published in 1997.

    In the USA, the page often includes the Library of Congress 
Cataloging-in-Publication (CIP)\index{CIP} data, 
which has to be obtained from the
Library of Congress. This provides some keywords about the book.

    The copyright page is also the place for the ISBN\index{ISBN} 
(International
Standard Book Number) number. This uniquely identifies the book. For example:
ISBN 0-NNN-NNNNN-2. The initial 0 means that the book was published in an
English-speaking country, the next group of digits identify the publisher,
the third group identifies the particular book by the publisher, and the final
digit, 2 in the example, is a check digit.

    It is left as an exercise for the reader to garner more information about
obtaining CIP and ISBN data.\index{copyright page|)}

\section{\prMainmatter}

    The \pixmainmatter\ forms the heart of the book.

    Just as in all the other parts of a book the pages within the 
\pixmainmatter\
are included in the pagination, even though some folios\index{folio} may
not be expressed. The folios\index{folio} are normally presented as arabic 
numerals, with 
the numbering starting at 1 on the first recto page of the \pixmainmatter.

    The \pixmainmatter\ is at least divided into \emph{chapters}\index{chapter}, 
unless it is something like a 
young child's book which consists of a single short story.
When the material may be logically divided into sections larger than
chapters, the chapters may be grouped into \emph{parts}\index{part} which
would then be the highest level of division within the book.
Frederic Connes\index{Connes, Frederic}
has told me that in French typography there is often a division above the
part level. This is also sometimes the case with English typography where
it is typically called a \emph{book}\index{book} --- the
\btitle{Chicago Manual of Style}~\autocite[p. 21]{CMS} shows an example.
A single physical book may thus be divided into levels from \emph{book} 
through 
\emph{part} and \emph{chapter} to further refinements.  
Typically all of books\index{book!number}, parts\index{part!number} 
and chapters\index{chapter!number} are numbered.
Obviously, part numbering should be continuous throughout the book, but even
with parts the chapter numbering is also continuous throughout the book.

    The title of a part\index{heading!part} is usually on a recto page which 
just contains the part title, and number if there is one. 
Book titles\index{heading!book} are usually treated the same way.
Chapters\index{chapter} also start on recto pages but in this case the text 
of the chapter\index{chapter} starts on the same page as the chapter
title.

    Where chapters\index{chapter} are long, or when the material is
complicated, they may be divided into sections\index{section}, each introduced
by a subhead\index{subhead}, either numbered or unnumbered, with
the numbering scheme starting afresh within each chapter. Similarly sections
may be partitioned into subsections by inserting sub-subheads, but except 
for more technical works this is usually as fine as the subdivisions need 
go to. Normally there are no required page breaks before the start of any 
subhead\index{subhead} within a chapter\index{chapter}.

    The title page of a part\index{part}
 or chapter\index{chapter} need not have the folio\index{folio} expressed, nor
a possibly textless verso page before the start of a chapter\index{chapter}, but all other 
pages should display their folios\index{folio}.

    There may be a final chapter\index{chapter} in the \pixmainmatter\ called Conclusions, 
or similar, which may be a lengthy summary of the work presented, untouched
areas, ideas for future work, and so on.

    If there are any numbered appendices\index{appendix} 
they logically come at the end of
the \pixmainmatter. Appendices are often `numbered' alphabetically rather
than numerically, so the first might be Appendix A, the second Appendix B,
and so on.

    An epilogue\index{epilogue} or an afterword\index{afterword} is a 
relatively short piece that the author may
include. These are not normally treated as prominently as the preceding
chapters\index{chapter}, and may well be put into the \pixbackmatter\ if they are 
unnumbered.

\section{\prBackmatter}

    The \pixbackmatter\
is optional but if present conveys information ancilliary
to that in the \pixmainmatter. The elements are not normally numbered, so an
unnumbered appendix\index{appendix} would normally come in the \pixbackmatter.

    Other elements include Notes, a Glossary\index{glossary}
 and/or lists of symbols\index{symbol} or 
abbreviations\index{abbreviation}, which could be in the 
\pixfrontmatter\ instead. These elements 
are normally unnumbered, as is any list of contributors\index{contributor}, 
Bibliography\index{bibliography} or Index\index{index}.

    In some instances appendices\index{appendix} 
and notes may be given at the end of each
chapter\index{chapter} instead of being lumped at the back.

    The first element in the \pixbackmatter\ starts on a recto page but the 
remainder may start on either recto or verso pages.

    In older books it was often the custom to have a colophon\index{colophon}
as the final element in a book. This is an inscription which includes 
information about the production and design of the book and nearly 
always indicates which fonts\index{font} were used.


\section{Signatures and casting off}
\index{signature|(}

    Professionally printed books have many pages printed per sheet of (large)
paper\index{paper}, which is then folded and cut where necessary to produce a 
\emph{gathering}\index{gathering} or \emph{signature} of several smaller 
sheets. An 
unfolded sheet is called a \emph{broadside}\index{broadside}. 
Folding a sheet in half produces a one sheet 
\emph{folio}\index{folio} signature with two leaves and four pages. 
Folding it in half again and cutting along the original fold gives a 
two sheet \emph{quarto}\index{quarto} signature with four leaves
and eight pages. 
Folding in half again, 
results in a four sheet \emph{octavo}\index{octavo} signature with eight
leaves and 16 pages, and so on as listed in \tref{tab:signatures}.

\begin{table}
\centering
\caption{Common signatures} \label{tab:signatures}
\begin{tabular}{lcccrccrccrc} \toprule
Name      & Folds & Size             & \multicolumn{3}{c}{Sheets} & 
\multicolumn{3}{c}{Leaves} & \multicolumn{3}{c}{Pages} \\ \midrule
Broadside & 0     & $a \times b$     & &  1 & & &  1 & & &   2 & \\
Folio     & 1     & $b/2 \times a$   & &  1 & & &  2 & & &   4 & \\
Quarto, \emph{4to} & 2 & $a/2 \times b/2$ & & 2 & & & 4 & & & 8 & \\
Octavo, \emph{8vo} & 3 & $b/4 \times a/2$ & & 4 & & & 8 & & & 16 & \\
\emph{16mo} & 4   & $a/4 \times b/4$ & &  8 & & & 16 & & &  32 & \\
\emph{32mo} & 5   & $b/8 \times a/4$ & & 16 & & & 32 & & &  64 & \\
\emph{64mo} & 6   & $a/8 \times b/8$ & & 32 & & & 64 & & & 128 & \\ \bottomrule
\end{tabular}%
\index{broadside}\index{folio}\index{quarto}\index{4to}\index{octavo}%
\index{8vo}\index{16mo}\index{32mo}\index{64mo}%
\end{table}

    In \tref{tab:signatures} the Size column is the untrimmed size of a 
leaf\index{leaf} in 
the signature
with respect to the size of the broadside. When made up into a book the
leaves will be trimmed to a slightly smaller size, at the discretion of the
designer and publisher; typically a minimum of 1/8 inch or 3 millimetres
would be cut from the top, bottom and \foredge\ of a leaf.

    Other folds can produce other signatures. For example a 
\emph{sexto}\index{sexto},
obtained by folding in thirds and then folding in half, is a three sheet
signature with six leaves and 12 pages.

    Paper has always been made in a wide range of sizes for a myriad of uses.
Table~\ref{tab:uspapersizes} lists some common American paper sizes.

\begin{table}
\centering
\caption[Some American paper sizes]{Some American paper sizes (in inches)}\label{tab:uspapersizes}
\begin{tabular}{lll}\toprule
`Dollar bill'   & \abyb{7}{3}    & Used for origami, not bills \\ 
Statement       & \abyb{8.5}{5.5}   & \\
Executive       & \abyb{10.5}{7.25} &  \\
Letter          & \abyb{11}{8.5} & Also in double, half or quarter size \\
Old (untrimmed) & \abyb{12}{9}   & Also called Architectural-A \\
Legal           & \abyb{14}{8.5} & \\
Ledger          & \abyb{17}{11}  & Also called Tabloid \\
Broadsheet      & \abyb{22}{17}  & As used in newsprint \\ \bottomrule
\end{tabular}
\end{table}

Traditionally the sizes are denoted by name but manufacturers did not 
necessarily make paper of the size that matched the name they gave it.
Some common names and trimmed sizes for British book work are given in 
\tref{tab:britpapersizes}.

\begin{table}
\centering
\caption[Some traditional British book paper sizes]{Some traditional British book paper sizes (in inches)}\label{tab:britpapersizes}
\begin{tabular}{lll}\toprule
Name        & Quarto             & Octavo \\ \midrule
pott        & \abyb{8}{6.5}      & \abyb{6.25}{4}{in} \\
foolscap    & \abyb{8.5}{6.75}   & \abyb{6.75}{4.25} \\
crown       & \abyb{10}{7.5}     & \abyb{7.5}{5} \\
post        & \abyb{10}{8}       & \abyb{8}{5} \\
large crown & \abyb{10.5}{8}     & \abyb{8}{5.25} \\
large post  & \abyb{10.25}{8.25} & \abyb{8.25}{5.25} \\
small demy  & \abyb{11.25}{8.5}  & \abyb{8.5}{5.675} \\
demy        & \abyb{11.25}{8.75} & \abyb{8.75}{5.675} \\
medium      & \abyb{11.5}{9}     & \abyb{9}{5.75} \\
small royal & \abyb{12.25}{9.25} & \abyb{9.25}{6.175} \\
royal       & \abyb{12.5}{10}    & \abyb{10}{6.25} \\
super royal & \abyb{13.5}{10.25} & \abyb{10.25}{6.75} \\
imperial    & \abyb{15}{11}      & \abyb{11}{7.5} \\ \bottomrule
\end{tabular}
\end{table}

The metric sizes, given in \tref{tab:metricpapersizes}, are those now 
recommended for book production where the metric system holds sway,
which includes the UK~\autocite[p. 104]{MCLEAN80}.

\begin{table}
\centering
\caption[Metric book paper sizes]{Metric book paper sizes (in mm)}\label{tab:metricpapersizes}
\begin{tabular}{lll}\toprule
  & untrimmed & trimmed \\ \midrule
metric crown octavo       & \abyb{192}{126} & \abyb{186}{123} \\
metric large crown octavo & \abyb{205}{132} & \abyb{198}{129} \\
metric demy octavo        & \abyb{222}{141} & \abyb{216}{138} \\
metric small royal octavo & \abyb{240}{158} & \abyb{234}{156} \\
A5                        &                 & \abyb{210}{148} \\ \bottomrule
\end{tabular}
\end{table}

    In making up the book, the pages in each signature are first fastened
together, usually by sewing through the folds. The signatures are then bound
together and the covers, end papers\index{paper!end} and spine are attached to form
the completed whole.


\begin{table}
\centering
\caption[Common American commercial paper sizes]%
        {Common American commercial paper sizes (in inches)} \label{tab:adriansizes}
\begin{tabular}{llll} \toprule
Sheet size & Book trim size & Common use & Pages per sheet (max) \\ \midrule
\abyb{45}{35} & \abyb{8.5}{5.5} & scholarly works & 32 pages \\
\abyb{50}{38} & \abyb{9.25}{6.125} & major nonfiction & 32 pages \\
\abyb{66}{44} & \abyb{8}{5.375} & fiction \& minor fiction & 64 pages \\
\abyb{68}{45} & \abyb{8.25}{5.5} & major fiction \& nonfiction & 64 pages \\
\abyb{45}{35} & \abyb{11}{8.5} & children's books, manuals & 16 pages \\
\abyb{50}{38} & \abyb{12.125}{9.25} & art monographs, children's books & 16 pages \\
\bottomrule
\end{tabular}
\end{table}

    Commercial printers use paper larger than shown in the previous tables;
they print several (final) pages on a single sheet, then fold it and trim it
down to the finished page size. Table~\ref{tab:adriansizes} is from
\autocite[p. 59]{ADRIANWILSON93}. He also says that other common trimmed sizes are
\abybm{9.25}{6.125}{in} out of \abybm{50}{38}{in} sheets, 
\abybm{10.25}{8.25}{in} out of \abybm{45}{35}{in} sheets, and so on.

\newpage


    Publishers like the final typeset book to be of a length that just fits
within an integral number of signatures\index{signature}, 
with few if any blank pages required
to make up the final signature. Casting off\index{casting off} is the
process of determining how many lines a given text will make in a given
size of type, and hence how many pages will be required.

    To cast off you need to know how many characters there will be in
a line, and how many characters there are, or will be, in the text. 
For the purposes of casting off, `characters' includes punctuation as well
as letters and digits. The
first number can be easily obtained, either from copy fitting tables or
by measurement; this is described in more detail in \S\ref{sec:tblock}.
The second is more problematic, especially when the manuscript has yet
to be written. A useful rule of thumb is that words in an English text
average five letters plus one space (i.e., six characters); 
word length in technical texts might be greater than this.

    To determine the number of words it is probably easiest to type a
representative portion of the manuscript, hand count the words and then
divide that result by the proportion of the complete text that you have
typed. For example, if you have typed 1/20\,th of the whole, then divide
by 1/20, which is equivalent to multiplying by 20. To fully estimate
the number of pages required it is also necessary to make allowance for
chapter\index{chapter} titles, illustrations\index{illustration}, and so forth.

    If it turns out, say, that your work will require 3 signatures plus 2
pages then it will be more convenient to make it fit into 3 signatures,
or 4 signatures minus a page or two. This can be done by expanding or cutting
the text and/or by changing the font\index{font!change} 
and/or by changing the number or width
of lines on a page.

    When I was editing a technical journal the authors were given a word 
limit. The primary reason was not that we were interested in the actual
word count but rather so that we could estimate, and possibly limit, 
the number of pages allotted
to each article; we used \emph{octavo}\index{octavo} 
signatures\index{signature} and no blank pages. 
I suspect that it is
the same with most publishers --- it is the page count not the word count
that is important to them.

    In some special cases, extra pages may be `tipped in'\index{tip in} to
the body of the book. This is most likely to occur for 
illustrations\index{illustration} which
require special paper\index{paper} for printing and it would be too costly to use
that paper\index{paper} for the whole work. Another example is for a fold-out of some sort,
a large map, say, or a triple spread illustration\index{illustration}. The tipped in pages
are glued into place in the book and may or may not be paginated. For
tipped in illustrations\index{illustration}, a List of Illustrations may well start with
a phrase like: `Between pages 52 and 53'.

\index{signature|)}

\section{Paper}

\index{paper|(}

    Paper, on which I assume your work will be printed, can be thought of
in seven categories, six of which are used in the making of books. The 
categories are: 
\begin{description}
\item[Special]\index{paper!special} is not used for books. 
It includes `wet strength tissues' and
other sanitary, cosmetic and industrial papers.
\item[Wrapping]\index{paper!wrapping} papers are for protective purposes. 
Of these kraft paper is made from
unbleached chemical wood fibre sized with resin. The fibres are long and 
strong, hence the name `kraft' from the German word for `strength'. The usual
colour is brown. Kraft paper is used in bookbinding for reinforcing endpapers
and, strengthening and shaping spines.
\item[Printing]\index{paper!printing} paper covers a wide range, 
from economical to expensive,
in surface finish from rough to highly polished (for fine art four colour
printing), and in colour.
\item[Writing]\index{paper!writing} paper is suitable for all stationery 
requirements. Ledger
paper is made from rag fibre, or a mixture of rag and wood pulp, 
and is strong, opaque and durable, with a smooth
surface. It is used for visitor's and account books and registers, and for
fine printing. Bank and bond papers are of good quality, strong, durable 
and nearly Ph neutral; they are made from fibres of chemical wood
sized with resin. In books they are mainly used for strengthening damaged
signatures. Artists' and designers' drawing papers usually have a rough 
surface --- cartridge paper, made from well sized chemical wood fibres, 
is often used for tipped on endpapers.
\item[Decorative]\index{paper!decorative} papers used for the endpapers 
and sides of books are 
of an extensive variety of colours, textures, patterns, and quality. Any
decorative paper used in a book should be strong with a good firm surface.
\item[Ingres]\index{paper!Ingres} and similar papers are mould-made 
from linen and/or cotton 
with a little
wood pulp. They come from Europe in a variety of quiet colours and are 
used in fine bindings for sides and endpapers.
\item[Japanese]\index{paper!Japanese} papers and tissues are mould-made 
from good quality rag fibre.
They are fine but strong and are extensively used for repairing documents,
mending leaves, and replacing damaged or missing areas. I find 
Kozo\index{paper!Kozo} paper very useful for repairing documents and, for 
example, as hinges when bookbinding. Although not paper, there are some 
wonderful Japanese bookcloths for binding covers.
\end{description}

\index{paper!machine-made|(}

    Machine-made paper, which is the commonest, comes in a number of 
sometimes overlapping categories, of which the main ones are:
\begin{description}
\item[Antique]\index{paper!antique} papers are soft textured papers originally
made for letterpress printing, but there are now surface sized ones for
offset lithography. 
\item[Machine finish]\index{paper!machine finish} papers have varying degrees 
of surface smoothness. They are also known as 
super-calendered\index{paper!super-calendered} or 
English\index{paper!English finish}.
\item[Coated]\index{paper!coated} paper has been flooded with fine clay and
adhesive to make them particularly good for halftones. Finishes range from 
dull through matte to glossy. 
\item[Impregnated]\index{paper!impregnated} papers are also known as
pigmented\index{paper!pigmented}. They are surface sized, lightly coated
and calendered and can take halftones, especially by lithography.
\item[Text]\index{paper!text} papers are textured and coloured and are 
useful for limited editions, book jackets and end papers. They often have 
a deckle edge on the two long sides.
\item[Cover]\index{paper!cover} papers are heavier varieties of text and other
papers and are typically used for pamphlet binding and paperback covers.
\item[Moldmade]\index{paper!moldmade} papers are made by machine to resemble
handmade papers, with deckle edges. They come in a wide range of textures,
colours, and weights. The available range includes papers suitable for
binding sides, endpapers, book jackets, or the text block.
\end{description}

    Handmade paper comes as single sheets but machine made paper can be 
obtained in either rolls or sheets. For some letterpress printing 
I recently bought some Strathmore 400
Drawing Paper as \abybm{3}{30}{feet} rolls at about 1/3 the price of the
same quantity of paper in sheet form; the downside was that I had to slice 
it up into
the sheet size I wanted to use, but in this case the upside was that I tore
rather than cut and obtained sheets with deckle edges on all four sides
so that at the end it looked rather like handmade paper.

    Paper for printing comes in different grades according to the intended 
use. The common ones, together with their manufactured size in inches, 
are~\autocite{POCKETPAL}:
\begin{description}
\item[Bond (\abyb{17}{22})]\index{paper!bond} Commonly used for letters and
business forms. They have surfaces acceptible for both pen and pencil.

\item[Coated (\abyb{25}{38})]\index{paper!coated} For high quality printed
work because of their surface smoothness and uniform ink receptivity.

\item[Text (\abyb{25}{38})]\index{paper!text} These often have interesting 
textures and a range of colours and are used for notices, brochures and 
booklets. There are often treated with a special sizing making them more
resistant to water penetration and easy to print.

\item[Book (\abyb{25}{38})]\index{paper!book} Are used for book and trade 
printing. They are less expensive than text papers but come in a wider range
of weights and bulk.

\item[Offset (\abyb{25}{38})]\index{paper!offset} Are similar to the coated and
uncoated book paper used for letterpress printing but with sizing added to
enhance offset printing.

\item[Cover (\abyb{20}{26})]\index{paper!cover} Complement coated and text
papers in matching colours and heavier weights for booklet covers, etc. It is
a handy rule of thumb that cover paper has about twice the thickness of text
paper of the same weight. If being used for postcards check that your choice 
is acceptable to the postal services.

\item[Index (\abyb{22.5}{35} and \abyb{25.5}{30.5})]\index{paper!index} Is 
stiff and easy to write on with a pen.

\item[Tag  (\abyb{24}{36})]\index{paper!tag} A strong utility paper for making
tags.

\item[Bristol (\abyb{22.5}{28.5})]\index{paper!bristol} One of the board grades
but softer than index or tag and easier to fold.

\item[Newsprint(\abyb{24}{36})]\index{paper!newsprint} The paper newspapers 
are printed on.

\item[Lightweight]\index{paper!lightweight} Covers a wide range of
specialty papers such as onion skin or Bible paper.

\item[Digital]\index{paper!digital} Specialty paper for use with
digital printers. Each technology (e.g., inkjet, laser) and often 
the particular printer tends to have its own requirements.
\end{description}
\index{paper!machine-made|)}


    It is not often that books include information about the paper on which
they are printed. If they do they are likely to be fine press books or 
limited editions, but even then most I have seen are silent on the matter. 
A few trade books do include details. Among the more popular 
papers\footnote{Meaning that I know that they have been used in more than
one book.} I have
come across are: 
Arches\index{paper!Arches}, of various kinds; 
Curtis Rag\index{paper!Curtis Rag};
Fabriano\index{paper!Fabriano};
Glatfelter\index{paper!Glatfelter};
Linweave Early American\index{paper!Linweave Early American}, which
has been used by the University of California press;
Mohawk Superfine\index{paper!Mohawk superfine};
Warren's Old Style\index{paper!Warren's Old Style}, which has been used for 
several books published by the University of California; and 
Strathmore\index{paper!Strathmore} of several kinds.

   At the time of writing I have finished hand letterpress printing a 
small book of 35 printed pages on Strathmore 400 Drawing Paper and an 
accordian book on Chinese Scholars on Southworth\index{paper!Southworth} 
\U{32}{lb} cotton paper. I am in the process of letterpress printing a collection
of poems as individual broadsides and for these I am using a variety of
papers from my local retail paper suppliers, all in letterpaper size ---
the regular US paper size for most non-commercial printing (in most of the 
world this would be the A4 size). These include: Wausau Royal Silk \U{24}{lb}
(90\gsm);
Construction \U{70}{lb} text; Exact Opaque Colors 24/\U{60}{lb} text; Wausau Exact 
Vellum Bristol \U{67}{lb} (145\gsm); Curious Lightspecs \U{70}{lb} text; Eames Furniture
Weave \U{80}{lb} text; Speckletone Kraft \U{70}{lb} text; and various Southworth papers. 
Thus, in different weights (thicknesses), colours and surface finishes.


\begin{table}
\centering
\caption{US basis size of various papers}\label{tab:basisweight}
\begin{tabular}{lll} \toprule
Paper type & Size & Ream \\
           & (inches) & (sheets) \\ \midrule
Bond, writing, ledger & \abyb{17}{22} & 500 \\ 
Coated, book, offset, text & \abyb{25}{28} & 500 \\
%%Manuscript cover      & \abyb{18}{31} & 500 \\
%%Blotting              & \abyb{19}{24} & 500 \\
%%Box cover             & \abyb{20}{24} & 500 \\
Cover                 & \abyb{20}{26} & 500 or 1000 \\
Index                 & \abyb{25\,\slashfrac{1}{2}}{30\,\slashfrac{1}{2}} & 500 \\
Tag                   & \abyb{24}{36} & 500 \\
Bristol          & \abyb{22\,\slashfrac{1}{2}}{28\,\slashfrac{1}{2}} & 500 \\
Newsprint     & \abyb{24}{36} & 500 \\
Tissue        & \abyb{24}{36} & 480 \\
Paperboard            & \abyb{12}{12} & 1000 \\ % (1000 square feet per ream) \\
\bottomrule
\end{tabular}
\end{table}


\begin{table}
\centering
\caption{Approximate paper weight equivalents}\label{tab:weightequivs}
\begin{tabular}{lccc} \toprule
Paper grade & Grammage & Thickness &  Caliper \\ 
            & \gsm\    & inches    & mm       \\ \midrule
%%%30\,lb text                  &  44 \\
\U{16}{lb} bond, \U{40}{lb} text     & 59  & 0.0032 & 0.081 \\
\U{45}{lb} text                  & 67  & 0.0036 & 0.092 \\
\U{20}{lb} bond                  & 75  & 0.0038 & 0.097 \\
\U{24}{lb} bond, \U{60}{lb} text     & 89  & 0.0048 & 0.120 \\
\U{70}{lb} text                  & 104 & 0.0058 & 0.147  \\
\U{80}{lb} text                  & 118 & 0.0061 & 0.155 \\
\U{67}{lb} bristol, \U{100}{lb} text & 148 & 0.0073 & 0.185 \\
\U{60}{lb} cover, \U{90}{lb} index   & 162 & 0.0074 & 0.188 \\
\U{65}{lb} cover, \U{80}{lb} bristol & 176 & 0.0078 & 0.198 \\
\U{110}{lb} index                & 199 & 0.0085 & 0.216 \\
\U{80}{lb} cover                 & 216 & 0.0092 & 0.234 \\
\bottomrule
\end{tabular}
\end{table}

    As you can see, paper comes in various weights. In most of the world this
is simply specified as grams per square meter (\gsm, sometimes designated as
\emph{gsm}). This quantity is properly called \emph{grammage}\index{grammage}
but English
speaking countries often use \emph{weight} instead. Typical office
paper is about 80\gsm, so an A4 sheet (1/16\sqrd{m}) weighs about 5g.

    In the US and a few other places it is much more complicated. The
weight is expressed in terms of \emph{basis weight}\index{basis weight} 
in pounds (lbs) for a known quantity of the
paper. The `known quantity' is a ream of paper of given dimensions. However, 
the `given dimensions' and the `ream' varies according to the type of paper.
The dimensions are usually not those of the finished product, but of the paper
as made before being cut into the final sizes. 
The advertised weight on a package of paper is no measure of the actual 
weight of the package that is being presented for sale. 
In some cases, though, an `M weight'\index{M weight} is also given, 
which is the weight (in
pounds) of 1000 cut sheets. Suppliers will often charge by the M weight as it
is always consistent for a given paper size and it also makes the shipping
weight easy to calculate. Some basis weights
are listed in \tref{tab:basisweight}.


    
    Sheets \abybm{17}{22}{in} can be cut into four \abybm{8\,\half}{11}{in}
letterpaper sized sheets, the regular size used by business. 
Similarly sheets \abybm{25}{38}{in} can be cut into sixteen
\abybm{6}{9}{in} book-sized sheets with little wastage. These two ream sizes
are among the most common ones.

    Paper thickness, or \emph{caliper}\index{caliper}, is a common measurement
used for some printing applications. This is usually measured directly
because it can only be estimated as the density of the paper is not known.
However reasonable estimates can be made for different weights of
the same kind of paper.

    Approximate paper weight equivalents are shown in \tref{tab:weightequivs}.
These are approximate because there are trade-offs between the characteristics
within the same basic weight which vary between manufactures.

   

\index{paper|)}

%%%\clearpage
%%%\raggedbottom
\chapter{The page}  \label{chap:lpage}

    Authors usually want their works to be read by others than themselves,
and this implies that their manuscript will be reproduced in some manner.
It is to be hoped that the published version of their work will attract 
readers and there are two aspects to this. The major is the actual content
of the work --- the thoughts of the author couched in an interesting
manner --- if something is boring, then there are too many other interesting
things for the reader to do than to plow on until the bitter end, 
assuming that he
even started to read seriously after an initial scan. The other aspect is
the manner in which the content is displayed. Or, in other words, 
the \emph{typography}
of the book, which is the subject of this chapter.

    The essence of good typography is that it is not noticeable at first,
or even second or later, glances to any without a trained eye. If your
initial reaction when glancing through a book is to exclaim about its layout
then it is most probably badly designed, if it was designed at all. Good
typography is subtle, not strident. 

    With the advent of desktop publishing
many authors are tempted to design their own books. It is seemingly all
too easy to do. Just pick a few of the thousands of fonts\index{font} that are available,
use this one for headings\index{heading}, 
that one for the main text, another one for
captions, decide how big the typeblock\index{typeblock} is to be, and there you are.

    However, just as writing is a skill that has to be learned, typography
is also an art that has to be learned and practised. There are hundreds
of years of experience embodied in the good design of a book. These are
not to be cast aside lightly and many authors who design their own books
do not know what some of the hard-earned lessons are, let alone that what
they are doing may be the very antithesis of these. An expert can break
the rules, but then he is aware that he has good reasons for breaking them.

    The author supplies the message and the typographer supplies the medium.
Contrary to Marshall McLuhan, the medium is \emph{not} the message, 
and the typographer's job is not to
intrude between the message and the audience, but to subtly increase the
reader's enjoyment and involvement. If a book shouts `look at me!' then it
is an advertisement, and a bad one at that, for the designer.


\section{The shape of a book}

    Books come in many shapes and sizes, but over the centuries certain
shapes have been found to be more pleasurable and convenient than others.
Thus books, except for a very very few, are rectangular in shape. The 
exceptions on the whole are books for young children, although I do
have a book edited by Fritz Spiegl and published by Pan Books entitled
\textit{A Small Book of Grave Humour}, which is in the shape of a tombstone
--- this is an anthology of epitaphs. Normally the height of a book, when 
closed, is greater than the width. Apart from any aesthetic reasons, 
a book of this shape is physically more comfortable to hold than one which 
is wider than it is high.

    It might appear that the designer has great freedom in choosing the
size of the work, but for economic reasons this is not normally the case.
Much typographical design is based upon the availabilty of certain 
standard industrial sizes of sheets of paper\index{paper!size}. 
A page size of \abybm{12}{8}{inches} will be much more expensive than one 
which fits on a standard
US letter sheet\index{paper!size!letterpaper} 
of \abybm{11}{8\:\Mfrac{1}{2}}{inches}. 
Similarly, one of the standard sizes
for a business envelope is \abybm{4\:\Mfrac{1}{8}}{9\:\Mfrac{1}{2}}{inches}. 
Brochures for mailing
should be designed so that they can be inserted into the envelope with 
minimal folding. Thus a brochure size of \abybm{5}{10}{inches} will be 
highly inconvenient, no matter how good it looks visually.

    Over the years books have been produced in an almost infinite variety
of proportions,
where by \emph{proportion}\index{proportion}
I mean the ratio of the height to the width of a
rectangle. However, certain proportions occur time after time throughout
the centuries and across many different countries and 
civilizations. This is because some proportions are inherently
more pleasing to the eye than others are. These pleasing proportions are
also commonly found in nature --- in  physical, biological, and chemical
systems and constructs. 

\index{proportion!page|(}

    Some examples of pleasing proportions can be
seen in Japanese wood block prints, such as the \textit{Hoso-ye} size
(\ratio{2}{1}) which is a double square, the \textit{Oban} (\ratio{3}{2}), %($3 : 2$),
the \textit{Chuban} (\ratio{11}{8}) and the \textit{Koban} size
(\ratio{{\sqrt{2}}}{1}). Sometimes these prints were made up into books, but
were often published as stand-alone art work. Similarly Indian paintings,
at least in the 16th to the 18th century,
often come in the range \ratio{1.701}{1} to \ratio{13}{9}, thus being around
\ratio{3}{2} in proportion.

    In medieval Europe page proportions were generally in the range
\ratio{1.25}{1} to \ratio{1.5}{1}. Sheets of paper\index{paper} were typically 
produced in the
proportion \ratio{4}{3} (\ratio{1.33}{1}) or \ratio{3}{2} 
(\ratio{1.5}{1}). 
All sheet proportions
have the property that they are reproduced with each alternate
folding of the sheet.
For example, if a sheet starts at a size of \abyb{60}{40} 
(i.e., \ratio{3}{2}),
then the first fold will make a double sheet of size \abyb{30}{40}
(i.e., \ratio{3}{4}). The next fold will produce a quadrupled sheet of size
\abyb{30}{20}, which is again \ratio{3}{2}, and so on. 
 The Renaissance typographers tended to like taller books, and their 
proportions would go up to \ratio{1.87}{1}
or so. The style nowadays has tended to go back towards the medieval
proportions.

    The standard ISO page proportions are 
\ratio{{\sqrt{2}}}{1} (\ratio{1.414}{1}). These
have a similar folding property to the other proportions, except in this case
each fold reproduces the original page proportion.
Thus halving an A0 sheet 
(size \abybm{1189}{841}{mm}) produces an A1 size sheet (\abyb{594}{841}),
which in turn being halved produces the A2 sheet (\abyb{420}{594}), down
through the A3, A4 (\abybm{210}{297}{mm}), A5, \ldots{} sheets.

For many years it was thought that it was impossible to fold a sheet of 
paper\index{paper}, no matter how large and thin, more than eight times 
altogether. This is not so as in 2002 a high school student, Britney Gallivan,
managed to fold a sheet of paper in half twelve times (see, for example,
\url{http://mathworld.wolfram.com/Folding.html}).


   There is no one perfect proportion for a page, 
although some are clearly better
than others. For ordinary books both publishers and readers tend to prefer
books whose proportions range from the light 
\ratio{9}{5} (\ratio{1.8}{1}) to the heavy
\ratio{5}{4} (\ratio{1.25}{1}). Some examples are shown in \fref{flpage:prop}.
 Wider pages, those with proportions less than
\ratio{{\sqrt{2}}}{1} (\ratio{1.414}{1}),
are principally useful for documents that need
extra width for tables\index{table}, marginal notes\index{marginalia}, 
or where multi-column\index{column!multiple} printing is preferred. 

\begin{figure}
\centering
\setlength{\unitlength}{1pc}
\begin{picture}(24,38)
\put(0,4){\begin{picture}(24,34)
  \put(0,0){\framebox(24,34){}}
  \thicklines \put(19.78,0){\line(0,1){34}}
  \thinlines
  \put(16,0){\line(0,1){34}}
  \put(17.78,0){\line(0,1){34}}
  \put(18.48,0){\line(0,1){34}}
  \put(19.2,0){\line(0,1){34}}
  \put(20.81,0){\line(0,1){34}}
  \put(21.33,0){\line(0,1){34}}
  \put(22.63,0){\line(0,1){34}}
  \put(0,-0.5){\begin{picture}(24,2)
    \put(16,0){\makebox(0,0){\textsc{a}}}
    \put(17.78,0){\makebox(0,0){\textsc{b}}}
    \put(18.48,0){\makebox(0,0){\textsc{c}}}
    \put(19.2,0){\makebox(0,0){\textsc{d}}}
    \put(19.78,0){\makebox(0,0){{$\varphi$}}}
    \put(20.81,0){\makebox(0,0){\textsc{e}}}
    \put(21.33,0){\makebox(0,0){\textsc{f}}}
    \put(22.63,0){\makebox(0,0){\textsc{g}}}
    \put(24,0){\makebox(0,0){\textsc{h}}}
    \end{picture}}
  \end{picture}}
  \put(0,0){\begin{picture}(24,4)
    \put(0,0){\begin{picture}(8,4)
      \put(0,2){\textsc{a} $2 : 1$}
      \put(0,1){\textsc{b} $9 : 5$}
      \put(0,0){\textsc{c} $1.732 : 1$ ($\sqrt{3}{} : 1$)}
      \end{picture}}
    \put(8,0){\begin{picture}(8,4)
      \put(0,2){\textsc{d} $5 : 3$}
      \put(0,1){{$\varphi$} $1.618 : 1$ ($\varphi{} : 1$)}
      \put(0,0){\textsc{e} $1.538 : 1$}
      \end{picture}}
    \put(16,0){\begin{picture}(8,4)
      \put(0,2){\textsc{f} $3 : 2$}
      \put(0,1){\textsc{g} $1.414 : 1$ ($\sqrt{2}{} : 1$)}
      \put(0,0){\textsc{h} $4 : 3$}
      \end{picture}}
    \end{picture}}
\end{picture}
\setlength{\unitlength}{1pt}
\caption{Some page proportions} \label{flpage:prop}
\end{figure}



    In books where the illustrations\index{illustration} are the primary 
concern, the shape of the illustrations\index{illustration} is generally 
the major influence on the page proportion.
The page size should be somewhat higher than that of the average 
illustration\index{illustration}. The extra height is required for the 
insertion of captions\index{caption} describing the
illustration\index{illustration}. 
A proportion of \ratio{{\pi}}{e} (\ratio{1.156}{1}), 
which is slightly higher
than a perfect square, is good for square illustrations.\footnote{Both $e$
and $\pi$ are well known mathematical numbers. $e$ ($= 2.718 \ldots$)
is the base of natural logarithms and $\pi$ ($= 3.141 \ldots$) is the
ratio of the circumference of a circle to its diameter.}
The \ratio{e}{{$\pi$}}
(\ratio{0.864}{1}) proportion is useful for landscape 
photographs  taken with a \abyb{4}{5}
format camera, while those from a \U{35}{mm} camera (which produces a negative
with a \ratio{2}{3} proportion) are better accomodated on 
an \ratio{0.83}{1} page.
\index{proportion!page|)}

\subsection{The golden section and Fibonacci series}

\index{golden section|(}
    Typographers need a modicum of mathematical ability, but no more
than an average teenager can do --- basically simple arithmetic. You can
skip this section if you wish as it just provides some background 
mathematical material which might be of interest.

    Since ancient Greek times or even before, the golden section, which
is denoted by the Greek letter $\varphi$ (phi), has been considered to be
a particularly harmonious proportion\index{proportion}. It should come as no surprise, then,
that this also has applications in typography.

    The Greeks were interested in geometry (think of Euclid). They discovered
that if you divide a straight line into two unequal parts then a certain
division appeared to have an especially appealing aesthetic quality about it. 
Call the length of the line $l$ and the length of the two parts $a$ and $b$, 
where $a$ is the smaller and $b$ is the larger. The division in question
is when the ratio of the larger to the smaller division ($b/a$) is the same
as the ratio of the whole line to the larger division ($l/b$).
More formally, two elements embody the golden section, symbolised by
$\varphi$, when the ratio of the larger
to the smaller is the same as the ratio of the sum of the two to the larger.
If the two elements are $a$ and $b$, with $a < b$, then
\begin{equation}
\varphi = \frac{b}{a} = \frac{a+b}{b} = (1+\sqrt{5})/2
\end{equation}

    The golden section has been called by a number of different names
during its history. Euclid\index{Euclid}
 called it the `extreme and mean ratio' while
Renaissance writers called it the `divine proportion\index{proportion}'; now it is
called either the `golden section' or the `golden ratio'. The symbol
$\varphi$ is said to come from the name of the Greek artist 
Phidias\index{Phidias}
(C5th \textsc{bc}) who often used the golden section in his sculpture.
A rectangle whose sides are in the same proportion\index{proportion} as the golden section
is often called a `golden rectangle'.
The front of the Parthenon on the Acropolis in Athens is a golden rectangle,
and such rectangles appear often in Greek architecture.
The symbol of the Pythagoran school was the star pentagram, 
%%%%shown in \fref{flpage:spent}, 
where each line is divided in the golden section.


    The approximate decimal value for $\varphi$ is $1.61803$. 
The number has some unusual properties. If you add one to $\varphi$
you get its square, while subtracting one from $\varphi$ gives its 
reciprocal.
\begin{eqnarray}
  \varphi + 1 & = & \varphi^{2} \\
  \varphi - 1 & = & 1/\varphi
\end{eqnarray}
It also has a very simple definition as the continued fraction
\begin{equation}
\varphi = 1 + \frac{1}{\displaystyle 1 + \frac{1}{\displaystyle 1 + \frac{1}{\displaystyle 1 + \frac{1}{1 + \cdots}}}}
\end{equation}


    In 1202 Leonardo Pisano, 
also known as Leonardo Fibonacci, wrote a
book called \textit{Liber Abbaci.}\footnote{Book of the Abacus.} One of the 
topics he was interested in was population growth. The book included
this exercise: \index{Fibonacci series|(}
\begin{quote}
How many pairs of rabbits\index{rabbit} can be produced from a single 
pair in a year?
Assume that each pair produces a new pair of offspring every month,
a rabbit becomes fertile at age one month, and no rabbits die during the
year.
\end{quote}
After a month there will be two pairs. At the end of the next month the
first pair will have produced another pair, so now there are three pairs.
At the end of the following  month the original pair will have produced a
third pair of offspring and their firstborn will also have produced a pair, 
to make five pairs in all. And so on. 
If, like the rabbits, you are not too exhausted
to continue, you can get the following series of 
numbers\footnote{The numbers at the start of the series
depend on whether you consider the initial pair of rabbits to be adults or 
babies.\label{fn:rabbits}}:
\begin{displaymath}
0,1,1,2,3,5,8,13,21,34, 55, 89 \ldots
\end{displaymath}
After the first two terms, each term in the series is the sum of the two
preceding terms. Also, as one progresses along the series, the ratio of
any adjacent pair of terms oscillates around $\varphi$ ($= 1.618 \ldots$),
approaching it ever more closely.
\begin{eqnarray*}
  8/5 & = & 1.6 \\
  13/8 & = & 1.625 \\
  21/13 & = & 1.615 \\
  34/21 & = & 1.619 \\
  55/34 & = & 1.6176 \\
  89/55 & = & 1.6182
\end{eqnarray*}

    For the mathematically inclined there is another, to me, typographically
striking
relationship between $\varphi$ and the Fibonacci series. Define the
Fibonacci numbers as $F_{n}$, where
\begin{equation}
\begin{array}{cccc}
F_{0} = 0;\ \ & F_{1}=1;\ \ & F_{n+2}=F_{n+1} + F_{n},  &  n \geq 0.
\end{array}
\end{equation}
Then
\begin{equation}
 F_{n} = \frac{1}{\sqrt{5}}(\varphi^{n} - (- \varphi)^{-n})
\end{equation}

    Both the Fibonacci series and the golden section appear in nature.
The arrangement of seeds in a sunflower, the pattern on the surface of a 
pinecone, and the spacing of leaves around a stalk all exhibit Fibonacci
paterns (for example see~\autocite{CONWAY96}). Martin Gardner~\autocite{GARDNER66}
reports on a study of 65 women that claimed that the average ratio of a 
person's height to the height of the navel is $1.618+$ --- suspiciously 
close to $\varphi$. According to Dan Brown, the author of 
\btitle{The Da Vinci Code}, Mario Livio's 
\btitle{The Golden Ratio}~\autocite{LIVIO02} 
`\ldots unveils the history and mystery of the remarkable
number phi in such a way that \ldots you will never again look at a pyramid,
pinecone, or Picasso in the same light'.

\index{Fibonacci series|)}
\index{golden section|)}

%%%%%%%%%%%%%%%%%%%%%%%%%%%
%%%%%%\endinput
%%%%%%%%%%%%%%%%%%%%%%%%%%%

\section{The spread} \label{sec:spread}
\index{spread|(}
\index{proportion!page|(}
\index{proportion!typeblock|(}

    The typeblock\index{typeblock} is that part of the page which is normally 
covered with type. The same proportions that are useful for the shape of a 
page are also useful for the shape of the typeblock. This does not mean, 
though, that the proportions of the page and the typeblock should be the same. 
For instance, a square typeblock on a square page is inherently dull.

    When we first start to learn to read we scan horizontally along each line
of text. As our skills improve we tend to scan vertically rather than
horizontally. A tall column\index{column} of text helps in this process, 
provided that the column\index{column} is not too wide.

    A page in a book will typically contain several elements. Principal
among these is the typeblock\index{typeblock}, but there are also items like 
the folio\index{folio} (that is, the page number), 
a running header\index{header} and/or footer\index{footer} 
which carries the chapter\index{chapter} 
and/or book title, and possibly marginalia\index{marginalia} and 
footnotes\index{footnote}. These latter
elements, although essential to the content of the book, are minor visual
elements compared to the typeblock\index{typeblock}. 
But even minor decoration can obscure
or kill an otherwise good design.

  The major concern is the positioning of the typeblock on the page. 
The mere fact of positioning the typeblock\index{typeblock} also has 
the result of producing margins\index{margin} onto the page. 
Page design is a question of balancing the page proportions
with the proportions of the typeblock and the proportions 
of the margins\index{margin} to 
create an interesting yet harmonious composition. A single page, except
for a title page, is never the subject of a design but rather the design
is in terms of the two pages that are on view when a book is opened --- the
left and right hand pages are considered as a whole. More technically, the
design is in terms of a \emph{double spread}.



\begin{table}
\caption{Some page designs} \label{tlpage:allp}
\centering
%%\DeleteShortVerb{\|}
%\begin{tabular}{|l|l|l|c|} \hline
\begin{tabular}{|r|r|rrrrr|l|} \hline
\multicolumn{1}{|c|}{$P$} & \multicolumn{1}{c|}{$T$} & \multicolumn{5}{c|}{Margins \& Columns} & 
\multicolumn{1}{c|}{Figure}          \\ 
 & & \multicolumn{1}{c|}{$s$} & \multicolumn{1}{c|}{$t$} & \multicolumn{1}{c|}{$e$} & 
     \multicolumn{1}{c|}{$f$} & \multicolumn{1}{c|}{$g$}     &                 \\ \hline
$\sqrt{3}$ & $2$     & $w/13$   & $8s/5$ & $16s/5$ & $16s/5$ &       & \ref{fb:1} left \\ %Bringhurst
$\sqrt{3}$ & $e/\varphi$ & $w/10$ & $2s$ & $2s$    & $3s$    &       & \ref{fb:1} right \\ % Machie
$12/7$     & $1.701$ & $w/7$    & $8s/5$ & $8s/5$  & $14s/5$ &       & \ref{fb:2} left \\ % Grenfell
$e/\varphi$ & $7/4$  & $w/10$   & $5s/4$ & $5s/3$  & $11s/8$ &       & \ref{fb:2} right \\ % JKJ
$\varphi$  & $1.866$ & $w/9$    & $s$    & $2s$    & $7s/3$  &       & \ref{fb:3} left \\ %Paris
$\varphi$  & $\varphi$ & $w/12$ & $2s$   & $5s/2$  & $4s$    &       & \ref{fb:3} right \\ %Dowding
$8/5$      & $1.634$ & $2w/15$  & $7s/5$ & $9s/5$  & $13s/5$ &       & \ref{fb:4} left \\ %Rogers
$19/12$    & $7/4$   & $2w/15$  & $s$    & $9s/8$  & $11s/8$ &       & \ref{fb:4} right \\ %Anatomy
$19/12$    & $\sqrt{3}$ & $w/7$ & $s$    & $5s/4$  & $1.84s$ &       & \ref{fb:5} left \\ % Cornford
$19/12$    & $8/5$   & $w/12$   & $7s/5$ & $8s/5$  & $2s$    &       & \ref{fb:5} right \\ %Abeced
$\pi/2$    & $9/5$   & $w/9$    & $3s/2$ & $5s/2$  & $3s$    &       & \ref{fb:6} left \\ %Dwiggins
$e/\sqrt{3}$ & $1.71$ & $w/10$  & $11s/8$ & $24s/11$ & $8s/3$ &      & \ref{fb:6} right \\ %Two Men
$1.553$    & $1.658$ & $w/11$   & $\varphi s$ & $\varphi s$ & $\varphi s$ & & \ref{fb:7} left \\ %Express
$1.538$    & $\sqrt{7}$ & $w/10$ & $s$   & $23s/6$ & $3s/2$  &       & \ref{fb:7} right \\ %T&H
$3/2$      & $2$     & $w/5$    & $s/2$  & $s$     & $s$     &       & \ref{fb:8} left \\ %Rome
$3/2$      & $1.701$ & $w/9$    & $s$    & $2s$    & $7s/3$  &       & \ref{fb:8} right \\ %Venice
$3/2$      & $\pi/2$ & $w/13$   & $2s$   & $10s/3$ & $30s/7$ &       & \ref{fb:9} left \\ %Magellan
$3/2$      & $3/2$   & $w/9$    & $3s/2$ & $2s$    & $3s$    &       & \ref{fb:9} right \\ %Gutenberg
$3/2$      & $1.68$  & $w/23$   & $2s$   & $5s$    & $2s$    &       & \ref{fb:10} left \\ % Pers Mss
$3/2$      & $3/2$   & $w/10$   & $2s$   & $5s/2$  & $2.85s$ &       & \ref{fb:10} right \\ % Pers Bk
$1.48$     & $1.376$ & $w/12$   & $7s/4$ & $2s$    & $7s/2$  &       & \ref{fb:11} left \\ %Goudy
$13/9$     & $\sqrt{2}$ & $w/30$ & $2s$  & $9s/2$  & $4s$    & $s/2$ & \ref{fb:11} right \\ %Doomsday
$\sqrt{2}$ & $\varphi$ & $w/9$  & $s$    & $2s$    & $2s$    &       & \ref{fb:12} left \\ %Orig A4
$\sqrt{2}$ & $\varphi$ & $w/8$  & $s$    & $5s/3$  & $5s/3$  &       & \ref{fb:12} right \\ %Mod A4
$7/5$      & $1.641$   & $w/7$  & $s$    & $8s/5$  & $8s/5$  &       & \ref{fb:13} left \\ %Emery Walker
%%%$1.294$    & $\varphi$ & $0.176w$ & $1.03s$ & $1.685s$ & $13s/9$ &   & \ref{fb:13} right \\ %LaTeX
$17/22$    & $1.594$ & $0.176w$ & $1.21s$ & $1.47s$ & $1.05s$ &      & \ref{fb:13} right \\ %LaTeX
$1.294$    & $13/9$  & $w/12$   & $s$    & $2s$    & $10s/7$ & $s/2$ & \ref{fb:14} left \\ %Wilson
$9/7$      & $19/9$  & $2w/5$   & $5s/8$ & $5s/8$  & $5s/6$  &       & \ref{fb:14} right \\ %Kuniyoshi
$5/4$      & $13/11$ & $w/10$   & $3s/2$ & $2s$    & $8s/3$  &       & \ref{fb:15} left \\ %Fens
%%%$7/6$    & $17/15$ & $w/13$   & $s$    & $s$     & $7s/5$  & $.382$ & \ref{fb:15} right \\ %Durer
$7/6$      & $55/48$ & $w/10$   & $9s/10$ & $8s/10$ & $13s/10$ & $1.05s$ & \ref{fb:15} right \\ %Durer
%$1.176$    & $1.46$  & $0.107w$ & $5s/6$ & $2.41s$ & $3s/2$  &       & \ref{fb:12} left \\ %Art
$e/\pi$    & $0.951$ & $w/9$    & $s$    & $2s$    & $3s/2$  &       & \ref{fb:16} left \\ %Hammer & Hand
$5/7$      & $2/3$   & $w/9$    & $s/2$  & $2s/3$  & $s$     & $s/3$ & \ref{fb:16} right \\ \hline %Hokusai
\end{tabular}
%%\MakeShortVerb{\|}
\end{table}

  Table~\ref{tlpage:allp} gives some examples of 
page designs. These are arranged in increasing order of
fatness. In this table, and afterwards, I have just used a single number
to represent the ratio of the page height to the width; that is, for example,
$1.5$ instead of \ratio{1.5}{1} or $12/7$ instead of \ratio{12}{7}.
The following symbols are used in the table:
\begin{description}
\item[Proportions]:
  \begin{itemize}
  \item[$P$] = page proportion = $h/w$
  \item[$T$] = typeblock proportion = $d/m$
  \end{itemize}
\item[Page size]:
  \begin{itemize}
  \item[$w$] = width of page
  \item[$h$] = height of page
  \end{itemize}
\item[Typeblock]:
  \begin{itemize}
  \item[$m$] = measure (i.e., width) of primary typeblock
  \item[$d$] = depth (excluding folios, running heads, etc.)
%%  \item[$n$] = measure of secondary column
%%  \item[$c$] = column width, when there are two columns
  \end{itemize}
\item[Margins]:
  \begin{itemize}
  \item[$s$] = spine margin (back margin)
  \item[$t$] = top margin (head margin)
  \item[$e$] = \foredge\ (front margin)
  \item[$f$] = foot margin (bottom margin)
  \item[$g$] = internal gutter (on a multi-column page)
  \end{itemize}
\end{description}

    Theoretically the following relationship holds among the several
variables:
\begin{displaymath}
f + t - T(s + e) = w(P - T)
\end{displaymath}
However, due to measurement and other difficulties, the numbers given in 
the table do not always obey this rule but they are close enough to give
a good idea of the relative values. In any event, page design is not a
simple arithmetic exercise but requires aesthetic judgement.

    The designs are also shown in \figurerefname s~\ref{fb:1} 
to~\ref{fb:16}. Each of these shows a double page spread; the 
page width has been kept constant throughout the series to enable easier
visual comparison --- it is the relative proportions, not the absolute size, 
that are important. I have only shown the pages and the typeblocks to avoid
confusing the diagrams with headers\index{header}, footers\index{footer} 
or folios\index{folio}.


\begin{figure}
\centering
\begin{minipage}[b]{\pwlayi}
\drawaspread{\pwlayii}{1.732}{2}{.0769}{1.6}{3.2}{0} % Bringhurst
\end{minipage}
\hfill
\begin{minipage}[b]{\pwlayi}
\drawaspread{\pwlayii}{1.732}{1.684}{.1}{2}{2}{0} % Machiavelli
\end{minipage}
\caption[Two spreads: Canada, 1992 and England, 1970]%
        {Two spreads: (Left) Canada, 1992. % Bringhurst.
         (Right) England, 1970.} \label{fb:1}
\end{figure}

    Shown in \fref{fb:1} are two modern books. On the left is the layout
for Robert Bringhurst's\index{Bringhurst, Robert} 
\btitle{The Elements of Typographical Style} published
by Hartley \& Marks in 1992, and designed by Bringhurst~\autocite{BRINGHURST99}. 
The text face is
Minion\facesubseeidx{Minion} set with $12$pt leading on a $21$pc measure. 
The captions are set in Scala Sans\facesubseeidx{Scala Sans}. The original
size is \abybm{227}{132}{mm} and is printed on Glatfelter laid 
\index{paper!Glatfelter}paper. 
I highly recommend this book if you are
interested in typography. 

The layout on the right is The Folio Society's\index{Folio Society}
1970 edition of \btitle{The Prince} by Niccol\`{o} Machiavelli. The original
size is \abybm{216}{125}{mm} and the text is set in \abyb{12/13}{22} 
Centaur\facesubseeidx{Centaur}.
Chapter titles\index{chapter!design} are set as raggedright block 
paragraphs using Roman numbers
and small caps for the text; not all chapters start a new page. There are
no running headers and the folios\index{folio} 
are set at the center of the footer\index{footer!design}.
The ToC is typeset like the standard \ltx\ ToC\index{ToC!design} 
but with the chapter titles
in small caps.



\begin{figure}
\centering
\begin{minipage}[b]{\pwlayi}
\drawaspread{\pwlayii}{1.714}{1.701}{.143}{1.6}{1.6}{0} % Grenfell
\end{minipage}
\hfill
\begin{minipage}[b]{\pwlayi}
\drawaspread{\pwlayii}{1.68}{1.75}{.1}{1.25}{1.667}{0} % JKJ
\end{minipage}
\caption[Two spreads: USA, 1909 and England, 1964.]%
        {Two spreads: (Left) USA, 1909.
         (Right) England, 1964.} \label{fb:2}
\end{figure}

    Figure~\ref{fb:2} (left) illustrates a small book by Wilfred T.~Grenfell
entitled \btitle{Adrift on an Ice-Pan} published in 1909 by the Riverside
Press\index{Riverside Press} of Boston. The text is set with a leading 
of $16$pt on a $16$pc 
measure. The large leading and small measure combine to give a very 
open appearance. The original size is \abybm{184}{107}{mm}. 
 The half-title\index{half-title}\index{page!half-title} is set in 
bold uppercase about 1/3
of the way down the page. Uppercase is used for chapter\index{chapter!design}
headings which are centered. Captions for the photographs\index{illustration}
are also uppercase and are listed on an illustrations page. 
The folios\index{folio} are
centered in the footer\index{footer!design} and enclosed in square 
brackets (e.g., [17]), and the
headers\index{header!design} contain the book title, centered, 
and in uppercase.



On the right is another book from the
Folio Society\index{Folio Society} --- \btitle{Three Men in a Boat} 
by Jerome K.~Jerome printed
in 1964. The original size is \abybm{215}{128}{mm} and is typeset with
Ehrhardt\facesubseeidx{Ehrhardt} at \abyb{11/12}{22}. 
Chapter\index{chapter!design} titles are
centered and simply consist of  the word `CHAPTER' followed by the number.
There are no headers and the folio\index{folio} 
is set between square brackets (like [27])
in the center of the footer\index{footer!design}. The ToC\index{ToC!design}
title is centered, and the chapter entries are like standard \ltx\ except
that the numbers are set in a roman font while the texts, which give a 
summary of the chapter contents, are typeset in italic.

\begin{figure}
\centering
\begin{minipage}[b]{\pwlayi}
\drawaspread{\pwlayii}{1.618}{1.87}{.111}{1}{2}{0} % Paris
\end{minipage}
\hfill
\begin{minipage}[b]{\pwlayi}
\drawaspread{\pwlayii}{1.618}{1.618}{.0833}{2}{2.5}{0} % Dowding
\end{minipage}
\caption[Two spreads: France, 1559 and Canada, 1995]%
        {Two spreads: (Left) France, 1559.
         (Right) Canada, 1995.} \label{fb:3}
\end{figure}

   Jean de Tourmes\index{de Tourmes, Jean}, a Parisian publisher, 
printed \btitle{Histoire et Chronique}
by Jean Froissart in 1559. This is a history book with the main text in
roman and sidenotes in italic at roughly 80\% of the size of the main text.
The layout is shown in \fref{fb:3} (left). The gutter (not shown) between 
the main text and the sidenote\index{sidenote} column\index{column} 
is very small, 
but the change in fonts and sizes enables the book to be read with no 
confusion. 


Another Hartley \& Marks typography book --- \btitle{Finer Points in
  the Spacing \& Arrangement of Type} by Geoffrey
Dowding\index{Dowding, Geoffrey} --- is shown at the right of
\fref{fb:3}.  This is typeset in Ehrhardt\facesubseeidx{Ehrhardt} at
\abyb{10.5/14}{23} on a page size of \abybm{231}{143}{mm} on
Glatfelter\index{paper!Glatfelter} Laid Offset paper.  The
half-title\index{half-title} is uppercased, centered, and in the upper
quarter of the page. On the title page\index{title page} the title is
typeset with a large bold italic font while the author's name is set
using normal uppercase and the publisher is set in small caps. Dowding
uses `part' instead of `chapter'.  Chapter\index{chapter!design} heads
are centered with the number written out, like `PART ONE', and below
this is the title set in large italics.  Section\index{section!design}
heads are in uppercase and subsection heads in small caps, both
centered. Folios\index{folio} are in the center of the
footer\index{footer!design}; verso running heads\index{header!design}
consist of the book title in small caps and centered, and recto heads
contain the chapter title in italics and centered. On the
contents\index{ToC!design} page the part (chapter) numbers and titles
are centered, using small caps and large italics respectively (and no
page numbers). Section titles are in small caps, left justified with
the page numbers right justified. Titles from the \pixfrontmatter\ and
\pixbackmatter, for example the Foreword and Bibliography, are typeset
in italics.



\begin{figure}
\centering
\begin{minipage}[b]{\pwlayi}
\drawaspread{\pwlayii}{1.6}{1.634}{.1333}{1.4}{1.8}{0} % Rogers
\end{minipage}
\hfill
\begin{minipage}[b]{\pwlayi}
\drawaspread{\pwlayii}{1.583}{1.75}{.1333}{1}{1.125}{0} % Anatonomy
\end{minipage}
\caption[Two spreads: USA, 1949 and 1990]%
        {Two spreads: (Left) USA, 1949.
         (Right) USA, 1990.} \label{fb:4}
\end{figure}

    Bruce Rogers\index{Rogers, Bruce} (1870--1957) 
described how he came to design his Centaur\facesubseeidx{Centaur} typeface in
\btitle{Centaur Types}, a privately published book by his studio October
House in 1949. The layout of this book, which of course was typeset in
Centaur, is shown at the left of \fref{fb:4}. Centaur is an upright
seriffed type based on Nicolas Jenson's\index{Jenson, Nicolas} type as 
used in \btitle{Eusebius} published in 1470. 
\btitle{Centaur Types} demonstrates typefaces other than
Centaur, and also includes exact size reproductions of the engraver's 
patterns. It is set at \abyb{14/16}{22} on a page size of \abybm{240}{150}{mm}.

   Figure~\ref{fb:4} (right) is the layout of another book on typefaces.
It is \btitle{The Anatomy of a Typeface} by Alexander Lawson published by
David R.~Godine in 1990~\autocite{LAWSON90}.
This is set in Galliard\facesubseeidx{Galliard} with $13$pt leading and a measure of $24$pc on
a page size of \abybm{227.5}{150}{mm} on Glatfelter\index{paper!Glatfelter} 
Offset Smooth Eggshell paper. The half-title\index{half-title}\index{page!half-title}
is set in uppercase in the upper quarter of the page. On the 
title\index{page!title} page the title is in uppercase in a large outline 
font, with a double rule above and a short single rule below. The author
is set in small caps (both upper- and lowercase like \textsc{Lawson})
and the publisher is in regular lowercase small caps.
Chapter\index{chapter!design} heads are centered with the number set between
a pair of fleurons\index{fleuron}, followed by the title in 
large uppercase, and with
a short rule between the title and the start of the text. 
The folios\index{folio}
are in the center of the footer\index{footer!design} with a short rule
above them; there are no running headers. The contents\index{ToC!design}
page is set with the body type; chapter numbers are flushleft with a 
following period and the page numbers are flushright.



\begin{figure}
\centering
\begin{minipage}[b]{\pwlayi}
\drawaspread{\pwlayii}{1.583}{1.731}{.143}{1}{1.25}{0} % Cornford
\end{minipage}
\hfill
\begin{minipage}[b]{\pwlayi}
\drawaspread{\pwlayii}{1.583}{1.6}{.0833}{1.4}{1.6}{0} % Abecedarium
\end{minipage}
\caption[Two spreads: England, 1908 and USA, 1993]%
        {Two spreads: (Left) England, 1908.
         (Right) USA, 1993.} \label{fb:5}
\end{figure}

    \btitle{Microcosmographica Academia} by F. M. Cornford is shown in
\fref{fb:5}. Despite its title, it is written in English and was published
by Bowes \& Bowes, London, in 1908. It is a dryly humourous look at academic
politics as practised in Cambridge University at the turn of the nineteenth
century (probably in the twentieth and twenty-first as well). 
It is set with $14$pt leading
on $22$pc. The original page size is \abybm{216}{136}{mm}.
The half-title\index{half-title}\index{page!half-title} 
is in normal uppercase in the upper
sixth of the page; the title\index{page!title} page is all uppercase in
various sizes. Chapter\index{chapter!design} heads are centered with first 
the number in Roman numerals and below the title in uppercase.
Folios\index{folio} are centered in the footer\index{footer!design} 
and there are no running heads. There is no table of contents.



The right of this figure illustrates a book with another unusual title ---
\btitle{The Alphabet Abecedarium} by Richard A.~Firmage and published by
David R.~Godine in 1993. It is set in 
Adobe Garamond\facesubseeidx{Adobe Garamond}\index{Garamond!Adobe} on a $27$pc measure
with $14$pt leading. The original page size is \abybm{227.5}{150}{mm}. The
book gives a history of each letter of the Latin alphabet. 
Chapter\index{chapter!design} heads are centered and consist of an 
ornamental version of the letter in question. One
unusual feature is that there is a deep footer\index{footer!design} 
on each page showing many examples of typefaces of the letter being 
described. Verso running headers\index{header!design} consist of the book
title in mixed small caps and centered with the folio\index{folio} flushleft.
Recto headers have the folio\index{folio} flushright, and centered is the alphabet, 
typeset in small caps except for the current letter which is enlarged.


\begin{figure}
\centering
\begin{minipage}[b]{\pwlayi}
\drawaspread{\pwlayii}{1.571}{1.8}{.111}{1.5}{2.5}{0} % Dwiggins
\end{minipage}
\hfill
\begin{minipage}[b]{\pwlayi}
\drawaspread{\pwlayii}{1.562}{1.709}{.1}{1.375}{2.182}{0} % Two Men
\end{minipage}
\caption[Two spreads: USA, 1931 and England, 1968]%
        {Two spreads: (Left) USA, 1931.
         (Right) England, 1968.} \label{fb:6}
\end{figure}


    W.~A.~Dwiggins was, among many other things, an American book designer.
Figure~\ref{fb:6} (left) shows his layout of H.~G.~Wells' \btitle{The Time
Machine} for Random House in 1931. The page size is \abyb{231}{147}{mm}.

The right of the figure illustrates the layout of a book called 
\btitle{Two Men --- Walter Lewis and Stanley Morrison at Cambridge}
by Brooke Crutchley\index{Crutchley, Brooke} 
and published by Cambridge University 
Press\index{Cambridge University Press} in 1968. This is typeset in 
Monotype Barbon\facesubseeidx{Monotype Barbon}\index{Barbon!Monotype}
with $17.5$pt leading on a $26$pc measure on a \abybm{253}{162}{mm} page.
Crutchley was the Cambridge University 
Printer\index{Cambridge University Press!Crutchley, Brooke} and each year would produce
a limited edition of a book about Cambridge or typography, and preferably
both together, for presentation to friends of the Press. The tradition of
the Printer's Christmas Book\index{Cambridge University Press!Christmas Book} 
was started by Stanley 
Morison\index{Morison, Stanley}\index{Cambridge University Press!Morison, Stanley} 
in 1930 and continued until 1974. The books usually consisted of a short 
essay on a
particular topic, so they did not have chapter heads, tables of contents,
or other appurtenances, apart from a Preface.



\begin{figure}
\centering
\begin{minipage}[b]{\pwlayi}
\drawaspread{\pwlayii}{1.553}{1.685}{.0909}{1.618}{1.618}{0} % Express
\end{minipage}
\hfill
\begin{minipage}[b]{\pwlayi}
\drawaspread{\pwlayii}{1.538}{2.647}{.1}{1}{3.833}{0} % Thames & Hudson
\end{minipage}
\caption[Two spreads: USA, 1994 and England, 1988]%
        {Two spreads: (Left) USA, 1994.
         (Right) England 1988.} \label{fb:7}
\end{figure}

\enlargethispage{\baselineskip}


    A modern technical book layout is given in \fref{fb:7}. The book
is \btitle{Information Modeling the EXPRESS Way} by Douglas Schenck and Peter
Wilson, published by Oxford University Press (New York) in 1994. This is
set in Computer Modern Roman\facesubseeidx{Computer Modern Roman} at \abyb{10/12}{27} 
on a page \abybm{233}{150}{mm}. 
It has the typical \ltx\ appearance with perhaps the exception of the
epigraphs\index{epigraph} after each chapter\index{chapter!design} heading.



Ruari McLean's\index{McLean, Ruari} \btitle{The Thames and Hudson Manual of
Typography} (1988) is at the right in \fref{fb:7}. This is typeset in \abyb{10/11}{20} 
Monophoto Garamond\facesubseeidx{Monophoto Garamond}\index{Garamond!Monophoto} 
on a \abybm{240}{156}{mm} page. The wide
\foredge{} is used for small illustrations\index{illustration}. 
Notes are also set in this
margin\index{margin} rather than at the foot of the page.
The half-title\index{half-title}\index{page!half-title} 
is in a bold font, flushright, in the 
upper quarter of the page; there is a wood engraving of a galleon at the 
bottom, also flushright. The title\index{page!title} uses a mixture of fonts
and is set flushright; an example title page based on this design is shown
in \fref{fig:titleTH}. Chapter\index{chapter!design} are on recto
pages and consist of the number and title in a bold font, flushleft and near
the top of the page, and an engraving of some kind is at the bottom
right of the page; there is no other text on this page, the body of
the chapter starting at the top of the following verso page. 
Folios\index{folio} are in the footers\index{footer!design} at the outer edge
of the page. Running headers\index{header!design} contain the chapter
title in small caps flushright in the outer margin.

\begin{figure}
\centering
\begin{showtitle}
\titleTH
\end{showtitle}
\caption{Title page design based on \btitle{The Thames and Hudson Manual of
Typography} (1988)} \label{fig:titleTH}
\end{figure}


\begin{figure}
\centering
\begin{minipage}[b]{\pwlayi}
\drawaspread{\pwlayii}{1.5}{2}{.2}{.5}{1}{0} % Rome
\end{minipage}
\hfill
\begin{minipage}[b]{\pwlayi}
\drawaspread{\pwlayii}{1.5}{1.7}{.111}{1}{2}{0} % Venice
\end{minipage}
\caption[Two spreads: Italy, 1523 and 1499]%
        {Two spreads: (Left) Italy, 1523.
         (Right) Italy 1499.} \label{fb:8}
\end{figure}

 Many page layouts in earlier days were constructed by
drawing with compass and ruler, usually based on regular geometric figures; 
the use of squares, pentagons and hexagons being particularly
prevelant. Unusually, the typeblock\index{typeblock} in \fref{fb:8} (left) 
is centered on the page. The typeblock\index{typeblock} is based on a 
square, the depth being twice the measure. The book, \btitle{Canzone} by 
Giangiorgio Trissino, is a volume of poems and was published in Rome 
about 1523 by Ludovico degli Arrighi\index{Arrighi, Ludovico degli}. 
Prose works
from the same typographer followed the normal style of having the \foredge\
wider than the spine margin\index{margin!spine}\index{margin!inner}.

    The page proportion in \fref{fb:8} (right) is also a simple \ratio{3}{2}
ratio. The proportions of the typeblock, being \ratio{1.7}{1}, 
are based upon a pentagon.
The book is \btitle{Hypnerotomachia Poliphili} by Francesco Colonna and was
published by Aldus Manutius\index{Manutius, Aldus} in Venice in 1499. 
The story of this,
including some reproductions from the original, is told by Helen
Barolini~\autocite{BAROLINI92}.


\begin{figure}
\centering
\begin{minipage}[b]{\pwlayi}
\drawaspread{\pwlayii}{1.5}{1.571}{.0769}{2}{3.333}{0} % Magellan
\end{minipage}
\hfill
\begin{minipage}[b]{\pwlayi}
\drawaspread{\pwlayii}{1.5}{1.5}{.111}{1.5}{2}{0} % Gutenberg
\end{minipage}
\caption[Two spreads: France/Portugal, 1530 and Gutenberg, C15th]%
        {Two spreads: (Left) France/Portugal, 1530.
         (Right) Gutenberg, C15th.} \label{fb:9}
\end{figure}

    In 1519 the Portugese explorer Ferdinand Magellan set sail from 
Sanl\'{u}car de Barramada, near C\'{a}diz in Spain, 
with five ships and about 270 men.
Three years later one ship and 18 men returned, having made the first
circumnavigation. Among the few survivors was Antonio Pigafetta who recorded
the adventure. 
A very few manuscripts of his report are in existence.
The layout of one of these manuscripts which is in the Beinecke Rare
Book and Manuscript Library at Yale is shown at the left of \fref{fb:9}.
The manuscript, which is written in French, is called 
\btitle{Navigation et descouurement de la Inde superieure et isles
de Malueque ou naissent les cloux de Girofle} (Navigation and discovery
of Upper India and the Isles of Molucca where the cloves grow) is written
in a beautiful humanistic minuscule\index{minuscule!humanist}. 
There are 27 lines to a page, which
is \abybm{286}{190}{mm} and made of vellum. The text measure is $29.5$pc
and the `leading' is $21$pt. The wide outer (\foredge) margin\index{margin!outer} is used 
for sidenotes\index{sidenote}
indicating highlights of the story. The manuscript was probably prepared 
soon before 1530; the scribe and where he worked is unknown.

    Many of the books produced by Johannes Gutenberg\index{Gutenberg, Johannes}
(1398--1468) and his early successors followed the form shown in
\fref{fb:9} (right). This set of proportions was also often used in
medieval incunabula\index{incunabula}\footnote{Early books, especially 
those printed before 1500.}  and manuscripts. The page and typeblock 
proportions are the same (\ratio{3}{2}). The margins\index{margin} are in the 
proportions $2 : 3 : 4 : 6$.
A graphical method for constructing this, and similar designs, is 
shown later in \fref{flpage:lgut}.

\begin{figure}
\centering
\begin{minipage}[b]{\pwlayi}
\drawaspread{\pwlayii}{1.5}{1.68}{.043}{2}{5}{0} % Persian Mss
\end{minipage}
\hfill
\begin{minipage}[b]{\pwlayi}
\drawaspread{\pwlayii}{1.5}{1.5}{.1}{2}{2.5}{0} % Persian book
\end{minipage}
\caption[Two spreads: Persia, 1525 and USA, 1975]%
        {Two spreads: (Left) Persia, 1525.
         (Right) USA, 1975.} \label{fb:10}
\end{figure}

     Two versions of the same publication are shown in \fref{fb:10}.
On the left is a Persian manuscript \btitle{Khamsch of Nizami} written
about 1525. The page size is about \abybm{324}{216}{mm}. The 
illustrations\index{illustration} and
the typeblock\index{typeblock} are inextricably mixed. On the right is a translation of
some of the manuscript published as \btitle{Tales from the Khamsch of Nizami}
by the Metropolitan Museum of Art, New York, in 1975. The modern version
has a page size of \abybm{300}{200}{mm}, slightly smaller than the original
but in the same proportions. The typeblock\index{typeblock} is $32$pc wide 
and the type is set with a $15$pt leading.

\begin{figure}
\centering
\begin{minipage}[b]{\pwlayi}
\drawaspread{\pwlayii}{1.48}{1.375}{.0833}{1.75}{2}{0} % Goudy
\end{minipage}
\hfill
\begin{minipage}[b]{\pwlayi}
\drawaspread{\pwlayii}{1.45}{1.414}{.0333}{2}{4.5}{2.175} % Doomsday
\end{minipage}
\caption[Two spreads: USA, 1952 and England, 1087]%
        {Two spreads: (Left) USA, 1952.
         (Right) England, 1087.} \label{fb:11}
\end{figure}

    Frederic Goudy\index{Goudy, Frederic} was a prolific American 
type designer. Shown at the left of
\fref{fb:11} is the layout of his book \btitle{The Alphabet and Elements
of Lettering} published by the University of California Press in 1952.
This is typeset in his University of California Old 
Style\facesubseeidx{University of California Old Style}, which has
interesting ct and st ligatures. The measure is $36$pc and the leading
is $18$pt. The first half of the book gives a short history of the development
of writing and fonts. The second half consists of 27 plates, one for each 
letter of the alphabet, and the last one for the ampersand character. These 
show the evolution of each letter from Roman times to the mid-twentieth
century.


    Figure~\ref{fb:11} (right) shows the layout of the English 
\btitle{Domesday Book} which is a manuscript book written in 1087. 
It records all the domains won by William the Conqueror in 1066. 
The book is written in a Carolingian minuscule\index{minuscule!Carolingian}
in two columns\index{column!double}, with 44 lines per column ragged right. 
The two columns have slightly different widths. The first part of the book 
is more meticulously written than the later parts, where the scribe appears 
to be in haste to finish.


\begin{figure}
\centering
\begin{minipage}[b]{\pwlayi}
\drawaspread{\pwlayii}{1.414}{1.618}{.111}{1}{2}{0} % A4 orig
\end{minipage}
\hfill
\begin{minipage}[b]{\pwlayi}
\drawaspread{\pwlayii}{1.414}{1.618}{.125}{1}{1.667}{0} % A4 mod
\end{minipage}
\caption[Two spreads for ISO page sizes]%
        {Two spreads: (Left) ISO (1).
         (Right) ISO (2).} \label{fb:12}
\end{figure}

    Figure~\ref{fb:12} shows two different layouts for a page corresponding
to the ISO international standard proportion of $\sqrt{2}$. In each case
the typeblock\index{typeblock} is the same and proportioned in the 
golden section\index{golden section}, but the margins\index{margin} are 
different. The layout on the left provides adequate
room for marginal\index{marginalia} notes in the \foredge.


\begin{figure}
\centering
\begin{minipage}[b]{\pwlayi}
\drawaspread{\pwlayii}{1.404}{1.641}{.143}{1}{1.6}{0} % Emery Walker
\end{minipage}
\hfill
\begin{minipage}[b]{\pwlayi}
\drawaspread{\pwlayii}{1.294}{1.618}{.176}{1.037}{1.685}{0} % Latex
\end{minipage}
\caption[Two spreads: England, 1973 and LaTeX $\U{10}{pt}$ book style]%
        {Two spreads: (Left) England, 1973.
         (Right) LaTeX $10$pt book style.} \label{fb:13}
\end{figure}

Another of the Cambridge University Printer's\index{Cambridge
  University Press!Christmas Book} Christmas books is at the left of
\fref{fb:13}. In this case it is \btitle{Emery Walker --- Some Light
  on his Theories of Printing and on his Relations with William Morris
  and Cobden-Sanderson} by Colin Franklin and published in 1973. The
page size is \abybm{295}{210}{mm} with a measure of $31$pc set with
$15$pt leading.  Unusually for this series it has chapter
heads\index{chapter!design} which are simply the number centered above
the title in a large font. It also has
illustrations\index{illustration} which are listed on an illustrations
page where the caption\index{caption} titles are set flushleft and
page numbers flushright. The page is divided into two lists. The first
has a heading (centered) in italics of \textit{`In text'} with
\textit{`page'} flushright above the page numbers.  The second has the
centered heading \textit{`In pocket at end'} and there are no page
numbers in this list as the corresponding illustrations are not bound
into the book, instead thay are loosely inserted in a pocket at the
end of the book.


On the right is the default layout provided by the
LaTeX $10$pt book class on US letterpaper\index{paper!size!letterpaper}.

\begin{figure}
\centering
\begin{minipage}[b]{\pwlayi}
\drawaspread{\pwlayii}{1.294}{1.444}{.0833}{1}{2}{0.5} % Adrian Wilson
\end{minipage}
\hfill
\begin{minipage}[b]{\pwlayi}
\drawaspread{\pwlayii}{1.286}{2.11}{.4}{.625}{.625}{0} % Kuniyoshi
\end{minipage}
\caption[Two spreads: USA, 1967 and England, 1982]%
        {Two spreads: (Left) USA, 1967.
         (Right) England, 1982.} \label{fb:14}
\end{figure}

    Adrian Wilson\index{Wilson, Adrian}, who died in 1988, was an 
acclaimed American book designer.
His work on book design, \btitle{The Design of Books}, out of print since
1988 but reissued in 1993 by Chronicle Books, is outlined
at the left of \fref{fb:14}. This is in two columns\index{column!double}, 
with many illustrations\index{illustration},
on letterpaper\index{paper!size!letterpaper} size pages. 
It is typeset in Palatino\facesubseeidx{Palatino} and 
Linotype Aldus\facesubseeidx{Linotype Aldus}\index{Aldus!Linotype} 
with $12$pt leading. 
Each column is $18$pc wide. The title page is a simple design and an
example based on it is shown in \fref{fig:titleDB}. 
Chapter\index{chapter!design} heads are 
flushright in a large italic, preceeded by the number. 
Section\index{section!design} heads are flushleft in uppercase and 
subsection heads are also flushleft but in normal sized italics.
The Contents\index{ToC!design} list is in the left hand column, typeset 
using the normal font with titles flushleft and page numbers flushright;
there is an engraving in the bottom half of the right hand column.
There are no running headers. Verso footers\index{footer!design} have the
chapter title flushleft in small caps with the folio\index{folio}
to the left of this (i.e., in the margin); recto footers similarly have
the chapter title flushright and the folio to the right in the margin.

\begin{figure}
\centering
\begin{showtitle}
\titleDB
\end{showtitle}
\caption{Title page design based on Adrian Wilson's \btitle{The Design of Books}} \label{fig:titleDB}
\end{figure}

The other layout in this
figure is B.~W.~Robinson's \btitle{Kuniyoshi: The Warrior Prints} published
by Phaidon, Oxford in 1982. The page size is \abybm{310}{242}{mm} with a
measure of $28.5$pc. The type is set with $13$pt leading. The wide spine
margin\index{margin!spine}\index{margin!inner} is used for some small 
reproductions of Japanese 
woodblock prints, some of which extend across the binding itself. 
The majority of the book has no text apart from captioning the many 
reproduced prints which take up full pages.

\begin{figure}
\centering
\begin{minipage}[b]{\pwlayi}
\drawaspread{\pwlayii}{1.25}{1.182}{.1}{1.5}{2}{0} % Fens
\end{minipage}
\hfill
\begin{minipage}[b]{\pwlayi}
\drawaspread{\pwlayii}{1.167}{1.133}{.077}{1}{1}{0.382} % Durer
\end{minipage}
\caption[Two spreads: England, 1972 and Switzerland, 1980]%
        {Two spreads: (Left) England, 1972.
         (Right) Switzerland, 1980.} \label{fb:15}
\end{figure}

    \btitle{The Waterways of the Fens} by Peter Eden with drawings by
Warwick Hutton is another of the Cambridge 
Printer's\index{Cambridge University Press!Christmas Book} Christmas books.
This is set with $17$pt leading on a measure of $27$pc. The original
page size is \abybm{195}{150}{mm} and is illustrated on the left 
of \fref{fb:15}. 
The amount of text on a page varies
and there are many line drawings, some of which take a double spread.
Folios\index{folio} are in the outer margin level with the top
line of text.

On the right of this figure is another art book, namely \btitle{D\"{u}rer}
by Fedja Anzelewsky published by Chartwell Books in 1980. This is set in
two columns\index{column!double} with $14$pt leading on a $23.5$pc measure, 
although there are more illustrations\index{illustration} than text. The page
size is \abybm{280}{240}{mm}, considerably larger than its companion in
the figure, yet with much smaller margins\index{margin}.
Roman numerals are used for chapter\index{chapter!design} heads which are
set flushleft in a large font. Immediately below the chapter head is a line
of `section' titles, flushleft, in a font size intermediate between the
chapter head and the body. A centered dot is used to separate the section
titles. Folios\index{folio} are set in the foot\index{footer!design} flush
with the outside of the typeblock; there are no running heads. 
The Table of Contents\index{ToC!design} title matches the chapter heads.
Chapter title entries are set flushleft with their page numbers flushright.
The section titles are set in a line below the chapter entry, again separated
by centered dots. In the text, references to 
illustrations\index{illustration} are placed in the outside margins in 
a small font.

\begin{figure}
\centering
\begin{minipage}[b]{\pwlayi}
%\drawaspread{\pwlayii}{1.176}{1.46}{.107}{.833}{2.41}{0.25} % Art
\drawaspread{\pwlayii}{.865}{.951}{.111}{1}{2}{0} % Hammer & Hand
\end{minipage}
\hfill
\begin{minipage}[b]{\pwlayi}
\drawaspread{\pwlayii}{.714}{.667}{.111}{0.5}{0.667}{0.333} % Hokusai
\end{minipage}
\caption[Two spreads: England, 1969 and USA 1989]%
        {Two spreads: (Left) England, 1969.
         (Right) USA, 1989.} \label{fb:16}
\end{figure}


    Two more layouts for illustrated books are given in \fref{fb:16}.
In this case the illustrations\index{illustration} are drawings in landscape 
mode (i.e., they are wider than they are high); the shape of the drawings 
has had a major effect on the page proportions. In the case on the left 
the page proportion is in the ratio \ratio{{\pi}}{e}. The measure is 
longer than usual at $37$pc and to compensate for this the leading of $17$pt 
is also larger than customary. It is typeset in Centaur\facesubseeidx{Centaur}.
The book is
\btitle{Hammer and Hand} by Raymond Lister with drawings by Richard Bawden.
It was published in 1969 by Cambridge University 
Press\index{Cambridge University Press!Christmas Book} and is another of
the University Printer's Christmas books. Folios\index{folio}
are in a large font at the outside edge of the page and level with the 
top text line.


Shown on the right of \fref{fb:16} is \btitle{Hokusai --- One Hundred Poets}
by Peter Morse and published by George Braziller in 1989. The introductory
text is set in two columns\index{column!double} as shown. The body consists 
of illustrations\index{illustration} of Japanese wood block prints, 
originally in the large \textit{oban} size of about \abybm{250}{380}{mm}.
The half-title\index{half-title}\index{page!half-title} 
is set in a large font in the top
right hand corner of the page, but the text on the title\index{page!title}
page is centered. The main body is organised with a wood block print
on each recto page and the commentary on the facing verso page. At the
top of each commentary page, centered, is the number of the print, a short
rule, the title of the print in a large italic font, a longer rule,
and then side by side the poem in Japanese and an English translation.
The commentary itself is underneath, set in three raggedright 
columns\index{column!multiple}; minor heading in the commentary are
flushleft in small caps. The folio\index{folio} is at the center
of the foot\index{footer!design}. At the end of the book is a commentary on
those poems where there are no known illustrations; this is typeset
raggedright in four columns\index{column!multiple}.


\subsection{A geometric construction} \label{sec:gutenbergpage}

    Nowadays it is easy to pick and calculate any kind of page proportion
that takes your fancy, but how did the early printers do it? They certainly
did not have the use of calculators and I suspect that they had only enough
arithmetic to keep their accounts. Printing was a craft and craftsmen did
not release their trade secrets lightly. I believe that most of the designs
were based on simple geometric figures, which required nothing more than
a ruler and a pair of compasses.

 Jan Tschichold\index{Tschichold, Jan} gives a simple construction for the 
layout of many of the books based on Gutenberg's\index{Gutenberg, Johannes} 
work~\autocite[pages 44--57]{TSCHICHOLD91}, 
which is shown in \fref{flpage:lgut}.
The construction actually divides the page up into ninths (the point
\textsc{p} in the diagram, which is at the intersection of the main and half
diagonal construction lines, is one third of the way down and across both the
page and the typeblock\index{typeblock}). This construction can be used 
no matter what the page proportions and will give the same relative result.


\begin{figure}
\centering
\setlength{\unitlength}{1pc}
\begin{picture}(20,15)
\put(0,0){\framebox(20,15){}}
\thicklines
 \put(10,0){\line(0,1){15}} % spine
\put(0,0){\line(4,3){20}} % ll to tr diag
\put(20,0){\line(-4,3){20}} % lr to tl diagonal
\put(0,0){\line(2,3){10}}  % ll to tm line
\put(20,0){\line(-2,3){10}} % lr to tm line
\put(13.333,0){\line(0,1){15}} % vertical line thro' half & full diags
\put(14,10){\makebox(0,0)[tl]{\textsc{p}}}
\put(13.333,15){\line(-4,-3){6.667}} % last line
\thinlines
\put(11.111,3.333){\framebox(6.667,10){}}
\put(2.222,3.333){\framebox(6.667,10){}}
\end{picture}
\setlength{\unitlength}{1pt}
\caption{The construction of the Gutenberg page design}
\label{flpage:lgut}
\end{figure}

\index{proportion!typeblock|)}
\index{proportion!page|)}
\index{spread|)}


\section{The typeblock} \label{sec:tblock}

\index{typeblock|(}
    The typeblock is not just a rectangular block of text. If the typeblock
does consist of text, then this will normally be broken up into 
paragraphs\index{paragraph}; it is not good authorial style to have 
paragraphs that are longer than a page. Also, the typeblock may include 
tables\index{table} and illustrations\index{illustration} which provide 
relief from straight text. Some pages may have chapter or section 
headings\index{heading} on them which
will also break the run of the text. In general the typeblock will
be a mixture of text, white space, and possibly non-text items.

    Consider a typeblock that includes no illustrations\index{illustration} 
or tables\index{table}.
The lines of text must be laid out so that they are easy to read.
Common practice, and more recently psychological testing, has shown that
long lines of text are difficult to read. Thus, there is a physiological
upper limit to the width of the typeblock. From a practical viewpoint,
a line should not be too short because then there is difficulty in justifying
the text.

    Experiments have shown that the number of characters in a line of
single column\index{column} text on a page should be
in the range 60 to 70 for ease of reading. The range may be as much
as 45 to 75 characters but 66 characters is often
considered to be the ideal number. Much shorter and the eye is dashing
back and forth between each line. Much longer and it is hard to pick up the
start of the next line if the eye has to jump back too far --- the same line
may be read twice or the following line may be inadvertently jumped over.
For double column\index{column!double} text the ideal number of characters 
is around 45, give or take 5 or so.

    Bringhurst\index{Bringhurst, Robert}~\autocite{BRINGHURST99} gives a 
method for determining the number
of characters in a line for any font\index{font!measuring}: 
measure the length of the lowercase alphabet and use a 
copyfitting\index{copyfitting} 
table that shows for a given alphabet 
length and line length, the average number of characters in that line.
 Table~\ref{tab:copyfitting} is an
abridged version of Bringhurt's copyfitting table.
For example, it suggests that a font with a length of \U{130}{pt} should be
set on a measure of about \U{26}{pc} for a single column\index{column!single} 
or in an \U{18}{pc} wide column if there are multiple\index{column!multiple} 
columns.
 

\begin{table}
%%%\DeleteShortVerb{\|}
\centering
\caption{Average characters per line. The bold numbers mark the
  combination that gives 60--70 chars pr line, whereas the italic
  marks the corresponding values around 45 chars. Abridged version of
  corresponding table in \cite{BRINGHURST99}.} \label{tab:copyfitting}
\begin{tabular}{r|rrrrrrrr} \hline
Pts. & \multicolumn{8}{c}{Line length in picas} \\
     & 10 & 14 & 18 & 22 & 26  & 30  & 35 & 40 \\ \hline
80   & \textit{40} & \textbf{56} & \textbf{72} & 88 & 104 &     &    &    \\
85   & \textit{38} & \textit{53} & \textbf{68} & 83 & 98 & 113 &    &    \\
90   & \textit{36} & \textit{50} & \textbf{64} & 79 & 93 & 107 &    &    \\
95   & 34 & \textit{48} & \textbf{62} & 75 & 89 & 103 &    &    \\
100  & 33 & \textit{46} & \textbf{59} & \textbf{73} & 86 & 99 & 116 &   \\
105  & 32 & \textit{44} & 57 & \textbf{70} & 82 & 95 & 111 &   \\
110  & 30 & \textit{43} & 55 & \textbf{67} & 79 & 92 & 107 &   \\
115  & 29 & \textit{41} & 53 & \textbf{64} & 76 & 88 & 103 &   \\
120  & 28 & \textit{39} & \textit{50} & \textbf{62} & 73 & 84 & 98 & 112 \\
125  & 27 & 38 & \textit{48} & \textbf{59} & \textbf{70} & 81 & 94 & 108 \\
130  & 26 & 36 & \textit{47} & 57 & \textbf{67} & 78 & 91 & 104 \\
135  & 25 & 35 & \textit{45} & 55 & \textbf{65} & 75 & 88 & 100 \\
140  & 24 & 34 & \textit{44} & 53 & \textbf{63} & 73 & 85 & 97 \\
145  & 23 & 33 & \textit{42} & 51 & \textbf{61} & \textbf{70} & 82 & 94 \\
150  & 23 & 32 & \textit{41} & \textit{51} & \textbf{60} & \textbf{69} & 81 & 92 \\
155  & 22 & 31 & \textit{40} & \textit{49} & 58 & \textbf{67} & 79 & 90 \\
160  & 22 & 30 & 39 & \textit{48} & 56 & \textbf{65} & 76 & 87 \\
165  & 21 & 30 & 38 & \textit{46} & 55 & \textbf{63} & 74 & 84 \\
170  & 21 & 29 & 37 & \textit{45} & 53 & \textbf{62} & 72 & 82 \\
175  & 20 & 28 & 36 & \textit{44} & 52 & \textbf{60} & \textbf{70} & 80 \\
180  & 20 & 27 & 35 & \textit{43} & 51 & 59 & \textbf{68} & 78 \\
185  & 19 & 27 & 34 & \textit{42} & \textit{49} & 57 & \textbf{67} & 76 \\
190  & 19 & 26 & 33 & 41 & \textit{48} & 56 & \textbf{65} & 74 \\
195  & 18 & 25 & 32 & 40 & \textit{47} & 54 & \textbf{63} & 72 \\
200  & 18 & 25 & 32 & 39 & \textit{46} & 53 & \textbf{62} & \textbf{70} \\ 
220  & 16 & 22 & 29 & 35 & \textit{41} & \textit{48} & 56 & \textbf{64} \\
240  & 15 & 20 & 26 & 32 & 38 & \textit{44} & 51 & 58 \\
260  & 14 & 19 & 24 & 30 & 35 & 41 & \textit{48} & 54 \\
280  & 13 & 18 & 23 & 28 & 33 & 38 & \textit{44} & 50 \\
300  & 12 & 17 & 21 & 26 & 31 & 35 & 41 & \textit{47} \\
320  & 11 & 16 & 20 & 25 & 29 & 34 & 39 & \textit{45} \\
340  & 10 & 15 & 19 & 23 & 27 & 32 & 37 & 42 \\
\hline
\end{tabular}
%%%\MakeShortVerb{\|}
\end{table}

    Morten H{\o}gholm\index{H{\o}gholm, Morten} has done some curve fitting
to the data. He determined that the expressions
\begin{equation}
L_{65} = 2.042\alpha + 33.41 \label{eq:L65}
\end{equation}
and
\begin{equation}
L_{45} = 1.415\alpha + 23.03 \label{eq:L45}
\end{equation}
fitted aspects of the data, where $\alpha$ is the length of the alphabet
in points, and $L_{i}$ is the suggested width in points, for a line with
$i$ characters (remember that \U{1}{pc} = \U{12}{pt}).

   The vertical height of the typeblock should be constant from page to
page. The lines of text on facing pages should be aligned
horizontally across the spine, which also means that they will be at
the same place on both sides of a leaf. Alignment across the spine
means that the eye is not distracted by an irregularity at the centre
of a spread, and leaf alignment stops ghosting of text through a thin page,
giving a crisper look to the work. 
So, the spacing between lines should
be constant. This implies that the depth of the typeblock should be an
integral multiple of the space required for each line; that is, be specified
as a multiple of the leading. A ten point type, for example, will normally
have two points between lines, to give a leading of 12 points. This can be
written as 10/12. Usefully, one pica is 12 points so with a \U{12}{pt} leading
vertical distances can be conveniently expressed in picas 
(one pica per line). Another implication of this is that any space left 
for illustrations\index{illustration} or tables\index{table}, or
the amount of space taken by chapter and section headings\index{heading} 
should also be
an integer multiple of the leading\index{leading}.

    A ten point type set solid is described as 10/10. The theoretical
face of the type is ten points high, from the top of a \emph{d} to the bottom
of a \emph{p}, and the distance of the baseline of one row of text to the
next row of text is also ten points. Note that if a \emph{p} is vertically
above a \emph{b} then the ascender of the \emph{b} will meet the descender
of the \emph{p}. To avoid this, the vertical separation between baselines 
is increased above the type size. Adding two extra points of vertical space
allows the text to breathe, and gives a leading of 12 points. Few fonts
read well when set solid. Typical settings are 9/11, 10/12, 11/13 and 12/15.
Longer measures require more leading than shorter ones, as do darker 
and larger
fonts compared with lighter and smaller fonts. More leading is also
useful if the text contains many super-\index{superscript} or 
sub-scripts\index{subscript}, or many uppercase letters.

\index{typeblock|)}


\subsection{Page color}

    One of the aims of the typographer is to produce pages that are uniform
in `color'\index{page color}. By this they mean that the 
typeblock\index{typeblock} has a reasonably constant
grayness, not being broken up by too much white space which is a distraction
to the reader. There will be white space around headings\index{heading}, 
which is acceptable as a heading is meant to attract attention. 
There may be white space between paragraphs\index{paragraph}, and this is 
usually under the control of the designer. But there 
may be vertical rivulets, or even rivers, of white space when the 
interword spaces on adjacent lines coincide; fixing this usually 
requires some handwork, either
by the author changing his wording so as to alter the location of
the spaces, or by the typesetter tweaking a little bit. 

    Another form of distraction is if too many lines end with hyphens, or
several adjacent lines start or end with the same text; not only does
this cause a stack of identical characters but will make it harder for 
the reader to reliably jump to the next line.

    The main font used controls the depth of the color of a page. To
see what color is produced by a particular font it is necessary to look
at a fairly long, preferably a page, piece of normal text. Fonts from
different families produce different colors, and so may mixed fonts from 
the same family. You can try this yourself by typesetting the same page
in, say, Computer Modern Roman, Italic, and Sans-serif fonts.
The books by Rogers~\autocite{ROGERS43}, Lawson~\autocite{LAWSON90},
Dowding~\autocite{DOWDING98}, and 
Morison~\autocite{MORISON99} all show pages set in many different fonts.

    

\subsection{Legibility}

    One of the principle requirements on the typography of a document is 
that the document is \emph{legible}. Legibility\index{legibility} 
means that the document 
is designed to be easily read under a certain set of circumstances. 
The criteria for
legibility on a poster that is placed on the side of a bus, for example, are
different from those that apply to a book to be read while sitting in an
easy chair. Essentially, the viewer should be able to read the document
with no physical strain caused by the appearance, but the contents, 
of course, may lead to anything ranging from acute mental strain to 
extreme boredom.

    Typefaces\index{typeface} and the layout of the 
typeblock\index{typeblock} must be 
chosen to optimise between legibility and `artistic' presentation. 
The design of the document should be almost invisible, giving full 
compliments to the author's communication. However, if you are a master, 
like Hermann Zapf\index{Zapf, Hermann}~\autocite{ZAPF00}, you can break the rules.

    

\subsubsection{Typefaces}
\index{typeface}

    The first European letter forms that have survived are Greek inscriptions
carved into stone. These were freehand carvings with thin strokes. In time,
the lettering became thicker and serifs started to appear. The Romans
picked up on this later style of letterform. In carving inscriptions, they
first wrote the inscription on the stone using a broad, flat brush. This
naturally led to serifs and differing thicknesses of the letter strokes,
depending on the angle of the stroke with respect to the movement and
orientation of the brush. Similarly the written letterforms included
serifs.

    Between the Roman times and Gutenberg there were many changes and
experiments in European letterforms. The scribes used different scripts
for titles, subheads, continuous text, illuminated initial letters, and so
on. In time, two families of letterforms evolved, 
called \emph{majuscules}\index{majuscule} and \emph{minuscules}\index{minuscule}. 
The former were larger and more formal, while the latter
were smaller and less formal. We now call these two divisions uppercase and
lowercase. The uppercase derives from Roman times, while the lowercase
acquired its fundamental form during the reign of the Holy Roman Emperor
Charlemagne\index{Charlemagne} a thousand years later. In order to 
promote communication
throughout his wide flung empire the Anglo Saxon Benedictine monk 
Alcuin\index{Alcuin}, at the behest of Charlemagne, established a common
script to be used; this is now called Carolingian 
minuscule\index{minuscule!Carolingian}.
A further division also appeared, between black letter (what is
commonly referred to as Gothic or Old English) type and the roman type.

    These types were all upright. Italic\index{italic} letterforms were 
cut in Italy
in the early sixteenth century, as a more cursive style. Initially these were
lowercase only, used in conjunction with uppercase roman. By the end of
the century, sloped roman capitals were also in use with italic.

\index{serif versus sans-serif type|(}
\index{font!seriffed|(}
\index{font!sans|(}
    The late nineteenth century saw the appearance again of 
Sans-serif\index{Sans-serif}
typefaces.

    Looking carefully at seriffed\index{serif} and sans fonts it is apparent that
the serifs have three main functions:
\begin{enumerate}
\item They help to keep letters apart.
\item At the same time, they help to keep letters in a word together. This
  helps with legibility\index{legibility} as research has shown that 
  we tend to recognize
  words by the shape of the word rather than by individual characters.
\item They help to differentiate between individual but similar letters.
\end{enumerate}

Long experience has shown that a seriffed font is easier to 
read\footnote{This is actually somewhat contentious as some take the view
that with enough practice, Sans-serif is just as easy to read.}
than a
Sans-serif font, particularly if part of the text is obscured. You can
try an experiment yourself to verify this. Try writing a phrase, once
using a Sans-serif font and then with a seriffed font. Cover up
the top halves of the two phrases and try to make out what they say. Then
repeat this, except this time cover up the bottom halves of the phrase.
Which is easier to read? Here are some example characters, firstly in sans-serif:
\begin{center}
{\Huge\sffamily a c l m n p q o}
\end{center}
and then in roman:
\begin{center}
{\Huge a c l m n p q o}
\end{center}

    Sans-serif fonts often require context to decipher the word. 
For example~\autocite{MCLEAN80},
seeing this in isolation
\begin{center}
{\Huge\sffamily lll}
\end{center}
does it stand for `Ill', `one hundred and eleven', `three', or something
completely different like a dingbat or a set of cricket stumps?

    
    There are three generally agreed legibility\index{legibility} 
principles for setting text for
continuous reading.

\begin{enumerate}
\item \emph{Sans-serif type is intrinsically less legible than seriffed type}~\autocite{WHEILDON95}.

    We have already seen that this is the case --- there is more variety
among seriffed letters than among sans-serif letters. Further, serifs
perform other functions as well, such as binding letters together within
a word.

    This is not to say that a sans-serif letterform is always more illegible
than a roman one. A poor seriffed form can be much more illegible than
a well used good sans-serif. In general, there is an illegibility factor
associated with sans-serif that must be borne in mind; for general
\emph{continuous} reading, a good seriffed form is more likely to be
easy on the eye than a good sans form.

\item \emph{Well designed upper- and lowercase roman type is easier to read than
any of its variants.}

    This is a guiding principle with many exceptions. Among the variants
can be considered to be italic and bold types. These have usually been
designed for a special purpose, like emphasing\index{emphasis} 
certain pieces of text, rather
than for general legibility. Some italic types, though, are as legible as their
roman counterparts. In the seventeenth century many books were set entirely
in italic, but we have become accustomed to the roman type.

\item \emph{Words should be set closer together than the space between lines.}

    All text is a mixture of ink and white space. The eye, when reading, 
tends to jump over the white spaces. Given a choice between two spaces, it 
will tend to jump over the smaller of the two. If the word spacing is greater
than the line spacing, then you can find yourself skipping from one line
to the next before finishing the first one.

    Further, if the lines are too long, then when the eye jumps back from
the end of one line to the start of the next, it may have difficulty in 
picking up the correct one.

    Text lines are justified by altering the inter-word spacing, and possibly
by hyphenating the last word on the line if the spacing would be too bad
otherwise. Sans-serif fonts often look best if set ragged right, as this will
keep the inter-word spacing constant. Text set in narrow 
columns\index{column!narrow} also often
looks best when set ragged right.


\end{enumerate}

\subsubsection{Seriffed versus Sans-seriffed fonts}


    As noted earlier there seems to be a permanent debate over the use
of seriffed and sans fonts. You will have to make up your own mind as
to what is best for any particular work, but here are a few general
comments from some of the literature on the subject.

\begin{itemize}
\def\makelabel#1{\noindent #1}
\item[Bohle~\autocite{BOHLE90}] notes: Readers prefer a roman typeface for body
  type because they are most used to seeing that face~\autocite{REHE72}.
  Roman type may well also be more readable than sans serif faces because
  the serifs help connect the letters to form the word shape when
  we read~\autocite{REHE72}.

\item[Craig~\autocite{CRAIG92}] says: You will find that the serifs on a typeface
  facilitate the horizontal flow necessary to comfortable reading.

\item[Degani~\autocite{DEGANI92}] in a study of pilots reading checklists
  in emergency cockpit situations decided that sans serif faces were
  better than serif faces.

\item[Schriver~\autocite{SCHRIVER97}] notes: Serif and sans serif typefaces 
  are likely to be equally preferred by 
  readers~\autocite{HARTLEY83,TINKER63}
  and read equally quickly~\autocite{GOULD87,HARTLEY83,ZACHRISSOM69}.
  Serif faces may be easier to read in continuous text than sans
  serif faces~\autocite{BURT59,HVISTENDAHL75,ROBINSON71,WHEILDON95}.

\item[Wheildon~\autocite{WHEILDON95}] did a series of studies with around
250 readers in Sydney, Australia, asking them to rate serif and sans fonts
in a variety of uses. Among the many results he reported:
\begin{itemize}
  \item More than five times as many readers are likely to show good
  comprehension when a serif body type is used instead of a sans serif
  body type.
  \item The top half of [uppercase] letters is more recognizable than
  the bottom half.
  \item There is little difference in legibility between headlines
  [section titles] set in serif and sans serif typefaces, or between
  roman and italic.
  \item Headlines set in capital letters are significantly less legible
  than those set in lowercase.
\end{itemize}

\end{itemize}

    The consensus, such as it is, seems to lean towards seriffed
typefaces for continuous reading, but for titling the choice is
wide open.


    To finish off in a lighter vein,  
    Daniel Luecking\index{Luecking, Daniel} had this to say on the subject
in a posting to \pixctt\ in January 2002.
\begin{quotation}
    It is often conjectured that seriffed fonts are easier to read because
the serifs contribute more points of difference between words. This is
often countered with the conjecture that they are easier to read because 
that is what we are used to reading. But no one can doubt that words
like Ill, Iliad and Illinois in a sans-serif font 
[e.g., \textsf{Ill, Iliad and Illinois}] are going to cause the eye/brain 
system at last momentary confusion while it sorts out which plain vertical
lines are uppercase i's and which lowercase L's.

    I don't know if this contributes anything, but I can say unequivocally
that serif fonts are somwhat easier to read upside down than sans-serif,
but sans-serif is far easier to read mirored than serif. (I spent much
of my time as a child reading comics on the floor with my brother. As
he hated reading any way but straight on, we faced in different directions
and I saw the page upside down. I tried mirror reading just to see if
I could do that as easily. Serif fonts were almost impossible, sans-serif
actually quite easy.)

\end{quotation}

    He later expanded on the mirror reading to me as follows.
\begin{quotation}
Here's an interesting (to me) anecdote about mirror reading: I was waiting
in line at an airport lunch counter, reading the menu posted on the wall,
when it suddenly struck me as odd that the menu was on the wall opposite
(so that one had to turn away from the counter to read it). Then I
realized in a sort of flash that I was reading it from a mirror. I turned
to look at the real menu and was momentarily disoriented (while my brain
turned itself around I guess) before I could read the actual menu. That
was when I first ran some tests to see why that was so easy to read and
other mirror writing was not. It seemed to be serif vs sans-serif, but
it might also be the typical letter forms: the typical serif lowercase
`a', the one with the 'flag' above the bowl [e.g., a], is particularly 
difficult to recognize compared to the simple `circle plus stick' 
[e.g., \textit{a}] form.

    Some sort of dyslexia (or eulexia), no doubt, when backwards
words are nearly as easy to read as normal ones.
\end{quotation}


\index{font!sans|)}
\index{font!seriffed|)}
\index{serif versus sans-serif type|)}



\index{font!sans|)}
\index{font!seriffed|)}


\subsection{Widows and orphans}

    Inconvenient page breaks can also cause a hiatus in the reader's perusal
of a work. These happen when a page break occurs near the start or end of
a paragraph\index{paragraph}\index{lonely line}. 

    A \emph{widow}\index{widow} is where the last line of a 
paragraph\index{paragraph} is the first line on the page. The term is 
sometimes also used to refer to when the last word in a 
paragraph\index{paragraph} is on a line by itself. A widow looks forlorn.
In German they are called \textit{Hurenkinder}\index{Hurenkinder} --- 
whores' children --- which seems rather cruel to me.
As Robert Bringhurst said, `A widow has a past but no future'. 
Typographically, widows should be avoided as they are a weak start to a page 
and may optically destroy the page and type rectangle. However, a single widow
is not too troubling if the header includes a rule across the width of the 
typeblock. Especially to be 
avoided are widows that are the only line on a page, for example at the end 
of a chapter\index{chapter}. Five lines on the last page of a 
chapter\index{chapter} is a reasonable minimum. 
Jan Tschichold~\autocite{TSCHICHOLD91} claims that Hurenkinder can always 
be avoided by even if a recto (verso) page must be made a line shorter or
longer than the corresponding verso (recto) page, which he considers to be 
less of an affront than widows.

    An orphan\index{orphan} is not nearly so troubling to typographers as a 
widow. An \emph{orphan} is where the first one or two lines of a 
paragraph\index{paragraph} are at the bottom of a page. In German they are 
called \textit{Schusterjungen}\index{Schusterjungen} --- cobbler's apprentices.
Bringhurt's memory trick for orphans is, `An orphan has a future but no 
past'.\index{lonely line} 

\subsection{Paragraphs and versals} \label{sec:versal}

\index{paragraph|(}

    Early books did not have paragraphs as we know them nowadays; the text
was written continuously, except for a break at a major division like the
start of a new book in a bible. Instead
the scribes used a symbol like \P\ 
(the pilcrow)\index{pilcrow (\P)} 
to mark the beginning of
paragraphs. This symbol is derived from the Greek $\Pi$, for
\textit{par\'{a}graphos}. Mind you, they often did not use any punctuation
at all and were sparing in their use of uppercase letters, 
so you might have seen something like this\footnote{But probably not.
The two `paragraphs' are Latin abecedarian sentences.}

\begin{quote}
usque \P\ te canit adcelebratque polus rex gazifier hymnis 
       \P\ transzephyrique globum scandunt tua facta per axem
\end{quote}

    Often the \P\ was colored red by the rubricators\index{rubricator} and the
scribe, or printer, would leave a blank space for the rubricator to add the \P.
This did not always happen and the start of a paragraph eventually became
marked by a space rather than a symbol.

    Nowadays paragraphs are ended by stopping the line of text at 
the end of the
paragraph, and then starting the next paragraph on a new line. The question
then becomes: how do you indicate a new paragraph when the last line of the
previous paragraph fills up the measure? There are two solutions, which 
unfortunately you
sometimes see combined. Either indent the beginning of the first line of
each paragraph, or put additional vertical space between the last and
first lines of paragraphs.

    The traditional technique, which has served well for centuries, is to
indent the first line of a paragraph\index{paragraph!indentation}. 
The indentation need not be large, about an em will be enough, but 
more will be required if the typeblock\index{typeblock} is wide.

    The other method is used mainly in business letters and is a recent 
invention. The first lines of paragraphs\index{paragraph!indentation} 
are not indented and typically one blank line is left between paragraphs. 
This may
perhaps be acceptable when using a typewriter, but seems to have no real
justification aesthetically. There is also the problem when a paragraph
both ends with a full line and ends a page. As the next paragraph then starts
at the top of the next page, the blank line separating the two paragraphs
has effectively dissappeared, thus leaving the reader in a possible state
of uncertainty as to whether the paragraph continues across the page break
or not.

    If the paragraph is the first one after a heading\index{heading}, 
then there
is no need to indicate that it is a new paragraph --- it is obvious from its
position. So, the first paragraph after a heading\index{heading} 
need not be indented\index{paragraph!indentation}, and for some centuries now 
the tradition is not to indent after a heading.
In some novels only chapters are headed yet each chapter is broken into
sections by putting additional vertical blank space between the sections.
Like nonindented paragraphs\index{paragraph!indentation}, 
this can cause problems where a section division
coincides with a page break. In this case, typographers sometimes use a
decoration to separate sections (for example, a short centered row of a few
asterisks).
   

%%%%%%%%\clearpage
%\subsection{Versals}

\drop{S}{}\textsc{ome typographers}\index{versal|(} 
like to start the first paragraph in a chapter
with a versal. A \emph{versal} is a large initial letter, either raised or
dropped. This comes from the scribal tradition of illuminating the first
letter of a manuscript. The versal may be raised or dropped, as already noted,
or it may be placed in the margin\index{margin}, or otherwise treated in 
a special manner.

\versal{S}\textsc{ome versals,} especially dropped versals, are very difficult
to typeset correctly. Many attempts of this kind are abject failures, so
be warned. For example, compare the dropped versals at the start of these
first two paragraphs. They are both of the same letter and font, yet the first
one is horrible compared to the one starting this paragraph.


\noindent {\huge A} \textsc{raised versal} is often easier to use to 
start a paragraph than a dropped versal. 
However, a raised versal should only be used
where there is naturally some vertical space above it. As you can see, extra
spacing has had to be inserted before this paragraph to accomodate the versal.
There are still problems with typesetting a raised versal but as these tend
to be subtler than with a dropped versal, readers are less likely
to notice problems.

Typically, small caps are used for a little while following a versal to 
provide a transition between the large versal font and the normal body font.
These should not continue throughout the first line as this tends to divorce
it from the remainder of the paragraph. \index{versal|)}

\noindent \textsc{Another way of starting} a paragraph is to use small
caps for the first few words. The font difference highlights the start
of the paragraph but in a much quieter manner than a versal does. Using
normal sized uppercase instead of the small caps is too much of a 
contrast with the lowercase.

\index{paragraph|)}

\subsection{Footnotes}

\index{footnote|(}
    Footnotes are considered to be part of the typeblock\index{typeblock}. 
They are typeset in the space allocated for the typeblock\index{typeblock}, 
in contrast to footers\index{footer}
which are typeset below the typeblock\index{typeblock}.

    Footnotes are normally set in the same type style as the 
typeblock\index{typeblock}. That is, if an upright seriffed font is used 
for the typeblock\index{typeblock}, it is
also used for the footnote. The
type size is smaller to distinguish the note from the body text and often
the leading in the footnote is also reduced from that in the main text body.
The bottom footnote line should be at the same height as the bottom line
of the typeblock\index{typeblock}. This usually requires some adjustment 
of the vertical space before the footnotes.

    A vertical blank space is often used to set off the footnotes from the
main text, and sometimes a short horizontal line is also used as demarcation.

\index{footnote|)}

\section{Folios}

\index{folio|(}

    The word \emph{folio} is a homonym. It can mean a leaf 
(two back-to-back pages) in a book, the size of a book or a book of
folded sheets (as in Shakespeare's first folio), or the printed page number
in a book. Here I use folio in this last sense.

    Documents should have folios, at a minimum to help the reader know where
he is. Occasionally books have their folios placed near the spine but this
positioning is unhelpful for navigation. The more usual positions are
either centered with respect to the typeblock\index{typeblock} or aligned 
with the outside of the typeblock\index{typeblock}, and sometimes even in 
the outside margin\index{margin!outer}. The folios
can be either at the top or bottom of the page but at least on pages 
with chapter\index{chapter} openings are normally placed at the bottom of 
the page so that they do not distract from the title text.

    Every page in a book is numbered, even if the page does not have a folio. 
In books, the folios for the front material are often in roman numerals.
The \pixmainmatter\ and \pixbackmatter\ folios are arabic numerals, 
with the sequence
starting from 1 after the \pixfrontmatter. In certain technical documents,
folios may be in the form of chapter number\index{chapter!number} and 
page number, with the page number starting from 1 in each new chapter. 
Other folio schemes are possible but unusual.

    Folios should be placed harmoniously with respect to the 
typeblock\index{typeblock} and page margins\index{margin}. The font used for 
the folios need not be the same as that for the typeblock\index{typeblock} 
but must at least be complementary and non-intrusive.

\index{folio|)}

\section{Headers and footers}

\index{header|(}\index{footer|(}
    Headers and footers are repetitive material that is placed at either 
the head or the foot of the page. Typically, folios\index{folio} are headers 
or footers, but not always as sometimes they are placed in the 
margin\index{margin} at or below the first line in the 
typeblock.\index{typeblock}

   For the time being I will not distinguish between headers and footers and 
just use the word header. Sometimes the header is purely decorative (apart
from a folio\index{folio}) like a horizontal line or some other non-textual 
marking. Normally, though, they have a functional use in helping the 
reader locate himself in the document.

    The most ubiquitous header is one which gives the title of the document.
If this is the only header, then I consider this to be decorative rather
than functional. As a reader I know what document I am reading and do not
need to be reminded every time I turn a page. More useful are headers that
identify the current part of the document, like a chapter\index{chapter} 
title or number.
When you put the document down and pick it up later to continue reading, these
help you find your place, or if you need to refer back to a previous chapter
for some reason, then it is a boon to have a chapter heading\index{heading} 
on each 
spread. The minimally functional headers are where the document title
is on one page and the chapter heading is on the facing page. In more technical
documents it may be more useful to have headers of chapter and section titles
on alternate pages. 

    Occasionally both headers and footers are used, in which case one normally
has constant text, like a copyright\index{copyright} notice. 
I have the feeling that using the latter is only functional for the 
publishers of the document when they fear photocopying or some such.

    The header text is usually aligned with the spine side 
of the typeblock\index{typeblock}, but may be centered on top of the 
typeblock\index{typeblock}. In any event, it should not interfere with 
the folio\index{folio}. The type style need not be the same as the style 
for the typeblock\index{typeblock}. For example, headers could be set in italic
or small caps, which, however, must blend with the style used for the 
typeblock.\index{typeblock}

\index{footer|)}\index{header|)}

\section{Electronic books}
\index{electronic books|(}\index{Ebook|see{electronic books}}

    For want of a better term I am calling electronic books, or Ebooks, 
those documents intended to be read on a computer screen. The vast bulk
of Ebooks are in the form of email but I'm more interested here in 
publications that are akin to hardcopy reports and books that require
more time than a few minutes to read.

    Unlike real books which have been available for hundreds of years there
is virtually no experience to act as a guide in suggesting how Ebooks should
appear. However, I offer some suggestions for the layout of Ebooks, based on
my experience of such works. 
Not considered are internal navigation aids
(e.g., hyperlinks) within and between Ebooks, nor HTML documents where
the visual appearance is meant to set by the viewing software and not 
by the publisher.

    The publication medium is obviously very different --- a TV-style 
screen with limited resolution and pretty much fixed in position versus
foldable and markable paper\index{paper} held where the reader finds it best.
These differences lead to the following suggestions.

    A book can be held at whatever distance is comfortable for reading, even
when standing up.
The computer user is normally either sitting in a chair with the monitor
on a desk or table, or is trying to read from a laptop, which may be 
lighter but nobody would want to hold one for any length of time. To try
and alleviate the physical constraints on the Ebook reader the font size
should be larger than normal for a similar printed book. This will provide
a wider viewing range. A larger font will also tend to
increase the sharpness of the print as more pixels will be available for
displaying each character.
    The font size should not be less than \U{12}{pt}. The font may have to be
more robust than you would normally use for printing, as fine hairlines 
or small serifs will not display well unless on a high resolution screen.

    I find it extremely annoying if I have to keep scrolling up and down
to read a page. Each page should fit within the screen, which means that
Ebook pages will be shorter than traditional pages. 
A suggested size for an Ebook page, in round numbers, is 
about 9 by 6 inches~\autocite{ADOBEBOOK} or 23 by 15 centimetres overall.

    The page design for printed books is based on a double spread. For
Ebooks the design should be based on a single page. The 
typeblock\index{typeblock} must be centered on the page otherwise it gets 
tiring, not to mention aggravating, if your eyes have to 
flip from side to side when moving from one page to the next. Likewise
any header\index{header} and the top of the typeblock\index{typeblock} 
must be at a constant height on the screen. A constant position for the 
bottom of the text is not nearly so critical.

    It is more difficult with an Ebook than with a paper book 
to flip through it to find a particular place. 
Navigation aids --- headers\index{header} and footers\index{footer} ---
are therefore more critical. Each page should have both a chapter 
(perhaps also a section) header\index{header} title and a page number. 
Note that I'm not considering HTML publications.

    Many viewers for Ebooks let you jump to a particular page. The page
numbers that they use, though, are often based on the sequence number from
the first page, not the displayed folio\index{folio}. In such cases it 
can be helpful to arrange for a continuous sequence of page numbers, 
even if the folios\index{folio} are printed using different styles. 
For example, if the \pixfrontmatter\ uses roman numerals and the 
\pixmainmatter\ arabic numerals and the last page of the
\pixfrontmatter\ is page xi, then make the first page of the \pixmainmatter\ 
page 12.

    I see no point in Ebooks having any blank pages --- effectively the
concept of recto and verso pages is irrelevant.

    Some printed books have illustrations\index{illustration} that are 
tipped in, and the tipped in pages are sometimes excluded from the 
pagination. In an Ebook the illustrations\index{illustration} have to 
be `electronically tipped in' in some fashion, either
by including the electronic source of the illustrations\index{illustration} 
or by providing some navigation link to them. Especially in the former case, 
the tipped in elements should be included in the pagination.

    Don't forget that a significant percentage of the population is 
color-blind.\index{color|(}\index{color!blind} 
The most common form is a reduced ability to distinguish
between red and green; for example some shades of pink may be perceived
as being a shade of blue, or lemons, oranges and limes may all appear to
be the same color. Along with color-blindness there may be a reduced
capacity to remember colors.

    I have seen Ebooks where color has been liberally used to indicate, say, 
different revisions of the text or different sources for the data in a graph. 
Unless the colors used are really distinctive 10\% or more of the potential 
readership will be lost or confused. Further,
Ebooks may be printed for reading off-line and if a non-color printer is
used then any colors will appear as shades of grey; these must be such that
they are both readily distinguishable and legible. Yellow on white is almost
as difficult to read as off-white on white or navy blue on black, all of
which I have seen on web sites but rarely have I seen the text after I 
have tried to print the page.

\index{color|)}
\index{electronic books|)}


\chapter{Styling the elements}

    A book should present a consistent typographic style throughout, although
some elements, principally those in the \pixfrontmatter\ and \pixbackmatter\ 
may be treated slightly differently than the main body. 

    Much of this chapter is based on my interpretation of my namesake's 
work~\autocite{ADRIANWILSON93} and the \btitle{Chicago Manual of Style}~\autocite{CMS}.

\section{\prFrontmatter}

\subsection{Title pages}

    The main and half-title pages are the gatekeepers to the book. As such,
they need to be welcoming and give an indication of the `look and feel'
of the contents. You don't want to scare off potential readers before they
have even cursorily scanned the contents.

    The half-title\index{page!half-title}\index{half-title page}, or bastard 
title\index{bastard title|seealso{half-title}}, page contains just the 
title of the
work, which is traditionally set high on the page --- perhaps about as high
as the chapter openings. The title page\index{page!title}\index{title page} 
itself presents the title in full, 
the author and maybe the illustrator or
other names likely to attract the reader, and perhaps the publisher and date.
Both the title pages are recto but the full title may be a double spread.
The full title layout in particular must be both attractive and informative.

    Quoting from Ruari McLean~\autocite[p. 148]{MCLEAN80} in reference to the 
title page he says:
\begin{quotation}
    The title-page states, in words, the actual title (and sub-title, if 
there is one) of the book and the name of the author and publisher, and
sometimes also the number of illustrations, but it should do more than that.
From the designer's point of view, it is the most important page in the
book: it sets the style. It is the page which opens communication with the
reader\ldots

    If illustrations play a large part in the book, the title-page opening 
should, or may, express this visually. If any form of decoration is used 
inside the book, e.g., for chapter openings, one would expect this to be
repeated or echoed on the title-page.

    Whatever the style of the book, the title-page should give a foretaste
of it. If the book consists of plain text, the title-page should at least 
be in harmony with it. The title itself should not exceed the width of the
type area, and will normally be narrower\ldots
\end{quotation}


\begin{figure}
\centering
\begin{showtitle}
\titleJT
\end{showtitle}
\caption{Title page design based on Ruari McLean's \btitle{Jan Tschichold: Typographer}} \label{fig:titleJT}
\end{figure}

\begin{figure}
\centering
\begin{showtitle}
\titleRF
\end{showtitle}
\caption{Title page based on a design by Rudolph Ruzicka for a book of Robert Frost's poetry} \label{fig:titleRF}
\end{figure}


\begin{figure}
\centering
\begin{showtitle}
\titleAM
\end{showtitle}
\caption{Title page based on a design by Will Carter for
\btitle{The Rime of the Ancient Mariner}} \label{fig:titleAM}
\end{figure}


\begin{figure}
\centering
\begin{showtitle}
\titlePM
\end{showtitle}
\caption{Title page design based on Nicholas Basbanes' \btitle{Gentle Madness}} \label{fig:titlePM}
\end{figure}


\begin{figure}
\centering
\begin{showtitle}
\titleAT
\end{showtitle}
\caption{Title page based on the design for \btitle{Anatomy of a Typeface}} \label{fig:titleAT}
\end{figure}


\begin{figure}
\centering
\begin{showtitle}
\titleLL
\end{showtitle}
\caption{Title page based on the design for \btitle{Lost Languages}} \label{fig:titleLL}
\end{figure}


\begin{figure}
\centering
\begin{showtitle}
\titleSW
\end{showtitle}
\caption{Title page based on the design for \btitle{The Story of Writing}} \label{fig:titleSW}
\end{figure}


\begin{figure}
\centering
\begin{showtitle}
\titleTMB
\end{showtitle}
\caption{Title page based on a design for the Folio Society's edition of
  \btitle{Three Men in a Boat} (first published in 1889)} \label{fig:titleTMB}
\end{figure}


\begin{figure}
\centering
\begin{showtitle}
\titleZD
\end{showtitle}
\caption{Title page based on a design for the Folio Society's edition of
  \btitle{Zuleika Dobson} (first published in 1911)} \label{fig:titleZD}
\end{figure}


\begin{figure}
\centering
\begin{showtitle}
\titleWH
\end{showtitle}
\caption{Title page based on a design for the Cambridge University 
Printer's Christmas book \btitle{Words in Their Hands}} \label{fig:titleWH}
\end{figure}


\begin{figure}
\centering
\begin{showtitle}
\titleGP
\end{showtitle}
\caption{Title page design for an annual International Federation for 
    Information Processing workshop} \label{fig:titleGP}
\end{figure}


\begin{figure}
\centering
%\begin{showtitle}
{\titleRB}
%\end{showtitle}
\caption{Title page of a Victorian booklet} \label{fig:titleRB}
\end{figure}



\subsubsection{Advertising}

    If the book is one of a series, or the author has been prolific, then
details of associated works may be provided. Titles by the same author
are usually set in italic, and a series title perhaps in small caps. The
font size should be no larger than for the main text. This advertising material
may be put on the copyright page but it is more often on the recto page
immediately after the half-title, before the title page, or on the verso of 
the half-title\index{half-title page}\index{page!half-title} page.

\subsubsection{Frontispiece}

    The traditional place for a frontispiece\index{frontispiece}, 
which may be the only
illustration in the book, is facing the title page. Every attempt must be
made to make the resulting double spread harmonious.

\subsection{Copyright page}

    The copyright\indextwo{copyright}{page} and related publishing 
information is set in small type
on the verso of the title page. In some instances the book designer's name may
be listed among the small print.

\subsection{Dedication}

    A dedication\index{dedication} is nearly always on a recto page and 
simply typeset. If 
pages are limited it could be placed at the top of the copyright 
page\indextwo{copyright}{page} instead.

\subsection{Foreword and preface}

    The same type size should be used for the headings 'foreword', `preface',
`acknowledgements', etc., and the similar ones in the \pixbackmatter. This
may be the same size as the chapter heads, or smaller. The body type should be
the same as for the \pixmainmatter\ text.
The foreword\index{foreword} starts on a recto page. It may face the 
copyright page\indextwo{copyright}{page}, or if 
there is a dedication\index{dedication} it will face the dedication's 
blank verso.

    The preface\index{preface}, which is the author's opening statement, 
is treated like a chapter opening, and commences on a recto page.

\subsection{Acknowledgements}

    If there are any acknowledgements\index{acknowledgements} and they 
require only a few sentences then they are often put at the end of the 
preface\index{preface}, if there is one. Otherwise the acknowledgements
should be treated as a distinct unit, like a foreword or preface, and commence
on a recto page.

\subsection{Contents and illustration lists}

    The table of contents\indextwo{contents}{list} is often laid out so 
that the page numbers
are not too distant from the titles, thus reducing the need for dotted lines,
even if this makes the contents block narrower than the text page width. 
An alternate strategy is to use more interlinear space between the entries
so that there is little or no difficulty in recognising which page
numbers belong to which titles. The parts of major sections should be
clearly separated to aid visual navigation.

   If the captions in an illustration list\indextwo{illustration}{list} 
are short, and it follows the
contents list then it should be set in the same style as the contents. 
If, however, the list is at the back of the book a smaller font size could 
be used.

\subsection{Introduction}

    The heading for an Introduction\index{introduction} is an unnumbered
chapter opening on a recto page. The body text is set in the same font 
as the main body text. 

\subsection{Part title page}

    If the book is divided into Parts\index{part} then the Part One 
page\indextwo{page}{part title} could either precede or follow the 
Introduction. Ideally the layout of the title pages for the Parts should
follow the layout of the book's half-title\indextwo{page}{half-title} page 
to provide a cohesiveness throughout the work.

\section{\prMainmatter}

\subsection{Chapter openings}

    Normally, chapters are the major divisions of a (volume of a) book, 
but if it is long
it may have higher divisions, such as parts, or even books. In any event,
chapters are numbered consecutively throughout the volume. The numbering
is usually arabic but if there are few chapters, or there are numbered
subdivisions within chapters, or the work has classical allusions, then
roman numerals might be appropriate.

    Chapters often begin on recto pages but if there are many short chapters
or reproduction cost is all important then they may begin on the page
immediately following the end of the previous chapter, or, in rare cases,
even on the same page if there is enough space. In any event, the first chapter
should start on a recto page.

\subsection{Mixed portrait and landscape pages}

    It is usual for all the pages in a document to be in 
portrait\index{orientation}\index{portrait}\indextwo{orientation}{portrait}
but this is not always possible if there are illustrations or tabular material
that is better displayed in 
landscape\index{landscape}\indextwo{orientation}{landscape} 
orientation rather than portrait.

    When a double spread consists of a portrait page and a landscape page, as 
in~figures~\ref{fig:portrait-landscape} and~\ref{fig:landscape-portrait}, there are
two choices for which way the landscape page is turned with respect to the
portrait page. In each case it seems more natural to me when the document 
has to be turned to the right to view the landscape page. 


    Figure~\ref{fig:landscape-landscape} shows double spreads where both pages
are in landscape orientation. Whichever way they face, they must both face in
the same direction so that the document has only to be turned once to read
both of them.

    The general rule for mixed landscape and portrait pages is that the
document is held in one position to read the 
portrait\indextwo{orientation}{portrait} pages and is turned
in a consistent direction, usually to the right, to read the 
landscape\indextwo{orientation}{landscape} pages. The key is consistency.

\begingroup
\newcommand*{\tht}{6}
\newcommand*{\twd}{8}
\newcommand*{\htwd}{4}
\setlength{\unitlength}{0.5cm}

\begin{figure}
\centering
\mbox{}\hfill
\begin{picture}(\twd,\tht)
  \put(0,0){\framebox(\htwd,\tht){\textsc{portrait}}}
  \put(\htwd,0){\framebox(\htwd,\tht){\rotatebox[origin=c]{90}{\textsc{landscape}}}}
\end{picture}
\hfill
\begin{picture}(\twd,\tht)
  \put(0,0){\framebox(\htwd,\tht){\textsc{portrait}}}
  \put(\htwd,0){\framebox(\htwd,\tht){\rotatebox[origin=c]{-90}{\textsc{landscape}}}}
\end{picture}
\hfill
\mbox{}

\caption[Portrait and landscape spreads]%
  {Portrait and landscape spreads: the layout on the left is preferable to the one on the right} \label{fig:portrait-landscape}
\end{figure}

\begin{figure}
\centering
\mbox{}\hfill
\begin{picture}(\twd,\tht)
  \put(\htwd,0){\framebox(\htwd,\tht){\textsc{portrait}}}
  \put(0,0){\framebox(\htwd,\tht){\rotatebox[origin=c]{90}{\textsc{landscape}}}}
\end{picture}
\hfill
\begin{picture}(\twd,\tht)
  \put(\htwd,0){\framebox(\htwd,\tht){\textsc{portrait}}}
  \put(0,0){\framebox(\htwd,\tht){\rotatebox[origin=c]{-90}{\textsc{landscape}}}}
\end{picture}
\hfill\mbox{}

\caption[Landscape and portrait spreads]%
  {Landscape and portrait spreads: the layout on the left is preferable to the one on the right}\label{fig:landscape-portrait}
\end{figure}

\begin{figure}
\centering
\mbox{}\hfill
\begin{picture}(\twd,\tht)
  \put(\htwd,0){\framebox(\htwd,\tht){\rotatebox[origin=c]{90}{\textsc{landscape}}}}
  \put(0,0){\framebox(\htwd,\tht){\rotatebox[origin=c]{90}{\textsc{landscape}}}}
\end{picture}
\hfill
\begin{picture}(\twd,\tht)
  \put(\htwd,0){\framebox(\htwd,\tht){\rotatebox[origin=c]{90}{\textsc{landscape}}}}
  \put(0,0){\framebox(\htwd,\tht){\rotatebox[origin=c]{-90}{\textsc{landscape}}}}
\end{picture}
\hfill\mbox{}

\caption[Double landscape spreads]%
  {Double landscape spreads: never use the layout on the right}\label{fig:landscape-landscape}
\end{figure}

\endgroup
\subsection{Extracts}

    Typographers use `extract'\index{extract} as a generic 
term for what I would think of as a quotation\index{quotation}. 
Essentially a quotation is an extract from some source.
Quotations\index{quotation}\index{extract} from other works, 
unless they are so short as to be 
significantly less than a line, should be set off from the main text. 
This could be
as an indented block, or by using a different type style or size, or a 
combination of these. A size difference of one or two points from the body
size is usually enough to be distinctive, for instance 11/13 or 10/12 point
for a 12/14 point body size. In any event, extra space, at least two or 
three points, should be inserted and below the extract.

\subsection{Footnotes and endnotes}

\renewcommand*{\thefootnote}{\fnsymbol{footnote}}
\let\oldfootnoterule\footnoterule
\renewcommand*{\footnoterule}{}
% the rule is reactivated in chapter 5


    This section is a synthesis of the views of Ruari McLean~\autocite{MCLEAN80},
Jan Tschichold~\autocite{TSCHICHOLD91} and 
Emerson Wulling~\autocite{WULLING-FOOTNOTES}.

    Footnotes\index{footnote} are ancilliary to the main material and 
expand in some way
upon the current theme. For instance, remarks that are too large or off the
main thread, or some side comment by the author, may be sunk to a footnote 
at the bottom of the page. By definition, a footnote is placed at the 
bottom of a page but, if it is long or space is short, may run over to a 
second, or even a third page. A footnote should have some immediate 
relevance to the reader.

Endnotes\index{endnotes}, which are collected together at the end of the 
document, include 
material similar to that in footnotes, but which is not of immediate interest.
If you use an endnote\index{endnote} it is safe to assume that only a small 
percentage of your readers will ever correlate it with the text.
 
    Within the text the presence of a footnote is indicated by a raised 
`reference mark'\index{reference mark} following the
word, phrase, or sentence to which it refers. The same mark is used 
to introduce the footnote at the bottom of the page.
The reference mark may be either a 
symbol or a number. For illustrative purposes I'm using symbols
as markers in this section.

   If there are many footnotes then it is convenient for the reader if numbers 
are used for the marks\index{footnote!mark}. The numbering may be continuous 
throughout the document,
or start afresh with each chapter; starting anew on each page may lead to
some confusion. When there is only an occasional footnote then symbols 
are usually preferred as reference marks.

    Endnotes may or may not be marked\index{endnote!mark} in the main text. 
If they are marked then numbers
should be used, not symbols. If there are both footnotes and marked endnotes 
then different series of marks must be used for the two classes of notes.

    There is some debate as to how reference marks should relate to the 
marked element in the main text. For example, should the mark immediately 
follow the element\footnote{This mark immediately folllows the main element.} 
or should there be a thin space\,\footnote{There is a thin space between this
mark and the element it is attached to.} separating the two. A convenient 
procedure is to use a thin space between the 
element\,\footnote{A thin space is used here.} and the mark when the end of 
the element is tall, and 
none\footnote{There is no extra space here.} when the end of the element is 
low. There is no need for any extra space between punctuation and a reference
mark.\footnote{There is no extra space here.}

    McLean, Tschichold, and Wulling are all agreed that there should not be
a rule separating the main text from any footnotes --- a space and change in
font size is sufficient to distinguish the two. The font size for 
footnotes\index{footnote!font size}
is usually two sizes smaller than the main font but with the same 
leading\index{leading}. For example if the main text is set 10/12 then 
footnotes would be set 8/10. Notes to tables, though, are often set even 
smaller; for instance at \U{6}{pt} or \U{7}{pt} in a \U{10}{pt} document. 
Each footnote should be introduced by the 
appropriate reference mark in the same font size as the note itself. If it is 
not raised then follow it by a period if it is a number, or by a space 
if a symbol, in order to distinguish it from the note's text. 
If there are any footnotes on a
short page, such as perhaps the last page of a chapter, they are placed
at the bottom of the page, not immediately below the last line of the main 
text.

   Endnotes may be set in the same font as the main text, but usually in
the same size\index{endnote!font size} as for footnotes.\footnote{That is, two
sizes smaller than the main text, but with the same leading.} 

   Endnotes may be grouped at the end of each chapter or collected together
towards the end of the document. If the latter, then they should be presented
in groups corresponding to the noted chapters. It is a courtesy to the reader
to indicate the page which gives rise to each note so that backward reference,
as well as forward reference, is facilitated; this is especially important
if there is no endnote mark\index{endnote!mark} in the main text. Endnotes
that have no reference mark\index{reference mark} in the text are usually
tagged with some words from the main text that identify the idea or statement
that they are referring to.
   In an endnote listing note numbers are usually either indented or the note
is set flush-and-hang\index{flush-and-hang} style; that is, with the 
first line set flushleft and any remaining lines indented.

Whether a note
is presented as a footnote or an endnote, it should always finish with a
period\index{period}.


\renewcommand*{\thefootnote}{\arabic{footnote}}
%%%%\afterpage{\let\footnoterule\oldfootnoterule}

\section{\prBackmatter}

    Divisions in the \pixbackmatter\ are not numbered.

    In commercial printing, saving a page here and a page there can save the
publisher money and hopefully at least some of the reduced cost will be passed
on to the readers. One way of reducing the number of pages is by reducing
the font size. The material in the \pixbackmatter\ is in some sense auxiliary 
to the \pixmainmatter, hence of less importance, and some of it may then be 
reasonably set in smaller type.

\subsection{Appendices}

    Appendices\index{appendix} come immediately after the main text. Depending
on their importance and interest they can be set in the same manner as the 
main text. If the appendices consist of long supporting documents they could
be set in a type one or two points smaller than the main text.

    In some instances there are appendices at the end of individual chapters,
where they form the last divisions of the chapters, and are treated as any of
the other divisions.

\subsection{Endnotes}

   In an endnote\index{endnotes} listing, note numbers are usually 
either indented or the note
is set flush-and-hang\index{flush-and-hang} style; that is, with the 
first line set flushleft and any remaining lines indented. There are, 
of course, corresponding numbers at the appropriate points in the main text.

   In another style, the endnotes are identified by a phrase taken from the 
main text together with the relevant page number. In this case there
are no numbers in the text to disturb the flow, but that often means that the
notes never get read.

    The notes are often set in the same sized type as used for footnotes or,
if they are exceptionally interesting, in a size intermediate with the main 
text.

\subsection{Bibliography}

   The list of books, etc., that the author has used as source material is
usually placed at the back of the book under the title `Bibliography'. In
some works there may be a bibliography\index{bibliography} at the end of 
each chapter and the title `References' is often used for these.

    A font size intermediate between those for quotations and footnotes is
very reasonable, and a slight extra space between entries, say two points,
can improve the readability.

\subsection{Glossary}

   The list of definitions of terms or symbols used in the text normally
comes towards the end of a book, although it could as well come towards
the end of the \pixfrontmatter, or a symbol list\indextwo{symbol}{list} 
in the \pixfrontmatter\ and a glossary\index{glossary}
in the \pixbackmatter. The terms are usually set in italics, or in textbooks 
in bold. Most often a flush-and-hang\index{flush-and-hang} style is used, with
perhaps one or two points extra leading between the entries.

    The type size could be the same as for quotations if the glossary is short,
or the size used for the bibliography for longer lists.

\subsection{Index}

    The entries in an index\index{index} are usually short and most indexes
are set in two, or more columns. As examples, author's names are usually 
relatively short so an index of names would typically be in two columns;
on the other hand, verse lines are relatively long and an index of first lines
is often set as a single column. In either case, the entries are usually set 
raggedright with the page numbers close to the corresponding item's text.
In multicolumn setting the gutter between the columns must be wide enough,
at least a pica, so
that the eye does not jump across it when reading an entry. The entries are
normally set flush-and-hang\index{flush-and-hang}.

    When there are subentries, or sub-subentries, they are typically
each indented by \U{1}{em} with respect to the major entry.

   A change of collation, such as between entries starting with `P' and those
with `Q' should be signalled by at least one or two blank lines. If the index
is long, then a suitable character (e.g., `Q') or word should be used as 
part of the break, indicating what is coming next. The index could be set
in the same size type as the bibliography\index{bibliography}.

\section{Type size}

\begin{table}
\centering
\caption{Some relative type sizes for elements of books}\label{tab:reltypesizing}
\begin{tabular}{lll} \toprule
Body size & 12/14  & 10/12 \\
Extracts  & 10/12 &  9/10 \\
Bibliography & 9/10 & 8/9 \\
Glossary     & 9/10 & 8/9 \\
Footnotes   & 8/9 & 8/8 \\
\bottomrule
\end{tabular}
\end{table}

    As indicated above, the type size is normally related to the `importance' 
of what is being set. Chapter headings are set in large type and footnotes 
set in small type. Of course, it is a matter of judgement as to what 
`important' means in any given work. Some possible combinations of type sizes
are given in \tref{tab:reltypesizing} though these should be considered as
starting points for a design rather than fixed rules.

\section{Poems and plays}

    In literature such as poems\index{poem} or plays\index{play} 
the length of the line is determined
by the author whereas in prose works the book designer establishes the measure.
For this kind of work the designer should respect the author's wishes as 
far as possible within a maximum text width.

\subsection{Poetry}

    If possible the type and measure should be chosen so that the longest
poetical line will fit on one printed line, so that the shape of the poem
is retained.

    Poems in a book of poetry will differ from one another in their width, 
and the best way of setting these is to optically center each poem on the 
page. However, blank verse and poems where the majority of the lines are
long are usually indented by a constant amount from the left margin.

    In some context verse lines are numbered, often every fifth or tenth line.
The numbers are usually small and right justified.

\subsection{Plays}

    When presenting a play a list of characters\indextwo{characters}{list} 
(\textit{Dramatis Personae}\index{Dramatis Personae?\textit{Dramatis Personae}})
is frequently given at the beginning of the play. It is presented between the
title and the start of the play itself, either on the same page as the title, 
or on a page by itself, or at the top of the first page of the play. The 
list may be ordered alphabetically, in order of appearance, or by the 
character's importance. A remark about a character, if less than a sentence, 
follows the name, separated by a comma. If the remark is one or several 
sentences they are set as usual.

    Act\indextwo{play}{act} and scene\indextwo{play}{scene} 
names and numbers are often treated in the same manner as
subheads in a prose work. A new act does not necessarily start a new page
but there should be at least twelve points above and six below the number.
A new scene has about eight points above and six points below the number.
Either arabic or roman numerals may be used for the numbers. If roman
is used for both, then uppercase for acts and lowercase for scenes.

   The name of each speaker in a play\indextwo{play}{speaker} must be readily 
identifiable and stand apart from the speech. Names are commonly set in a 
different font, such as small caps or italic, to the text which is usually set
in roman. They may be placed on a separate line, where they are most easily
identifiable, or, to save space, in the margin. The names are often 
abbreviated, and if so the abbreviations must be consistent throughout 
the work.

    Stage directions\index{play!stage directions}\index{stage directions} 
have to be differentiated from the text. They are usually set in italics 
and enclosed in brackets, or less often, in parentheses; speakers' names 
in stage directions, though, are set in roman to distinguish themselves. 
Directions at the start of a scene, such as saying who is entering, are 
typically centered while in the body of the scene are set flush right, 
often on a line by themselves.


\PWnote{2009/04/26}{Added section on selecting a typeface}
\section{Selecting a typeface}
\index{typeface!selecting}

    Choosing the `best' typeface to use is first a matter of practicalites
and secondly a question of aeshetics, feel, judgement, and other 
imponderables.

    First off\pagenote[First off]{Practically, though, the very first
thing is to select a typeface that you already have or that you can
\emph{legally} obtain.}, 
select a typeface that is both legible and readable.
The classical faces, those that have withstood the test of 
time\footnote{Measured in decades, if not centuries.} are
usually safe in this respect.

   Regarding the practicalities, 
if the work is to be presented on paper then
a more delicate face can be used than if it is to be displayed on a computer
screen where a more robust type is necessary (e.g., no fine serifs, no 
thin strokes), especially if it is a low resolution device.

  A work that includes many dates or dimensions needs a typeface with good
numerals and fractions. A face such as Palatino\facesubseeidx{Palatino}
 that has both lining and non-lining (oldstyle)
numerals is likely to be useful, or a face with numerals smaller than the
capitals, like Bell for example. Mathematics of course requires mathematical
symbols that do not clash with the regular text face.

    Where there are many quotations, particularly of verse, or references 
to titles of pictures or books implies that a good italic might be used
to advantage, for instance Joanna\facesubseeidx{Joanna} or 
Van Dijk\facesubseeidx{Van Dijk}.

    In technical works the font, size, and weight are used to distinguish 
between several layers of subdivisions. Faces with good relationships between
capitals, small caps, italic, and possibly boldness, should be sought out.  
Some sans bold types can mix with serif types; you might try 
Gill Sans\facesubseeidx{Gill Sans} or Futura\facesubseeidx{Futura}.



    After the practical aspects there are two main considerations: 
does the face match the `feel' of the work and is the face consistent 
with the period of the 
work? 

    The latter consideration is effectively a historical one. If you are
resetting a work that was originally published in the 1800's don't
use a twentieth century typeface, and certainly not a Sans-serif. On the
other hand using blackletter for setting a twentieth century novel, 
even a gothic one, is not a good idea even apart from the legibility and
readability factors.
    The face may be chosen to match the era of the text; perhaps a Sans-serif
for a book about modern art but not for one about medieval England. Using 
Caslon\facesubseeidx{Caslon}, which was created in England in the early 
18th century would 
probably not be a good choice for a book about Italian Renaissance art 
whereas Centaur\facesubseeidx{Centaur} which is based on Jenson's 
15th century Venetian roman
typeface would be, but then Centaur in turn would probably not be the
best choice for a physics book on the big bang and string theory. 

    The `feel' question is somewhat harder, partly depending on the 
presumed
sensibilities of the reader. If the work is about aspects of Italian
life then perhaps a light and quick Italian typeface would be 
preferable to a stolid English face. If it is a scientific work then
a clean and precise face would perhaps fit the bill. Technical
works often use analphabetic symbols, such as mathematical, chemical, or 
astrological symbols. Try and use a face that includes those symbols
that you need or, failing that, try and find complementary typefaces
that cover the range you need. The same applies if the work is 
multilingual, and especially if it involves different scripts
like, say, Latin and Greek; try and find matching faces, or at least 
typefaces that don't clash with each other.

%%%%%%%%%%%%%%%%%%%%%%%%%%%%%%%%%%%%%%%%%%%%%%% 

 

\begin{figure}
\begin{adjustwidth}{-0.05\textwidth}{-0.05\textwidth}
\centering
\begin{minipage}[t]{0.52\textwidth}\caslon
\Campion
\end{minipage}
\hfill
\begin{minipage}[t]{0.52\textwidth}\garamond
\Shelley
\end{minipage}

\vspace*{2\onelineskip}

\begin{minipage}[t]{0.52\textwidth}\bodoni
\Beddoes
\end{minipage}
\hfill
\begin{minipage}[t]{0.52\textwidth}\della
\Housman
\end{minipage}
\caption{Verses from four poems set with Caslon, Garamond, Bodoni and Della Robbia} \label{fig:poems1}
\end{adjustwidth}
\end{figure}


\begin{figure}
\begin{adjustwidth}{-0.05\textwidth}{-0.05\textwidth}
\centering
\begin{minipage}[t]{0.52\textwidth}\garamond
\Campion
\end{minipage}
\hfill
\begin{minipage}[t]{0.52\textwidth}\bodoni
\Shelley
\end{minipage}

\vspace*{2\onelineskip}

\begin{minipage}[t]{0.52\textwidth}\della
\Beddoes
\end{minipage}
\hfill
\begin{minipage}[t]{0.52\textwidth}\caslon
\Housman
\end{minipage}
\caption{Verses from four poems set with Garamond, Bodoni, Della Robbia
and Caslon}\label{fig:poems2}
\end{adjustwidth}
\end{figure}


\begin{figure}
\begin{adjustwidth}{-0.05\textwidth}{-0.05\textwidth}
\centering
\begin{minipage}[t]{0.52\textwidth}\bodoni
\Campion
\end{minipage}
\hfill
\begin{minipage}[t]{0.52\textwidth}\della
\Shelley
\end{minipage}

\vspace*{2\onelineskip}

\begin{minipage}[t]{0.52\textwidth}\caslon
\Beddoes
\end{minipage}
\hfill
\begin{minipage}[t]{0.52\textwidth}\garamond
\Housman
\end{minipage}
\caption{Verses from four poems set with Bodoni, Della Robbia, Caslon and Garamond}\label{fig:poems3}
\end{adjustwidth}
\end{figure}


\begin{figure}
\begin{adjustwidth}{-0.05\textwidth}{-0.05\textwidth}
\centering
\begin{minipage}[t]{0.52\textwidth}\della
\Campion
\end{minipage}
\hfill
\begin{minipage}[t]{0.52\textwidth}\caslon
\Shelley
\end{minipage}

\vspace*{2\onelineskip}

\begin{minipage}[t]{0.52\textwidth}\garamond
\Beddoes
\end{minipage}
\hfill
\begin{minipage}[t]{0.52\textwidth}\bodoni
\Housman
\end{minipage}
\caption{Verses from four poems set with Della Robbia, Caslon, Garamond and Bodoni}\label{fig:poems4}
\end{adjustwidth}
\end{figure}


    I was recently letterpress printing some poems and the fonts  
available in the print shop were Bodoni\facesubseeidx{Bodoni}, 
Caslon\facesubseeidx{Caslon}, Della Robbia\facesubseeidx{Della Robbia} and 
Garamond\facesubseeidx{Garamond}, 
all in \U{14}{pt}. The poems
were originally written within the period 701--2008 and my problem was to
try and choose what I thought would be the most appropriate font for each
poem. Caslon\facesubseeidx{Caslon} was created in 1720 so I decided that 
I would only use it for poems after that date. Although 
Garamond\facesubseeidx{Garamond} was created around 1550--1600 I
decided to use it for the pre-Caslon poems. 
To me the Bodoni\facesubseeidx{Bodoni} and 
Della Robbia\facesubseeidx{Della Robbia}  
had a very different feel to them, one being light and the other on the grim
side; I used these for the occasional poem instead of the 
Caslon\facesubseeidx{Caslon} or Garamond\facesubseeidx{Garamond}
to try and convey some of the emotional aspect of the piece. 
Figures~\ref{fig:poems1} through~\ref{fig:poems4} show verses from four
of the collection of poems set with the four fonts I had available. 
Decide what you would have chosen as being most appropriate. As a 
sidenote they are set here using \U{12}{pt} instead of the \U{10}{pt} body text size.
Poetry\index{poetry} can, with advantage, be set with larger type and 
leading than prose,
and in italic instead of roman.




\chapter{Picky points}
\let\footnoterule\oldfootnoterule

    The main elements of good typography are legibility\index{legibility} 
and page color\index{color}.
This chapter discusses some of the smaller points related to 
these topics.

\section{Word and line spacing}

    Research has shown that the competent reader recognises words by
their overall shape rather than by stringing together the individual letters
forming the words. A surprisingly narrow gap between words
is sufficient for most to distinguish the word boundaries.

    Most typographers state that the space between 
words\index{space!interword}
 in continuous
text should be about the width of the letter `i'. Any closer and the
words run together and too far apart the page looks speckled with white
spots and the eye finds it difficult to move along the line rather than
jumping to the next word in the next line. 
    Figure~\ref{fig:interword} illustrates different values of interword
spacing.

\setlength{\unitlength}{\fontdimen2\font}
\begin{figure}
\centering
%\fbox{%
\begin{minipage}{\textwidth}
\mbox{}\hrulefill\mbox{}
\begin{quotation}
\fontdimen2\font=2\fontdimen2\font
    The following paragraph is typeset with double the normal interword 
spacing for this font.

    Most typographers state that the space between words in continuous
text should be about the width of the letter `i'. Any closer and the
words run together and too far apart the page looks speckled with white
spots and the eye finds it difficult to move along the line rather than
jumping to the next word in the next line. 
Extra spacing after punctuation is not necessary.
\end{quotation}
\begin{quotation}
\fontdimen2\font=\unitlength
    The following paragraph is typeset with the normal interword spacing 
for this font.

    Most typographers state that the space between words in continuous
text should be about the width of the letter `i'. Any closer and the
words run together and too far apart the page looks speckled with white
spots and the eye finds it difficult to move along the line rather than
jumping to the next word in the next line. 
Extra spacing after punctuation is not necessary.
\end{quotation}
\begin{quotation}
\settowidth{\unitlength}{i}
\setlength{\unitlength}{0.75\unitlength}
\fontdimen2\font=\unitlength
    The interword spacing in the following paragraph is three-quarters of
the width of the letter `i'.
 
    Most typographers state that the space between words in continuous
text should be about the width of the letter `i'. Any closer and the
words run together and too far apart the page looks speckled with white
spots and the eye finds it difficult to move along the line rather than
jumping to the next word in the next line. 
Extra spacing after punctuation is not necessary.
\end{quotation}
\mbox{}\hrulefill\mbox{}
\end{minipage}
%} % end fbox
\fontdimen2\font=\unitlength \setlength{\unitlength}{1pt}
\caption{Interword spacings}\label{fig:interword}
\end{figure}

    In keeping with avoiding white spots, many typographers do not
recommend extra spacing after punctuation, although this does depend
partly on a country's typographic history and partly on the individual.
I always found typewritten texts with double spaces after the end
of sentences a particular eyesore. However, with typeset texts any 
extra spacing is usually not as large as that.

    The spacing between lines\index{space!interline} of text 
should be greater than the interword\index{space!interword}
spacing, otherwise there is a tendency for the eye to skip to the
next line instead of the next word. The space in question is the apparent
amount of whitespace between the bottom of the text on one line and the top
of the text on the next line. In a rough sense it is the leading minus the
actual height of the font.
Figure~\ref{fig:interline} illustrates
some text typeset with different line spacings. The normal interword
spacing is used in the samples. When text is set solid there is a tendancy
for the descenders on one line to touch, or even overlap, the ascenders on the 
following line.\footnote{\tx\ has a built-in mechanism that tries hard to 
prevent this from happening.}

\begin{figure}
\centering
%\fbox{%
\begin{minipage}{\textwidth}
\mbox{}\hrulefill\mbox{}
\normalfont\setlength{\unitlength}{\baselineskip}
\begin{quotation}
\normalfont\setlength{\baselineskip}{1em}
    This and the next paragraph are set solid --- the interline spacing 
is the same as the font size. \par
The normal interword spacing is used in these paragraphs.
    The spacing between lines of text should be greater than the interword
spacing, otherwise there is a tendency for the eye to skip to the
next line instead of the next word. \par
\end{quotation}
\begin{quotation}
\normalfont\setlength{\baselineskip}{\unitlength}
    This and the next paragraph are set with the normal interline spacing 
for the font. \par
The normal interword spacing is used in these paragraphs.
    The spacing between lines of text should be greater than the interword
spacing, otherwise there is a tendency for the eye to skip to the
next line instead of the next word. \par
\end{quotation}
\begin{quotation}
\normalfont\setlength{\parskip}{0.2\baselineskip}\setlength{\baselineskip}{1.2\unitlength}
    This and the next paragraph are set with the interline spacing 20\% 
greater than is normal for the font. \par
The normal interword spacing is used in these paragraphs.
    The spacing between lines of text should be greater than the interword
spacing, otherwise there is a tendency for the eye to skip to the
next line instead of the next word. \par
\end{quotation}
\mbox{}\hrulefill\mbox{}
\end{minipage}
%} % end fbox
\normalfont\setlength{\baselineskip}{\unitlength}
\setlength{\unitlength}{1pt}
\caption{Interline spacings}\label{fig:interline}
\end{figure}
\setlength{\unitlength}{1pt}

\PWnote{2009/03/31}{Added section on letterspacing}
\section{Letterspacing}
\index{letterspacing|(}

    \emph{Letterspacing} is the insertion of spaces between the letters
of a word.
    Frederic\index{Goudy, Frederic} Goudy (1865--1947), a very respected 
American typographer and type designer, is said to have been fond of saying, 
`A man who would letterspace lower case letters would steal sheep!' Writing
in 1999 Robert\index{Bringhurst, Robert} Bringhurst~\autocite[p. 31]{BRINGHURST99}
felt that to bring this dictum to modern times it was simply necessary to 
add that `A woman who would letterspace lowercase would also steal sheep.'

Letterspacing is usually restricted to titles composed of uppercase
letters with the intent of making the spaces between the letters
visually equal. Figure~\ref{fig:spacecaps} shows the word `HISTORY'
with various amounts of interletter spacings. With no extra spacing it
looks cramped compared with the spaced versions. Versions with uniform
spacing of thin (\U{0.167}{em}) and hair (\U{0.1}{em}) spaces are much
improved. However, the spacing of the letters `H', `I' and `S' leave
something to be desired, and is especially noticeable in the version
with thin spaces. Lastly a version with varied interletter spacing is
shown which is optically balanced; the spaces in this case are \U{0.09}{em},
\U{0.12}{em}, \U{0.1}{em}, \U{0.1}{em}, \U{0.07}{em} and \U{0.1}{em}. The difference between this
and the uniformly hair-spaced version is subtle.  Different letter
combinations and different fonts will require different amounts of
spaces.

    Figure~\ref{fig:spacesmallcaps} shows similar results for the same word
set in small caps instead of regular caps. In general you might find that 
if you do letterspace then uniform spacing is adequate for small caps.


\begin{figure}
\centering
\begin{tabular}{>{\Large}ll}
HISTORY & without letter spacing \\
H\,I\,S\,T\,O\,R\,Y & thin space between each letter \\
H\kern0.1em I\kern0.1em S\kern0.1em T\kern0.1em O\kern0.1em R\kern0.1em Y & hair space between each letter \\
H\kern0.09em I\kern0.12em S\kern0.1em T\kern0.1em O\kern0.07em R\kern0.1em Y & visually spaced \\

\end{tabular}
\caption{Letterspacing: uppercase letters} \label{fig:spacecaps}
\end{figure}

\begin{figure}
\centering
\begin{tabular}{>{\Large\scshape}ll}
history & without letter spacing \\
h\,i\,s\,t\,o\,r\,y & thin space between each letter \\
h\kern0.1em i\kern0.1em s\kern0.1em t\kern0.1em o\kern0.1em r\kern0.1em y & hair space between each letter \\
h\kern0.09em i\kern0.12em s\kern0.1em t\kern0.1em o\kern0.07em r\kern0.1em y & visually spaced \\
\end{tabular}

\caption{Letterspacing: small caps}\label{fig:spacesmallcaps}
\end{figure}

    There can be occasions, as with emphasis with fraktur fonts, when 
letterspacing lowercase will not get you hung for sheep stealing. These,
typically, are when dealing with some sans fonts such as a bold condensed
Univers. But be very careful. Italics should never be letterspaced as they
come from the handwriting tradition of `joined up letters', as my 
kindergarten teacher used to call them.

\index{letterspacing|)}

\section{Abbreviations and acronyms}

\index{abbreviation}
    The English style with abbreviations is to put a full stop (period) after
the abbreviation, unless the abbreviation ends with the same letter as the
full word. Thus, it is Mr for Mister, Dr for Doctor, but Prof.~for Professor.
No extra spacing should be used after the full stop, even if extra
spacing is normally used after punctuation.

    The general American, and English, trend nowadays is away from the use 
of periods (full stops) after abbreviations following the precept that
reducing typographic fussiness increases the ease of reading. Having said that,
where an abbreviation is a combination of abbreviations, such as Lt.~Col for
Lieutentant Colonel, often an internal period is used with a word space
between the elements.

    Acroynms\index{acronym} are typeset in uppercase but the 
question is, which uppercase?
The simple way is to use the uppercase of the normal font, like UNICEF, but
if there are too many acronyms scattered around the speckled effect starts
to intrude. If the font family has one, then small caps can be used,
giving \textsc{unicef}. If small caps are not available, or appear
undesireable, then a smaller size of the normal uppercase can be used,
such as {\small UNICEF} or {\footnotesize UNICEF}; some experimentation
may be required to select the appropriate size. These several versions
were input as:\par
\begin{tabular}{ll}
regular UNICEF text & \verb?regular UNICEF text? \\
regular \textsc{unicef} text & \verb?regular \textsc{unicef} text? \\
regular {\small UNICEF} text & \verb?regular {\small UNICEF} text? \\
regular {\footnotesize UNICEF} text & \verb?regular {\footnotesize UNICEF} text? 
\end{tabular}


\section{Dashes and ellipses}

\index{dash|(}
    Most fonts provide at least three lengths of dashes. The shortest is
the hyphen (-), then there is the en-dash (--) which is approximately the
width of the letter `n', and the largest is the em-dash (---) which is
approximately twice the length of an en-dash. An expert font may provide
more.

   Unsurprisingly, the hyphen\index{dash!hyphen}\index{hyphen} 
is used for hyphenation, such as in em-dash, or
at the end of a line where a word had to be broken.

    The en-dash\index{dash!en}\index{en-dash} 
is normally used between numerals to indicate a range. For
example a reference may refer to pages 21--27 in some journal or book. There
is no space surrounding the en-dash when used in this manner.

    The em-dash\index{dash!em}\index{em-dash}, 
or the en-dash, is used as punctuation --- often when making a side 
remark --- as a phrase separator.
 When en-dashes are used as punctuation it is normal to put spaces around them
but the question of spaces around an em-dash appears to be the subject of
some contention. Roughly half the participants in any discussion advocate
spaces while the other half view them as anathema. If you do use em-dashes
be sure to be consistent in your use, or otherwise, of spaces.

    Ellipses\index{ellipses} are those three, or is it four, 
dots indicating something is
missing or continues somewhat indefinitely. In the middle of a sentence,
or clause or \ldots\ they have a space on either side. At the end of
a sentence the English style is to have no spaces and include the full
stop, making four dots in all, like so\ldots.

   Dashes are also used to indicate missing characters or a word. Missing
characters in the middle of a word are indicated by a 
\U{2}{em}-dash\index{dash!2em@\U{2}{em}}\index{2em-dash@\U{2}{em}-dash} (a dash that
is twice as long as an em-dash), as in:
\begin{quote}
\textbf{snafu,} \textit{(U.S. slang)} \textit{n.} chaos. --- \textit{adj.}
  chaotic. [\textit{s}ituation \textit{n}ormal --- \textit{a}ll
  \textit{f}------d \textit{u}p.]
\end{quote}
A \U{3}{em}-dash\index{dash!1em@\U{3}{em}}\index{3em-dash@\U{3}{em}-dash} is used to indicate a 
missing word. When I lived in Maryland my
local small town newspaper was the \textit{Frederick Post.} 
The following is from an 
obituary I happened to read; I have hidden the name to protect the 
innocent.
\begin{quote}
  Although he had spent the last 92 years of his life here, 
Mr --------- was not a Fredericktonian.
\end{quote}

\index{dash|)}

\section{Punctuation}

\subsection{Quotation marks}

\index{quotation marks|(}
    Quotation marks surrounding speech and associated punctuation 
are a fruitful source of confusion.

    The American style is to use double quotes at the start (``) and 
end ('') of spoken words. If the speaker quotes in the speech then single
quote marks (` and ') are used to delineate the internal quotation\index{quotation}.

    The English practice is exactly the opposite. Main speech is delineated
by single quotes and internal quotations\index{quotation} by double quotes. 
In any event,
if single and double quotes are adjacent they should be separated by a thin
space\index{space!thin} in order to distinguish one from the other --- 
a full interword space is too wide.

    As there are likely to be few internal quotations\index{quotation} 
it seems to me that
the English practice produces a less spotty appearance than the American.
Figure~\ref{fig:qmarks} shows the same text typeset in both the English
and American styles. The example is from Lewis Carroll's 
\emph{Through the Looking Glass and what Alice Found There} 
and has an unusually large number of internal quotations\index{quotation}. 

\begin{figure}
\centering
\begin{minipage}{\textwidth}
\mbox{}\hrulefill\mbox{}
\begin{quotation}
    `There's glory for you!' 

    `I don't know what you mean by ``glory''\,', Alice said. 

    Humpty Dumpty smiled contemptuously. `Of course you don't --- till I tell
you. I meant ``there's a nice knock-down argument for you''!' 

    `But ``glory'' doesn't mean ``a nice knock-down argument''\,', Alice
objected. 

    `When \emph{I} use a word', Humpty Dumpty said, in a rather scornful
tone, `it means just what I choose it to mean --- neither more nor less'.
\end{quotation}
\mbox{}\hrulefill\mbox{}
\begin{quotation}
     ``There's glory for you!'' 

    ``I don't know what you mean by `glory,'\,'' Alice said. 

    Humpty Dumpty smiled contemptuously. ``Of course you don't --- till I tell
you. I meant `there's a nice knock-down argument for you!'\,'' 

    ``But `glory' doesn't mean `a nice knock-down argument,'\,'' Alice
objected. 

    ``When \emph{I} use a word,'' Humpty Dumpty said, in a rather scornful
tone, ``it means just what I choose it to mean --- neither more nor less.''
\end{quotation}
\mbox{}\hrulefill\mbox{}
\end{minipage}
\caption{Quotation marks: top English, bottom American}\label{fig:qmarks}
\end{figure}

    Where to put punctuation marks with quotes is vexatious. Again the
English and American practice tends to differ. The American tendency is
to put commas and periods inside the closing quote mark and colons and
semicolons after the mark. English editors prefer to put punctuation
after the mark.
In either case, it is difficult
to know exactly what to do. I get the impression that for every example
of the `correct' form there is a counter-example.
Some try and avoid the problem altogether by putting the lower marks, 
like commas or periods, 
directly below the quotation mark but that may cause problems if the 
resulting constructs look like question or exclamation marks. 
In \fref{fig:qmarks} I have tried to use the English and American 
punctuation styles in the respective examples but it is likely that there
are misplacements in both. I think it's basically a question of doing what
you think best conveys the sense, provided there is consistency.
\index{quotation marks|)}

\subsection{Footnote marks}

\index{footnote!mark|(}
    Where to put a footnote marker may be another vexed question in spite
of the general principal being easy to state: The mark goes immediately
after the text element that the note refers to.

    There is no doubt what this means\footnote{Except to some I know.} when
the text element is a word in the middle of other words. Doubt raises
its head when the reference is to a phrase, like this one\footnote{I hope
that this is a phrase.\label{fn:phrase}}, 
which is set off within commas, or when the note refers to a complete 
sentence.\footnote{Is this mark in the correct place?\label{fn:sentence}}

    Like punctuation and quotation marks\index{quotation marks}, 
should a footnote mark come before
or after the punctuation mark at the end of a phrase or a sentence? I have
shown both positions\footnote{Marks \ref{fn:phrase} and \ref{fn:sentence}.}
 in the previous paragraph. The 
general rule that I have deduced is that the mark comes after the 
punctuation, but there are always those who like to prove a rule.

\index{footnote!mark|)}

\index{symbol|(}
   There are other marks that may be associated with a word, like 
(registered) trademarks. These may produce ugly gaps. Sometimes these
cannot be avoided but it may be possible to change the text to minimise
the hiccup. There is an example of this on \pref{fn:ps}. I tried various
schemes in identifying `\pscript' as being a registered trademark of
Adobe Systems Incorporated. Among the discarded trials were:
\begin{quote}
\ldots languages like \pscript\texttrademark, presumably \ldots

\ldots languages like \pscript\textsuperscript{\textregistered}, presumably \ldots

\ldots like the \pscript\textsuperscript{\textregistered}{} language, presumably \ldots

\end{quote}
My final solution was to note the registered trademark information in
a footnote:
\begin{quote}
\ldots languages, like \pscript\footnote{\pscript{} is a registered 
trademark of Adobe Systems Incorporated.}, presumably \ldots
\end{quote}
In this case I decided that the footnote\index{footnote} was really tied to the word
`\pscript', taking the place of the registered symbol, so I put the
footnote mark\index{footnote!mark} before the comma rather than after it.
\index{symbol|)}

\subsection{Font changes}

\index{font!change|(}

    Sometimes a word or two may be set in a different font from the 
surrounding text, such as when emphasizing\index{emphasis} 
a word by setting it in an
italic font. If the word is followed by a punctuation mark the normal
practice is to set the punctuation mark using the new font instead of 
the normal
font. In some cases the font used for the punctuation may not be
particularly noticeable but sometimes it may be. 

    The \pixfrontmatter\ contains definitions of the word \textit{memoir,}
which is typeset using a bold font. The definitions thus commence like \\
\hspace*{2em} \textbf{memoir,} \textit{n.} \ldots \\
instead of \\
\hspace*{2em} \textbf{memoir}, \textit{n}. \ldots\footnote{Historical
  note: these notes started out as a part of the manual for the memoir
class, see \cite{MEMMAN}.} 

\index{font!change|)}

\section{Narrow measures}

\index{measure!narrow|(}
\index{column!narrow|(}

    Typesetting in a narrow column is difficult, especially if you are
trying to make the text flush left and right. As the lines get shorter
it becomes more and more difficult to fit the words in without an excessive
amount of interword spacing or word breaking at the ends of lines. 
In the limit, of course, there
will not be even enough room to put a syllable on a line.

    The best recourse in situations like this is to forget justification
and typeset raggedright\index{raggedright}. 
Raggedright looks far better than justified
text with lots of holes in it.
The question then is, to hyphenate or not to hyphenate?\index{hyphenation}

    With no hyphenation there is likely to be increased raggedness at
the line ends when compared with permitting some hyphenation. Hyphenation
can be used to reduce the raggedness but somehow short lines ending with
a hyphen may look a bit odd. This is where you have to exercise your
judgement and design skills.


%%%%%%%%%%%%%%%%%%%%%%%%%%%%%%%%%%%%%%%%%%%%%%%%%%%%%%%%%%%%



%% 109pt could be good for this
\newlength{\rag}
  \setlength{\rag}{0.25\textwidth} %% gives 90pt
  \addtolength{\rag}{10pt} 
  \addtolength{\rag}{9pt}  
%% 107pt seems good
\setlength{\rag}{107pt}
\newcommand{\ragtext}{%
The \ltx\ document preparation system is a special version of
Donald Knuth's \tx\ program. \tx\ is a sophisticated program
designed to produce high-quality typesetting, especially for
mathamatical works. It is extremely flexible albeit somewhat 
idiosynchratic. One can typeset justified, flushleft-raggedright,
centered, or raggedleft-flushright.}

\begin{figure}
\centering
\begin{minipage}[t]{\rag}
\raggedright
%%\noindent\verb?\raggedright?
\noindent No hyphenation
\vspace{\onelineskip}

\parindent=15pt\indent \ragtext
\end{minipage}
\hfill
\begin{minipage}[t]{\rag}
\raggedyright
%%\noindent\verb?\raggedyright?
\noindent Hyphenation (1)
\vspace{\onelineskip}

\indent \ragtext
\end{minipage}
\hfill
\begin{minipage}[t]{\rag}
\raggedyright[1em]
%%\noindent\verb?\raggedyright[1em]?
\noindent Hyphenation (2)
\vspace{\onelineskip}

\indent \ragtext
\end{minipage}
\caption{Raggedright text in narrow columns} \label{fig:raggedright}
\end{figure}

    Figure~\ref{fig:raggedright} shows a text set in a column
\the\rag{} wide with 
different raggednesses. Preventing hyphenation, as in the left column, 
resulted in very noticeably ragged text. Hyphenation has been allowed
in the other two columns, to differing degrees. I prefer
the one on the right but with different text and column widths the results
might have been different.

    Indexes\index{index!multiple column} are often typeset in double, 
or even triple or quadruple columns\index{column!multiple},
as each entry is typically short. Also, indexes are typically 
consulted
for a particular entry rather than being read as continuous text. To help
the eye, page numbers are normally typeset immediately after the 
name of the indexed topic, so indexes tend to be 
naturally raggedright\indextwo{index}{raggedright} as a matter of reader 
convenience.


\index{column!narrow|)}
\index{measure!narrow|)}

    Talking of hyphenation\index{hyphenation},
 each language has its own rules for allowable
hyphenation points. As you might now have come to suspect, English and 
American rules are different even though the language is nominally the same.
Broadly speaking, American English hyphenation points are typically based on
the sound of the word, so the acceptable locations are between syllables.
In British English the hyphenation points tend to be related to the
etymology of the word, so there may be different locations depending on 
whether the word came from the Greek or the Latin. If you are not sure
how a particular word should be hyphenated, look it up in a dictionary
that indicates the potential break points.

\section{Emphasis}

\index{emphasis|(}
    Underlining\index{underline} should \underline{emphatically} \underline{not} be
used to emphasise something in a typeset document. This is a hangover
from the days when manuscripts were typewritten and there was little
that could be done. The other way of emphasising something was to
put extra space between the characters of the w\,o\,r\,d being
e\,m\,p\,h\,a\,s\,i\,s\,e\,d, as has been done twice in this sentence
(for the words `word' and `emphasised' in case you didn't spot them).
Letterspacing\index{letterspacing} is usually confined to making fine adjustments to the 
physical spacing between letters in a book
title in order to make them appear to be optically uniformly spaced. 
As an aside, for me at least, that extra spacing just now produces the 
illusion that the
characters are slightly larger than normal, which is not the case.

    With the range of fonts and sizes available when typesetting there
are other methods for emphasis, although German typographers have used
letterspacing for emphasis with the fraktur and other similar font types.

    There are basically three aproaches: 
change the {\Large size} of the font;
change the \textbf{weight} of the font; or most usually, change the
\emph{shape} of the font. There is a creative tension when trying
to emphasise something --- there is the need to show the reader the 
emphasised
element, but there is also the desire not to interrupt the general flow
of the text. Out of the three basic options, changing the shape seems
to be a reasonable compromise between the need and the desire.
\index{emphasis|)}

\section{Captions and legends}

\index{caption|(}
\index{legend|(}

    I am not entirely sure what is the difference between a
caption and a legend as both terms refer to the title of an 
illustration\index{illustration} or table\index{table}. 
However, legend may also be used to refer to some explanatory 
material within an illustration\index{illustration}, such as the 
explanation of the symbols used on a map.

    In any event, captions and legends are usually typeset in a font that
is smaller than the main text font, and which may also be different from the
main font. For example, if the main font is roman and a sans font is used
for chapter titles, then it could be appropriate to use a small size
of the sans font for captions as well.

    The caption for a table\index{table} is normally placed above the 
table while captions for illustrations\index{illustration} are placed below.

\index{legend|)}
\index{caption|)}

\section{Tables}

\index{table|(}

    A table is text or numbers arranged in columns\index{column!multiple}, 
and nearly always
with a `legend'\index{legend} above each column describing the meaning of
the entries in the column. The legends and the column entries are
separated from each other, perhaps by some vertical space but more often
by a horizontal line.

    In general typographers dislike vertical lines in a table, which may
be likely to be used to separate the columns. I'm not sure why this is.
There is an obvious explanation when hand setting the individual characters
as, although it would be easy to set horizontal rules, it would be very 
difficult to get all the pieces of type with the bits of the vertical rules
aligned properly --- the eye is very sensitive to jags in what is meant to
be a straight line. In the days of digital typography the alignment problem
has gone away, so perhaps the antipathy to vertical lines is a tradition
from earlier days. On the other hand Edward Tufte~\autocite[p. 96]{TUFTE83} 
exhorts us to `Maximize the data-ink ratio' and to `Erase non-data-ink'
and Bringhurst~\autocite[p. 70]{BRINGHURST99} says `There should be a minimum
amount of furniture (rules, boxes, dots and other guiderails for
travelling through typographic space) and a maximum amount
of information'.

    If you want to use vertical lines, just be aware that not everybody
may appreciate your effort.


\index{table|)}

\section{Number formatting}

\index{number!formatting|(}

    Number formatting is country- and language-dependent. Continental 
Europe differs
from England, and in its turn the United States differs from England.

    Ignoring decimal numbers we have \emph{cardinal} and \emph{ordinal}
numbers. An ordinal\indextwo{ordinal}{number} number, like 3rd, 
indicates a position in a sequence,
while a cardinal\indextwo{cardinal}{number} number, like 3, 
expresses `how many'. \ltx\ typesets
numbers as cardinals, and these can be displayed as a sequence of
arabic digits or as upper- or lowercase Roman numerals. 

    In general text the tradition seems to be that cardinal numbers between
one and ten are spelled out rather than being presented as numerals.
In the United States numbers between one and ninety-nine are spelled (with
twenty-one, twenty-two, etc., being hyphenated). Similar customs
apply to ordinal numbers such as eighth, twenty-first. When not spelled out
ordinals should be set like 378th rather than 378\textsuperscript{th}.


    Regarding cardinal numbers represented as arabic digits,
some cultures prefer these to be presented as an unbroken string of
digits (e.g 12345). Other societies prefer the digits in longer 
numbers to separated, in some cases by commas (e.g., 12,345) or other 
punctuation marks (e.g., 12.345), and in others by small spaces (e.g., 12\:345); 
the digits are grouped into threes, counting from the right.

   When the arabic digits became generally used they, like the letters, 
were given both uppercase and lowercase forms. The uppercase form, like these
1 2 3 4 5 6 7 8 9 0, which
is the one normally supplied as part of a font are called 
\emph{titling figures}\indextwo{titling}{figures}, 
\emph{ranging figures}\indextwo{ranging}{figures}, or
\emph{lining figures}\indextwo{lining}{figures} because
they range or align with the uppercase.
Digits in this class all
have the same width so they are used in tables were numbers are meant to
be aligned vertically. They are also used when typesetting mathematics.

    The lowercase form, like \Moldstyle{1 2 3 4 5 6 7 8 9 0}, are called
\emph{text figures}\indextwo{text}{figures},
\emph{hanging figures}\indextwo{hanging}{figures},
\emph{lowercase figures}\indextwo{lowercase}{figures}, or
\emph{old-style figures}\indextwo{old-style}{figures}.
These may be used whenever the surrounding text is set in mixed case, or small caps; I
have seen them used typesetting the folios, but I must admit that they look very odd
to me in that situation.

    If you are typesetting mathematics, where you use lining figures, and are also using
old-style figures in the text then be very careful; `mathematical numbers' should always
be set with lining figures even if they are in the body of the text. For example: \\
\ldots from the equation the result is 42 \ldots \\
\ldots the men of the \Moldstyle{42}nd Foot performed magnificently \ldots.


\appendix

\chapter{Some typefaces}
\index{typeface}

   Most books or articles about typography at least mention some typefaces
by name, and may provide examples, and in some cases many examples.

   I have the commercial FontSite\index{FontSite} 
fonts\pagenote[commercial FontSite fonts]{In 2002 I purchased a CD containing 
512 fonts from \url{http://www.fontsite.com} but it appears that it is not 
in production any more. I understand that as of 2009/08/12 there were only 
39 copies still available.

   If you wish to \ltx\ the source of this document yourself and do not
have these fonts then you will get many error reports about missing fonts.
To avoid this use the \textsf{draft} class option. The document should
then process (without too many missing font errors), using the regular 
body typeface instead of any FontSite font.\index{FontSite}

    If you do have the fonts you may find that some glyphs are replaced
by others, in particular the `\P'.}
%
licensed from the SoftMaker/ATF library, supported for \ltx\ through 
Christopher\index{League, Christopher} League's estimable 
work~\autocite{TEXFONTSITE}. This enables me to show a few of the typefaces
displayed in some of the books mentioned in the bibliography, in particular
those by Birdsall~\autocite{BIRDSALL04}, Bringhurst~\autocite{BRINGHURST99} and
and Lawson~\autocite{LAWSON90}. To make it easier
to see and compare them at first there are a few characters displayed at \U{18}{pt}
and then text set at \U{12}{pt}.
 I have tried to indicate the typeface category
of each face, principally using those from \tref{tab:typecat}.
\pagenote[text set at \U{12}{pt}]{The Latin texts, when
suitably laid out, and their translations are as follows.
\vspace{\onelineskip}

\begin{minipage}[t]{0.45\textwidth}
Puella Rigensis ridebat \\
Quam tigris in tergo vehebat; \\
\hspace*{2em}Externa profecta,\\ 
\hspace*{2em}Interna revecta,\\ 
Risusque cum tigre manebat.
\end{minipage}
\hfill
\begin{minipage}[t]{0.45\textwidth}
There was a young lady of Riga \\
Who went for a ride on a tiger; \\
\hspace*{2em}They returned from the ride \\
\hspace*{2em}With the lady inside, \\
And a smile on the face of the tiger.
\end{minipage}
\vspace{\onelineskip}

\begin{minipage}[t]{0.45\textwidth}
Meum est propositum,\\  
In taberna mori,\\ 
Ut sint vina proxima,\\ 
Morientis ori.\\ 
Tunc cantabunt laetius\\ 
Angelorum chori;\\ 
`Sit Deus propitius\\ 
Huic potatori!' \par
{\raggedleft\footnotesize  The Arch Poet (fl. 1159--67)\par}
\end{minipage}
\hfill
\begin{minipage}[t]{0.45\textwidth}
I desire to end my days in a tavern drinking, \\
May some Christain hold the glass for me when I am shrinking; \\
That the Cherubim may cry, when they see me sinking, \\
`God be merciful to a soul of this gentleman's way of thinking.' \par
{\raggedleft\footnotesize Translation by Leight Hunt\par}
\end{minipage}
\vspace{\onelineskip}

\begin{minipage}[t]{0.45\textwidth}
Gaudeamus igitur,\\ 
Juvenes dum sumus\\ 
Post jucundum juventutem,\\ 
Post molestam senectutem,\\ 
Nos habebit humus. \par
{\raggedleft\footnotesize Students' song, 1267\par}
\end{minipage}
\hfill
\begin{minipage}[t]{0.45\textwidth}
Let us then rejoice, \\
While we are young. \\
After the pleasures of youth \\
And the burdens of old age \\
Earth will hold us.
\end{minipage}
}%% end of \pagenote

\section{Baskerville}
\facesubseeidx{Baskerville}

{\Pickfont{5bv} \atoq\ \onetoo} {\Pickit{5bv} \atoq\ \Qligs}\\%[\onelineskip]

{\pickfont{5bv} Baskerville, the first of the 
Transitional\typesubidx{Transitional} (Neoclassical\typesubidx{Neoclassical})
faces,
was designed by John Baskerville\index{Baskerville, John} in the
1750's and was cut by John Handy\index{Handy, John}. 
The face was more popular on the
Continent than in England where it was made. 
(See~\autocite{BRINGHURST99,LAWSON90})

\facetext}

\section{Bell}
\facesubseeidx{Bell}

{\Pickfont{5lb} \atoq\ \onetoo} {\Pickit{5lb} \atoq\ \Qligs}\\%[\onelineskip]

{\pickfont{5lb} The original Bell type was cut in London in 1788 by Richard
Austin\index{Austin, Richard} for a publisher named Bell. It is a 
Transitional\typesubidx{Transitional}
(Rationalist\typesubidx{Rationalist}) 
face. The FontSite version shown here is called 
LanstonBell\facesubseeidx{LanstonBell}. 
(See~\autocite{BIRDSALL04,BRINGHURST99,LAWSON90})

\facetext}

\section{Bembo}
\facesubseeidx{Bembo}

{\Pickfont{5boj} \atoq\ \onetoo} {\Pickit{5bo} \atoq\ \Qligs}\\%[\onelineskip]

{\pickfont{5boj} Bembo, produced by Monotype in 1929, is based on a roman
type cut by Francesco Griffo\index{Griffo, Francesco}  
in Venice in 1495. It is in the 
(Aldine/French\typesubidx{Aldine/French}, 
Renaissance\typesubidx{Renaissance},
Garald\typesubidx{Garald}) style. The FontSite version shown here is called
Bergamo\facesubseeidx{Bergamo}.
(See~\autocite{BRINGHURST99,LAWSON90})

\facetext}


\section{Bodoni}
\facesubseeidx{Bodoni}

{\Pickfont{5bdj} \atoq\ \onetoo} {\Pickit{5bd} \atoq\ \Qligs}\\%[\onelineskip]

{\pickfont{5bdj} Giambattista Bodoni\index{Bodoni, Giambattista} 
lived in Parma and designed hundreds of faces between 1765 and his death 
in 1813. Types in his style, now categorized as (Modern\typesubidx{Modern},
Romantic\typesubidx{Romantic}, Didone\typesubidx{Didone}), were 
revived in the first part of the twentieth century. 
(See~\autocite{BRINGHURST99,LAWSON90})

\facetext}


\section{Californian}
\facesubseeidx{Californian}

{\Pickfont{5uo} \atoq\ \onetoo} {\Pickit{5uo} \atoq\ \Qligs}\\%[\onelineskip]

{\pickfont{5uo} Frederic Goudy\index{Goudy, Frederic} cut his
  University of California Old Style\facesubseeidx{University of
    California Old Style} in 1938. Since then there have been many
  faces more or less faithfully based on his design. The general
  category of these is (Venetian\typesubidx{Venetian},
  Renaissance\typesubidx{Renaissance},
  Humanist\typesubidx{Humanist}). The FontSite version shown here is
  called University Old Style\facesubseeidx{University Old Style}.
  (See~\autocite{BRINGHURST99,LAWSON90})

\facetext}

\section{Caslon}
\facesubseeidx{Caslon}

{\Pickfont{5ca} \atoq\ \onetoo} {\Pickit{5ca} \atoq\ \Qligs}\\%[\onelineskip]

{\pickfont{5ca} 
William Caslon\index{Caslon, William} cut many faces between 1720 and 
his death in 1766. 
The first printing of the American Constitution used one of his types.
It falls into the (Dutch/English\typesubidx{Dutch/English}, 
Baroque\typesubidx{Baroque}, Garald\typesubidx{Garald}) category.
(See~\autocite{BRINGHURST99,LAWSON90})

\facetext}



\section{Centaur}
\facesubseeidx{Centaur}

{\Pickfont{5jr} \atoq\ \onetoo} {\Pickit{5jr} \atoq\ \Qligs}\\%[\onelineskip]

{\pickfont{5jr} Centaur was designed by Bruce Rogers\index{Rogers,
    Bruce} in 1912--14 based on Nicolas Jenson's\index{Jenson,
    Nicolas} roman type that he cut in Venice in 1465. It is perhaps
  the most faithfull rendition of Jenson's typeface, and is thus
  categorized as (Venetian\typesubidx{Venetian},
  Renaissance\typesubidx{Renaissance},
  Humanist\typesubidx{Humanist}). The FontSite version shown here is
  called Jenson Recut\facesubseeidx{Jenson Recut}.
  (See~\autocite{BRINGHURST99,LAWSON90,ROGERS49})

\facetext}


\section{Century}
\facesubseeidx{Century}

{\Pickfont{5cuj} \atoq\ \onetoo} {\Pickit{5cu} \atoq\ \Qligs}\\%[\onelineskip]

{\pickfont{5cuj} Century (Old Style) was created by 
Theodore De Vinne\index{De Vinne, Theodore} and 
Linn Boyd Benton\index{Benton, Linn Boyd} of the ATF as a magazine type,
named after De Vinne's \btitle{Century} magazine. It is essentially
a Transitional\typesubidx{Transitional} face.
(See~\autocite{BIRDSALL04,LAWSON90})


\facetext}

\section{Clarendon}
\facesubseeidx{Clarendon}

{\Pickfont{5cd} \atoq\ \onetoo} {\Picksl{5cd} \atoq\ \Qligs}\\%[\onelineskip]

{\pickfont{5cd} Clarendon is a Victorian face, one of many deriving from
a font cut by Benjamin Fox for the Fann Street Foundry, London, in 1845.
It is categorized as (Square Serif\typesubidx{Square Serif},
 Realist\typesubidx{Realist}, Mechanistic\typesubidx{Mechanistic}).
(See~\autocite{BIRDSALL04,BRINGHURST99,LAWSON90})

\facetext}


\section{Della Robbia}
\facesubseeidx{Della Robbia}

{\Pickfont{5de} \atoq\ \onetoo} {\Picksl{5de} \atoq\ \Qligs}\\%[\onelineskip]

{\pickfont{5de} Della Robbia was designed in 1902 by Thomas Maitland
Cleland\index{Cleland, Thomas Maitland} for the New York Type Foundry. 
It is based on 15th century Florentine inscriptional capitals.
It can be classed as a (Venetian\typesubidx{Venetian}, 
Renaissance\typesubidx{Renaissance}, Humanist\typesubidx{Humanist}) face.
(See~\autocite{CONSUEGRA04}) 

\facetext}


\section{Garamond}
\facesubseeidx{Garamond}

{\Pickfont{5gmj} \atoq\ \onetoo} {\Pickit{5gm} \atoq\ \Qligs}\\%[\onelineskip]

{\pickfont{5gmj} Claude Garamond\index{Garamond, Claude} 
(or Garamont) was one of several great
type cutters working Paris in the early sixteenth century. His style was
revived in the 1920's. Modern Garamonds are categorized as 
(Aldine/French\typesubidx{Aldine/French},  
Renaissance\typesubidx{Renaissance} or Baroque\typesubidx{Baroque},
Garald\typesubidx{Garald}).
(See~\autocite{BRINGHURST99,LAWSON90})

\facetext}

\begin{comment}
\section{Goudy Old Style}
\facesubseeidx{Goudy Old Style}

{\Pickfont{5go} \atoq\ \onetoo} {\Pickit{5go} \atoq\ \Qligs}\\%[\onelineskip]

{\pickfont{5go} Frederic Goudy\index{Goudy, Frederic} designed his Old Style
in 1915. It can be categorized as (Aldine/French\typesubidx{Aldine/French},
Renaissance\typesubidx{Renaissance}, Garald\typesubidx{Garald}).
(See~\autocite{LAWSON90})

\facetext}
\end{comment}

\section{Joanna}
\facesubseeidx{Joanna}

{\Pickfont{5js} \atoq\ \onetoo} {\Pickit{5js} \atoq\ \Qligs}\\%[\onelineskip]

{\pickfont{5js} Joanna was designed by Eric Gill\index{Gill, Eric} and was
first cut in 1930. 
The FontSite version shown here is called Jessica\facesubseeidx{Jessica}.
(See~\autocite{BIRDSALL04,BRINGHURST99}) % (Jessica)

\facetext}


\begin{comment}
%% this is the body font of the document
\section{Palatino}
\facesubseeidx{Palatino}

{\Pickfont{5plj} \atoq\ \onetoo} {\Pickit{5pl} \atoq\ \Qligs}\\%[\onelineskip]

{\pickfont{5plj} Palatino was designed by Hermann Zapf\index{Zapf, Hermann}
in 1948. It has been categorized as (Aldine/French\typesubidx{Aldine/French},
Lyrical Modernist\typesubidx{Lyrical Modernist} cum 
Neohumanist\typesubidx{Neohumanist}, Garald\typesubidx{Garald}).
The FontSite version shown here is called Palladio\facesubseeidx{Palladio}.
(See~\autocite{BRINGHURST99,LAWSON90}) % (Palladio)

\facetext}
\end{comment}


\section{Sabon}
\facesubseeidx{Sabon}

{\Pickfont{5svj} \atoq\ \onetoo} {\Pickit{5sv} \atoq\ \Qligs}\\%[\onelineskip]

{\pickfont{5svj} Sabon was designed by Jan Tschichold\index{Tschichold, Jan}
and issued in 1964 and can be categorized as 
(Aldine/French\typesubidx{Aldine/French}, 
 Renaissance\typesubidx{Renaissance},
 Garald\typesubidx{Garald}).
The FontSite version shown here is called Savoy\facesubseeidx{Savoy}.
(See~\autocite{BRINGHURST99,LAWSON90})  % (Savoy)


\facetext}



\section{Walbaum}
\facesubseeidx{Walbaum}

{\Pickfont{5wb} \atoq\ \onetoo} {\Picksl{5wb} \atoq\ \Qligs}\\%[\onelineskip]

{\pickfont{5wb} Walbaum types were originally cut by Justus Erich 
Walbaum\index{Walbaum, Justus Erich} about 1830. He was, together with
Bodoni\index{Bodoni, Giambattista} and Firmin Didot\index{Didot, Firmin},
one of the great type designers of the period. The face 
is categorized as (Modern\typesubidx{Modern}, 
Romantic\typesubidx{Romantic}, Didone\typesubidx{Didone}).
(See~\autocite{BIRDSALL04,BRINGHURST99})  


\facetext}


%%%%%%%%%%%%%%%%%%%%%%%%%%%%%%%%%%%%%%%%%%%% Sans faces

\begin{comment}
\section{Franklin Gothic}
\facesubseeidx{Franklin Gothic}

{\Pickfont{5fg} \atoq\ \onetoo} {\Pickit{5fg} \atoq\ \Qligs}\\%[\onelineskip]

{\pickfont{5fg} Franklin Gothic was designed for ATF by 
Morris Benton\index{Benton, Morris} in 1903 and is a 
(Realist\typesubidx{Realist} 
Sans-serif\typesubidx{Sans-serif}) face.
(See~\autocite{LAWSON90})

\facetext}
\end{comment}

\section{Futura}
\facesubseeidx{Futura}

{\Pickfont{5fu} \atoq\ \onetoo} {\Pickit{5fu} \atoq\ \Qligs}\\%[\onelineskip]

{\pickfont{5fu} Futura was designed by Paul Renner\index{Renner, Paul} 
in 1924--26. It has been categorized as (Sans-serif\typesubidx{Sans-serif},
Geometric Modernist\typesubidx{Geometric Modernist}, Geometric
Lineal\typesubidx{Geometric Lineal}). 
The FontSite version shown here is called Function\facesubseeidx{Function}.
(See~\autocite{BIRDSALL04,BRINGHURST99,LAWSON90})  % (Function)

\facetext}


\section{Gill Sans}
\facesubseeidx{Gill Sans}

{\Pickfont{5ch} \atoq\ \onetoo} {\Pickit{5ch} \atoq\ \Qligs}\\%[\onelineskip]

{\pickfont{5ch} Gill Sans was designed by Eric Gill\index{Gill, Eric} in 1927.
It can be classified as (Sans-serif\typesubidx{Sans-serif}, 
Geometric Modernist\typesubidx{Geometric Modernist} with 
Humanist\typesubidx{Humanist} hints, 
Lineal Humanist\typesubidx{Lineal Humanist}). 
The FontSite version shown here is called Chantilly\facesubseeidx{Chantilly}.
(See~\autocite{BIRDSALL04,BRINGHURST99,LAWSON90})  

\facetext}

\section{Goudy Sans}
\facesubseeidx{Goudy Sans}

{\Pickfont{5ys} \atoq\ \onetoo} {\Pickit{5ys} \atoq\ \Qligs}\\%[\onelineskip]

{\pickfont{5ys} Goudy Sans was designed by Frederic 
Goudy\index{Goudy, Frederic} in 1929--30. It is a Sans-serif\typesubidx{Sans-serif}.
(See~\autocite{BRINGHURST99,LAWSON90})

\facetext}

\section{Lydian}
\facesubseeidx{Lydian}

{\Pickfont{5ly} \atoq\ \onetoo} {\Pickit{5ly} \atoq\ \Qligs}\\%[\onelineskip]

{\pickfont{5ly} Lydian is a calligraphic Sans-serif\typesubidx{Sans-serif} 
face and was designed by Warren Chappell\index{Chappell, Warren} 
in the 1930's for ATF.
(See~\autocite{LAWSON90})

\facetext}


\section{News Gothic}
\facesubseeidx{News Gothic}

{\Pickfont{5ngj} \atoq\ \onetoo} {\Picksl{5ngj} \atoq\ \Qligs}\\%[\onelineskip]

{\pickfont{5ngj} News Gothic is a Sans-serif\typesubidx{Sans-serif} face
designed by Morris Benton\index{Benton, Morris} for ATF.
(See~\autocite{LAWSON90})

\facetext}


\section{Optima}
\facesubseeidx{Optima}

{\Pickfont{5opj} \atoq\ \onetoo} {\Pickit{5op} \atoq\ \Qligs}\\%[\onelineskip]

{\pickfont{5opj} Optima is a face designed
  by Hermann Zapf\index{Zapf, Hermann} in 1952--55. It has been classified
as (Sans-serif\typesubidx{Sans-serif}, Renaissance\typesubidx{Renaissance}
cum Neoclassical\typesubidx{Neoclassical}, 
Humanist Lineal\typesubidx{Humanist Lineal}).
The FontSite version shown here is called Opus\facesubseeidx{Opus}.
(See~\autocite{BRINGHURST99,LAWSON90}) %  (Opus)

\facetext}


\section{Syntax}
\facesubseeidx{Syntax}

{\Pickfont{5sx} \atoq\ \onetoo} {\Pickit{5sx} \atoq\ \Qligs}\\%[\onelineskip]

{\pickfont{5sx} Syntax, designed by Hans Eduard Meier\index{Meier, Hans Eduard}
and issued in 1969 is a Neohumanist\typesubidx{Neohumanist} 
Sans-serif\typesubidx{Sans-serif}.
The FontSite version shown here is called Struktur\facesubseeidx{Struktur}.
(See~\autocite{BRINGHURST99}) % (Struktur)

\facetext}

%%%%%%%%%%%%%%%%%%%%%%%%%%%%%%%%%%%%%%%%%%% script faces

\section{Legende}
\facesubseeidx{Legende}

{\Pickfont{5le} \atoq\ \onetoo\ \MakeUppercase{\atoq}} \\%[\onelineskip]

{\pickfont{5le} Legende is a Script\typesubidx{Script} face designed by 
Ernst Schneider\index{Schneider, Ernst} in 1937. It could be categorised
as having a Baroque\typesubidx{Baroque} feel to it.
(See~\autocite{BRINGHURST99,LAWSON90}) 

\facetext}

\begin{comment}
\section{Mistral}
\facesubseeidx{Mistral}

{\Pickfont{5ms} \atoq\ \onetoo\ \MakeUppercase{\atoq}} \\%[\onelineskip]

{\pickfont{5ms} Mistral, designed by Roger Excoffon\index{Excoffon, Roger}, 
is a Script\typesubidx{Script} face.
(See~\autocite{MCLEAN80}) 

\facetext}
\end{comment}


%%%\newcommand*{\Picksl}[1]{\thisfont{18}{14}{#1}{m}{sl}}

\section{Goudy Handtooled}
\facesubseeidx{Goudy Handtooled}

{\Pickfont{5gh} \atoq\ \onetoo} {\Pickfont{5gh} \MakeUppercase{\atoq}}\\%[\onelineskip]

{\thisfont{18}{14}{5gh}{m}{dp} \atoq\ \Onetoo}\\

{\pickfont{5gh} Goudy Handtooled was designed by Frederic 
Goudy\index{goudy, Frederic} in 1922 as a Decorative\typesubidx{Decorative} 
type. (See~\autocite{LAWSON90}) 

\facetext}

\section{Decorative}

%%%%%%%%%%%%%%%%%%%%%%%%%%%%%%%%%
%%\enlargethispage{-1\onelineskip}
%%%%%%%%%%%%%%%%%%%%%%%%%%%%%%%%
{\pickfont{5gh} Just for fun, here is a selection of some 
Decorative\typesubidx{Decorative} faces.
Unlike Goudy Handtooled the ones shown only come in uppercase. 
Also, \tref{tab:webo} shows the glyphs available in the 
Web-O-Mints\facesubseeidx{Web-O-Mints}
font which may be used to create borders and other decorative elements.
Web-O-Mints is one of two free decorative fonts made available by the 
Galapagos Design Group.\footnote{\url{http://www.galapagosdesign.com}}} \\

{\Pickfont{5ag}\facesubseeidx{Algerian}
 Algerian:
 \atoq\ \Onetoo} \\% {\Pickfont{5ag} \MakeUppercase{\atoq}}\\%[\onelineskip]

{\Pickfont{5dr}\facesubseeidx{Dresden}
 Dresden:
 \atoq\ \Onetoo} \\%{\Pickfont{5dr} \MakeUppercase{\atoq}}\\%[\onelineskip]

{\Pickfont{5eb}\facesubseeidx{Ebar deco}
 Erbar Deco:
 \atoq\ \Onetoo} \\% {\Pickfont{5eb} \MakeUppercase{\atoq}}\\%[\onelineskip]

{\Pickfont{5ga} Gallia:\facesubseeidx{Gallia}
 \atoq\ \Onetoo} \\% {\Pickfont{5ga} \MakeUppercase{\atoq}}\\%[\onelineskip]

\begin{table}
\centering
\caption{Glyphs in the Web-O-Mints font}\label{tab:webo}
\nohexoct
\fontsize{14}{14}
\xfonttable{U}{webo}{xl}{n}
\end{table}


\backmatter

\PWnote{2009/07/08}{Changed \cs{toclevel@section} so that Notes 
                    divisions appear in the bookmarks}
\makeatletter\renewcommand*{\toclevel@chapter}{-1}\makeatother 
\makeatletter\renewcommand*{\toclevel@section}{0}\makeatother
\clearpage
\printpagenotes
\clearpage

%%%%%%%%%%%%%%%%%%%%%%%%%%%%%%%%%%%% biblatex
\printbibliography[prenote=rectan]


\clearpage
\twocolindex
\pagestyle{index}
%\renewcommand{\chaptermark}[1]{}
\renewcommand{\preindexhook}{%
The first page number is usually, but not always, the primary reference to
the indexed topic.\vskip\onelineskip}
\indexintoc

%\makeatletter\renewcommand*{\toclevel@chapter}{-1}\makeatother 
%\makeatletter\renewcommand*{\toclevel@section}{0}\makeatother

%%%\raggedright  does nasty things to index entries
\printindex


%\onecolindex
%\renewcommand*{\preindexhook}{}
%\renewcommand*{\indexname}{Index of first lines}
%\printindex[lines]

\cleardoublepage
\pagestyle{empty}
\null\vfil

\begin{adjustwidth}{1in}{1in}
\begin{center}
{\Large\textsf{Colophon}}
\end{center}
\begin{center}
This manual was typeset using the LaTeX typesetting system
created by Leslie Lamport and the memoir class. 
The body text is set 10/\U{12}{pt} on a
\U{33}{pc} measure with Palatino designed by Hermann Zapf, which includes 
italics and small caps. Other fonts include
Sans, Slanted and Typewriter from Donald Knuth's 
Computer Modern family.

\end{center}

\end{adjustwidth}

\vfil

\end{document}

\endinput


%%%%%%%% The original in-document bibliography. Now replaced by memetc.bib
%%%%%%%% and biblatex package.
\begin{comment}

\renewcommand{\prebibhook}{%
CTAN is the Comprehensive TeX Archive Network. Information on how to
access CTAN is available at \url{http://www.tug.org}.
\par\vspace{\onelineskip}}

\begin{thebibliography}{GMS94A}
\small

\begin{comment}
\bibitem[AHK90]{IMPATIENT}
  Paul W. Abrahams, Kathryn Hargreaves and Karl Berry.
  \newblock \emph{TeX for the Impatient}.
  \newblock Addison-Wesley, 1990.
  \newblock (Available at
             \url{ftp://tug.org/tex/impatient})
\end{comment}

%%% keep
\bibitem[Ado01]{ADOBEBOOK}
  \emph{How to Create Adobe PDF eBooks}.
  \newblock Adobe Systems Inc.,
  \newblock 2001.
  \newblock (Available from 
             \url{http://www.adobe.com/epaper/tips/acr5ebook/pdfs/eBook.pdf})

\begin{comment}
\bibitem[Ars99]{URL}
  Donald Arseneau.
  \newblock \emph{The url package}.
  \newblock February, 1999.
  \newblock (Available from CTAN as 
             \url{/macros/latex/contrib/misc/url.sty})

\bibitem[Ars01a]{TITLEREF}
  Donald Arseneau.
  \newblock \emph{The titleref package}.
  \newblock April, 2001.
  \newblock (Available from CTAN as 
             \url{/macros/latex/contrib/misc/titleref.sty})

\bibitem[Ars01b]{CHAPTERBIB}
  Donald Arseneau.
  \newblock \emph{The chapterbib package}.
  \newblock September, 2001.
  \newblock (Available from CTAN as 
             \url{/macros/latex/contrib/cite/chapterbib.sty})

\bibitem[Ars07]{FRAMED}
  Donald Arseneau.
  \newblock \emph{The framed package} v0.95.
  \newblock October, 2007.
  \newblock (Available from CTAN as 
             \url{/macros/latex/contrib/misc/framed.sty})
\end{comment}

%%% keep
\bibitem[Bar92]{BAROLINI92}
  Helen Barolini.
  \newblock \emph{Aldus and his Dream Book}.
  \newblock Italica Press, 1992.
  \newblock ISBN 0--934977--22--4.

%%% keep
\bibitem[Bar01]{BARTRAM01}
  Alan Bartram.
  \newblock \emph{Five hundred years of book design}.
  \newblock Yale university Press, 2001.
  \newblock ISBN 0--300--09058--7.
  \newblock (First published 2001 by The British Library)

\begin{comment}
\bibitem[BDG89]{BIGELOW89}
  Charles Bigelow, Paul Hayden Duensing and Linnea Gentry (Eds).
  \newblock \emph{Fine Print on Type}. 1989.
  \newblock Fine Print, CA (ISBN 0--9607290-X) or
  \newblock Bedford Arts, CA (ISBN 0--938491--17--2).
\end{comment}

\bibitem[Bir04]{BIRDSALL04}
  Derek Birdsall.
  \newblock \emph{notes on book design}. 
  \newblock Yale University Press, 2004.
  \newblock ISBN 0--300--10347--6.

%%% keep
\bibitem[Boh90]{BOHLE90}
  Robert Bohle.
  \newblock \emph{Publication Design for Editors}.
  \newblock Prentice-Hall,
  \newblock 1990.

\begin{comment}
\bibitem[Ber02]{JURABIB}
  Jens Berger.
  \newblock \emph{The titlesec and titletoc packages}.
  \newblock September, 2002.
  \newblock (Available from CTAN in 
             \url{/macros/latex/contrib/titlesec})

\bibitem[Bez99]{TITLESEC}
  Javier Bezos.
  \newblock \emph{The titlesec and titletoc packages}.
  \newblock February, 1999.
  \newblock (Available from CTAN in 
             \url{/macros/latex/contrib/titlesec})

\bibitem[Bra94]{MAKEIDX}
  Johannes Braams \textit{et al}.
  \newblock \emph{Standard LaTeX2e packages makeidx and showidx}.
  \newblock November, 1994.
  \newblock (Available from CTAN as 
             \url{/macros/latex/base/makeindx.dtx(ins)})

\bibitem[Bra97]{ALLTT}
  Johannes Braams.
  \newblock \emph{The alltt environment}.
  \newblock June, 1997.
  \newblock (Available from CTAN as 
             \url{/macros/latex/base/alltt.dtx(ins)})
\end{comment}

%% keep
\bibitem[Bri99]{BRINGHURST99}
  Robert Bringhurst.
  \newblock \emph{The Elements of Typographic Style}.
  \newblock Hartley \& Marks, second edition,
  \newblock 1999. ISBN 0--88179--033--8.

%%% keep
\bibitem[Bur59]{BURT59}
  C.~L.~Burt.
  \newblock \emph{A Psychological Study of Typography}.
  \newblock Cambridge University Press,
  \newblock 1959.

\begin{comment}
\bibitem[Car94]{DELARRAY}
  David Carlisle.
  \newblock \emph{The delarray package}.
  \newblock March, 1994.
  \newblock (Available from CTAN in
             \url{/macros/latex/required/tools})

\bibitem[Car95]{AFTERPAGE}
  David Carlisle.
  \newblock \emph{The afterpage package}.
  \newblock October, 1995.
  \newblock (Available from CTAN in
             \url{/macros/latex/required/tools})

\bibitem[Car98b]{LONGTABLE}
  David Carlisle.
  \newblock \emph{The longtable package}.
  \newblock May, 1998.
  \newblock (Available from CTAN in
             \url{/macros/latex/required/tools})

\bibitem[Car98c]{ENUMERATE}
  David Carlisle.
  \newblock \emph{The enumerate package}.
  \newblock August, 1998.
  \newblock (Available from CTAN in
             \url{/macros/latex/required/tools})

\bibitem[Car98d]{REMRESET}
  David Carlisle.
  \newblock \emph{The remreset package}.
  \newblock August, 1998.
  \newblock (Available from CTAN in
             \url{/macros/latex/contrib/carlisle})

\bibitem[Car99]{TABULARX}
  David Carlisle.
  \newblock \emph{The tabularx package}.
  \newblock January, 1999.
  \newblock (Available from CTAN in
             \url{/macros/latex/required/tools})

\bibitem[CR99]{GRAPHICX}
  David Carlisle and Sebastian Rahtz.
  \newblock \emph{The graphicx package}.
  \newblock February, 1999.
  \newblock (Available from CTAN in
             \url{/macros/latex/required/graphics})

\bibitem[Car01]{DCOLUMN}
  David Carlisle.
  \newblock \emph{The dcolumn package}.
  \newblock May, 2001.
  \newblock (Available from CTAN in
             \url{/macros/latex/required/tools})

\bibitem[Car05]{COLOR}
  David Carlisle.
  \newblock \emph{Packages in the graphics bundle} (includes the color package).
  \newblock November, 2005.
  \newblock (Available from CTAN in
             \url{/macros/latex/required/graphics})

\bibitem[CK09]{BIDI}
  Fran\c{c}ois Charette and Vafa Khalighi.
  \newblock \emph{Bidi: A convenient interface for typesetting bidirectional
                  texts with XeLaTeX}.
  \newblock 2009.
  \newblock (Available from CTAN in
             \url{/macros/xetex/latex/bidi})
\end{comment}

%%% keep
\bibitem[CB99]{CHAPPELL99}
  Warren Chappell and Robert Bringhurst.
  \newblock \emph{A Short History of the Printed Word}.
  \newblock Hartley \& Marks, 1999.
  \newblock ISBN 0--88179--154--7.

\begin{comment}
\bibitem[CH88]{CHEN88}
  Pehong Chen and Michael A.~Harrison.
  \newblock `Index Preparation and Processing'.
  \newblock \emph{Software: Practice and Experience}, 19:8, pp. 897--915,
            September, 1988.
  \newblock (Available from CTAN in
             \url{/indexing/makeindex/paper})
\end{comment}

%%% keep
\bibitem[Che05]{CHENG05}
  Karen Cheng.
  \newblock \emph{Designing Type}.
  \newblock Yale University Press, 2005.
  \newblock ISBN 0--300--11150--9.

%%% keep
\bibitem[Chi93]{CMS}
  \newblock \emph{The Chicago Manual of Style}, Fourteenth Edition.
  \newblock The University of Chicago, 1993.
  \newblock ISBN 0--226--10389--7.

\begin{comment}
\bibitem[Coc02]{SUBFIGURE}
  Steven Douglas Cochran.
  \newblock \emph{The subfigure package}.
  \newblock March, 2002.
  \newblock (Available from CTAN in
             \url{/macros/latex/contrib/subfigure})
\end{comment}

\bibitem[Con04]{CONSUEGRA04}
  David Consuegra.
  \newblock \emph{American Type Design \& Designers}.
  \newblock Allworth Press, 2004.
  \newblock ISBN 1--58115--320--1.

%%% keep
\bibitem[CG96]{CONWAY96}
  John H.~Conway and Richard K.~Guy.
  \newblock \emph{The Book of Numbers}.
  \newblock Copernicus, Springer-Verlag, 1996.
  \newblock ISBN 0--387--97993--X.

%%% keep
\bibitem[Cra92]{CRAIG92}
  James Craig.
  \newblock \emph{Designing with Type: A Basic Course in Typography}.
  \newblock Watson-Guptill, NY,
  \newblock 1992.

\begin{comment}
\bibitem[Dal99a]{NATBIB}
  Patrick W.~Daly.
  \newblock \emph{Natural Sciences Citations and References}.
  \newblock May, 1999.
  \newblock (Available from CTAN in
             \url{/macros/latex/contrib/natbib})

\bibitem[Dal99b]{MAKEBST}
  Patrick W.~Daly.
  \newblock \emph{Customizing Bibliographic Style Files}.
  \newblock August, 1999.
  \newblock (Available from CTAN in
             \url{/macros/latex/contrib/custom-bib})
\end{comment}

%%% keep
\bibitem[Deg92]{DEGANI92}
  Asaf Degani.
  \newblock \emph{On the Typography of Flight-Deck Documentation}.
  \newblock NASA Contractor Report \# 177605.
  \newblock December, 1992.
  \newblock (Available from 
             \url{http://members.aol.com/willadams/typgrphy.htm#NASA})


\begin{comment}
\bibitem[Dow96]{DOWDING96}
  Geoffrey Dowding.
  \newblock \emph{Finer Points in the Spacing \& Arrangement of Type}.
  \newblock Hartley \& Marks, 1996.
  \newblock ISBN 0--88179--119--9.
\end{comment}

%%% keep
\bibitem[Dow98]{DOWDING98}
  Geoffrey Dowding.
  \newblock \emph{An Introduction to the History of Printing Types}.
  \newblock The British Library and Oak Knoll Press, 1998.
  \newblock ISBN 0--7123--4563--9 \textsc{uk},
                  1--884718--44--2 \textsc{usa}.

\begin{comment}
\bibitem[Dow00]{PATCHCMD}
  Michael J.~Downes.
  \newblock \emph{The patchcmd package}.
  \newblock July, 2000.
  \newblock (Available from CTAN in
             \url{/macros/latex/contrib/patchcmd})

\bibitem[Eij92]{TEXBYTOPIC}
  Victor Eijkhout.
  \newblock \emph{TeX by Topic}.
  \newblock Addison-Wesley, 1992.
  \newblock ISBN 0--201--56882--9.
  \newblock (Available from \url{http://www.eijkhout.net/tbt/}).

\bibitem[Eij99]{COMMENT}
  Victor Eijkhout.
  \newblock \emph{comment.sty}
  \newblock October, 1999.
  \newblock (Available from CTAN in
            \url{/macros/???????????})

\bibitem[Fai98]{MOREVERB}
  Robin Fairbairns.
  \newblock \emph{The moreverb package}.
  \newblock December, 1998.
  \newblock (Available from CTAN in
            \url{/macros/latex/contrib/moreverb})

\bibitem[Fai00]{FOOTMISC}
  Robin Fairbairns.
  \newblock \emph{footmisc --- a portmanteau package for customising
                  footnotes in LaTeX2e}.
  \newblock March, 2000.
  \newblock (Available from CTAN in
            \url{/macros/latex/contrib/footmisc})

\bibitem[FAQ]{FAQ}
  Robin Fairbairns.
  \newblock \emph{The UK TeX FAQ}.
  \newblock (Available from CTAN in
            \url{/help/uk-tex-faq})

\bibitem[Fea03]{BOOKTABS}
  Simon Fear.
  \newblock \emph{Publication quality tables in LaTeX}.
  \newblock March, 2003.
  \newblock (Available from CTAN in
            \url{/macros/latex/contrib/booktabs})

\bibitem[Fli98]{LETTRINE}
  Daniel Flipo.
  \newblock \emph{Typesetting `lettrines' in LaTeX2e documents}.
  \newblock March, 1998.
  \newblock (Available from CTAN in 
             \url{/macros/latex/contrib/lettrine})

\bibitem[Fly98]{FLYNN02}
  Peter Flynn.
  \newblock \emph{Formatting Information: A Beginner's Introduction to 
                  Typesetting with LaTeX2}.
  \newblock 2002.
  \newblock (Available from CTAN in 
             \url{/info/beginlatex})

\bibitem[Fra00]{CROP}
  Melchior Franz.
  \newblock \emph{The crop package}.
  \newblock February, 2000.
  \newblock (Available from CTAN in 
             \url{/macros/latex/contrib/crop})

\bibitem[FOS98]{FRIEDL98}
  Friedrich Friedl, Nicolaus Ott and Bernard Stein.
  \newblock \emph{Typography: An Encyclopedic Survey of Type Designs and
                  Techniques throughout History}.
  \newblock Black Dog \& Leventhal Publishers Inc., 1998.
  \newblock ISBN 1--57912--023--7.
\end{comment}

%%% keep
\bibitem[Gar66]{GARDNER66}
  Martin Gardner.
  \newblock \emph{More Mathematical Puzzles and Diversions}.
  \newblock Penguin Books, 1996.
  \newblock ISBN 0--14--020748--1.

%\bibitem[GMS94]{GOOSSENS94}
%  Michel Goossens, Frank Mittelbach and Alexander Samarin.
%  \newblock \emph{The LaTeX Companion}.
%  \newblock Addison-Wesley Publishing Company, 1994
%  \newblock (ISBN 0--201--54199--8), 1994.

\begin{comment}
\bibitem[GM\textsuperscript{+}07]{GCOMPANION}
  Michel Goossens, Frank Mittelbach, et al.
  \newblock \emph{The LaTeX Graphics Companion: Second edition}.
  \newblock Addison-Wesley, 2007.
  \newblock ISBN 0--321--50892--0.

\bibitem[GR99]{WCOMPANION}
  Michel Goossens and Sebastian Rahtz (with Eitan Gurari,
  Ross Moore and Robert Sutor).
  \newblock \emph{The LaTeX Web Companion: Integrating TeX, HTML and XML}.
  \newblock Addison-Wesley, 1999.
  \newblock ISBN 0--201--43311--7.
\end{comment}

%%% keep
\bibitem[Gou87]{GOULD87}
  J.~D.~Gould \textit{et al}.
  \newblock `Reading from CRT displays can be as fast as reading from paper'.
  \newblock \emph{Human Factors}, pp 497--517, 29:5, 1987.

%%% keep
\bibitem[HR83]{HARTLEY83}
  J.~Hartley and D.~Rooum.
  \newblock `Sir Cyril Burt and typography'.
  \newblock \emph{British Journal of Psychology}, pp 203--212, 74:2, 1983.

\begin{comment}
\bibitem[HM01]{HELLER01}
  Steven Heller and Philip B.~Meggs (Eds).
  \newblock \emph{Texts on Type: Critical Writings on Typography}.
  \newblock Allworth Press, 2001.
  \newblock ISBN 1--58115--082--2.

\bibitem[Hoe98]{HOENIG98}
  Alan Hoenig.
  \newblock \emph{TeX Unbound: LaTeX and TeX strategies for fonts,
                  graphics, and more}.
  \newblock Oxford University Press, 1998.
  \newblock ISBN 0--19--509686--X.
\end{comment}

%%% keep
\bibitem[HK75]{HVISTENDAHL75}
  J.~K.~Hvistendahl and M.~R.~Kahl.
  \newblock `Roman vs. sans serif body type: Readability and reader prference'.
  \newblock \emph{AANPA News Research Bulletin}, pp 3--11, 17 Jan., 1975.


\begin{comment}
\bibitem[Jon95]{INDEX}
  David M.~Jones.
  \newblock \emph{A new implementation of LaTeX's indexing commands}.
  \newblock September, 1995.
  \newblock (Available from CTAN in \url{/macros/latex/contrib/camel})

\bibitem[Keh98]{XINDY}
  Roger Kehr.
  \newblock \emph{xindy: A flexible indexing system}.
  \newblock February, 1998.
  \newblock (Available from CTAN in \url{/indexing/xindy})

\bibitem[Ker07]{XCOLOR}
  Uwe Kern.
  \newblock \emph{Extending LaTeX's color facilities: the xcolor package}.
  \newblock January, 2007.
  \newblock (Available from CTAN in 
             \url{/macros/latex/contrib/xcolor})
\end{comment}

%%% keep
\bibitem[Knu84]{TEXBOOK}
  Donald E.~Knuth.
  \newblock \emph{The TeXbook}.
  \newblock Addison-Wesley, 1984.
  \newblock ISBN 0--201--13448--9.

\begin{comment}
\bibitem[Knu86]{TEXPROGRAM}
  Donald E.~Knuth.
  \newblock \emph{TeX: The Program}.
  \newblock Addison-Wesley, 1986.
  \newblock ISBN 0--201--13437--3.

\bibitem[Knu87]{CM}
  Donald E.~Knuth.
  \newblock \emph{Computer Modern Typefaces}.
  \newblock Addison-Wesley, 1987.
  \newblock ISBN 0--201--134446--2.

\bibitem[Knu92]{METAFONT}
  Donald E.~Knuth.
  \newblock \emph{The METAFONT Book}.
  \newblock Addison-Wesley, 1992.
  \newblock ISBN 0--201--13444--6.
\end{comment}

%%% keep
\bibitem[Lam94]{LAMPORT94}
  Leslie Lamport.
  \newblock \emph{LaTeX: A Document Preparation System}.
  \newblock Addison-Wesley, 1994.
  \newblock ISBN 0--201--52983--1.
\begin{comment}

\bibitem[LEB04]{NTG}
  Leslie Lamport, Victor Eijkhout and Johannes Braams.
  \newblock \emph{NTG document classes for LaTeX version 2e}.
  \newblock June, 2004.
  \newblock (Available from CTAN in \url{/macros/latex/contrib/ntg})

\bibitem[LMB99]{CLASSES}
  Leslie Lamport, Frank Mittelbach and Johannes Braams.
  \newblock \emph{Standard document classes for LaTeX version 2e}.
  \newblock September, 1999.
  \newblock (Available from CTAN as \url{/macros/latex/base/classes.dtx})
\end{comment}

%%% keep
\bibitem[Law90]{LAWSON90}
  Alexander Lawson.
  \newblock \emph{Anatomy of a Typeface}.
  \newblock David R.~Godine, 1990. 
  \newblock ISBN 0--87923--333--8.

%%% keep
\bibitem[LA90]{LAWSONAGNER90}
  Alexander S.~Lawson with Dwight Agner.
  \newblock \emph{Printing Types: An Introduction}.
  \newblock Beacon Press, 1990. 
  \newblock ISBN 0--8070--6661--3.

\bibitem[Lea03]{TEXFONTSITE}
  Christopher League.
  \newblock \emph{TeX support for the FontSite 500 CD}.
  \newblock May 2003.
  \newblock (Available from 
             \url{http://contrapunctus.net/fs500tex})

\begin{comment}
\bibitem[Leh04]{FONTINST}
  Philipp Lehman.
  \newblock \emph{The Font Installation Guide}.
  \newblock December 2004.
  \newblock (Available from CTAN in 
             \url{/info/Type1fonts/fontinstallationguide})

\bibitem[Leu92]{LEUNEN92}
  Mary-Claire van Leunen.
  \newblock \emph{A Handbook for Scholars}.
  \newblock Oxford University Press, 1992.
  \newblock ISBN 0--19--506954--4.
\end{comment}

%%% keep
\bibitem[Liv02]{LIVIO02}
  Mario Livio.
  \newblock \emph{The Golden Ratio: The Story of Phi, the World's Most
     Astonishing Number}.
  \newblock Broadway Books, 2002.
  \newblock ISBN 0--7679--0816--3.

\begin{comment}
\bibitem[Lon91]{MULTIND}
  F.~W.~Long.
  \newblock \emph{multind}.
  \newblock August, 1991.
  \newblock (Available from CTAN as \url{/macros/latex209/contrib/misc/multind.sty})

\bibitem[Mad06]{CHAPSTYLES}
  Lars Madsen.
  \newblock \emph{Various chapter styles for the memoir class}.
  \newblock July, 2006.
  \newblock (Available from CTAN in \url{/info/latex-samples/MemoirChapStyles})

\bibitem[Mad07]{MEMEXSUPP}
  Lars Madsen.
  \newblock \emph{The Memoir Experimental Support Package}.
  \newblock 2007.
  \newblock (Available from CTAN in \url{/macros/latex/contrib/memexsupp})

\bibitem[McD98]{SECTSTY}
  Rowland McDonnell.
  \newblock \emph{The sectsty package}.
  \newblock November, 1998.
  \newblock (Available from CTAN in 
             \url{/macros/latex/contrib/secsty})

\bibitem[McL75]{MCLEAN75}
  Ruari McLean.
  \newblock \emph{Jan Tschichold: Typographer}.
  \newblock David R.~Godine, 1975.
  \newblock ISBN 0--87923--841--0.
\end{comment}

%%% keep
\bibitem[McL80]{MCLEAN80}
  Ruari McLean.
  \newblock \emph{The Thames \& Hudson Manual of Typography}.
  \newblock Thames \& Hudson, 1980.
  \newblock ISBN 0--500--68022--1.

\begin{comment}
\bibitem[McL95]{MCLEAN95}
  Ruari McLean (Ed).
  \newblock \emph{Typographers on Type}.
  \newblock W.~W.~Norton \& Co., 1995.
  \newblock ISBN 0--393--70201--4.
\end{comment}

%%% keep
\bibitem[MMc95]{MEGGS00}
  Philip B.~Meggs and Roy McKelvey (Eds).
  \newblock \emph{Revival of the Fittest: Digital Versions of the Classic Typefaces}.
  \newblock RC Publications, Inc., 2000.
  \newblock ISBN 1--883915--08--2.

\begin{comment}
\bibitem[Mit95]{DOCSHORTVRB}
  Frank Mittelbach.
  \newblock \emph{The doc and shortvrb packages}.
  \newblock May, 1995.
  \newblock (Available from CTAN in 
            \url{/macros/latex/base})

\bibitem[Mit98]{MULTICOL}
  Frank Mittelbach.
  \newblock \emph{An environment for multicolumn output}.
  \newblock January, 1998.
  \newblock (Available from CTAN in 
            \url{/macros/latex/required/tools})

\bibitem[MC98]{ARRAY}
  Frank Mittelbach and David Carlisle.
  \newblock \emph{A new implementation of LaTeX's tabular and array environment}.  \newblock May, 1998.
  \newblock (Available from CTAN in 
            \url{/macros/latex/required/tools})

\bibitem[MC00]{FIXLTX2E}
  Frank Mittelbach and David Carlisle.
  \newblock \emph{The fixltx2e package}.
  \newblock September, 2000.
  \newblock (Available from CTAN in 
            \url{/macros/latex/base})

\bibitem[MG\textsuperscript{+}04]{COMPANION}
  Frank Mittelbach, Michael Goossens, et al.
  \newblock \emph{The LaTeX Companion: Second Edition}.
  \newblock Addison-Wesley, 2004.
  \newblock ISBN 0--201--36299--6.
\end{comment}

%%% keep
\bibitem[Mor99]{MORISON99}
  Stanley Morison.
  \newblock \emph{A Tally of Types}.
  \newblock David R. Godine, 1999.
  \newblock ISBN 1--56792--004--7.

\begin{comment}
\bibitem[NG98]{SIDECAP}
  Rolf Niespraschk and Hubert G\"{a}\ss{}lein. 
  \newblock \emph{The sidecap package}.
  \newblock June, 1998.
  \newblock (Available from CTAN in 
            \url{/macros/latex/contrib/sidecap})

\bibitem[Oet]{LSHORT}
  Tobias Oetiker.
  \newblock \emph{The Not So Short Introduction to LaTeX2e}.
  \newblock (Available from CTAN in 
            \url{/info/lshort/english})

\bibitem[Oos96]{FANCYHDR}
  Piet van Oostrum.
  \newblock \emph{Page Layout in LaTeX}.
  \newblock June, 1996.
  \newblock (Available from CTAN in 
            \url{/macros/latex/contrib/fancyhdr})

\bibitem[Pak01]{SYMBOLS}
  Scott Pakin.
  \newblock \emph{The Comprehensive LaTeX Symbol List}.
  \newblock July, 2001.
  \newblock (Available from CTAN in 
            \url{/info/symbols/comprehensive})

\bibitem[dP84]{PARVILLE84}
  H.~de~Parville.
  \newblock Recreations mathematique: {La Tour d'Hanoi} et la question 
            du {Tonkin}.
  \newblock \emph{La Nature}, part {I}:285--286, Paris 1884.

\bibitem[Pat88a]{BIBTEX}
  Oren Patashnik.
  \newblock \emph{BibTeXing}.
  \newblock February, 1988.
  \newblock (Available from CTAN as 
            \url{/bibliography/bibtex/distribs/doc/btxdoc.tex})

\bibitem[Pat88b]{BIBTEXHACK}
  Oren Patashnik.
  \newblock \emph{Designing BibTeX Styles}.
  \newblock February, 1988.
  \newblock (Available from CTAN as 
            \url{/bibliography/bibtex/distribs/doc/btxhak.tex})
\end{comment}

%% keep
\bibitem[PP07]{POCKETPAL}
  \emph{Pocket Pal --- The Handy Book of Graphic Arts Production}.
  \newblock 20th Edition (Eds: Frank Ramano and Michael Riordan).
  \newblock International Paper Company, 2007.
  \newblock  ISBN 978--0--9772--7161--0.

\begin{comment}
\bibitem[Pug02]{MATHPAZO}
  Diego Puga.
  \newblock \emph{The Pazo Math fonts for mathematical typesetting
                  with the Palatino fonts}.
  \newblock May, 2002.
  \newblock (Available from CTAN in 
            \url{/fonts/mathpazo})

\bibitem[Rahtz01]{NAMEREF}
  Sebastian Rahtz.
  \newblock \emph{Section name references in LaTeX}.
  \newblock January, 2001.
  \newblock (Available from CTAN in 
            \url{/macros/latex/contrib/hyperref})

\bibitem[Rahtz02]{HYPERREF}
  Sebastian Rahtz.
  \newblock \emph{Hypertext marks in LaTeX}.
  \newblock May, 2002.
  \newblock (Available from CTAN in 
            \url{/macros/latex/contrib/hyperref})

\bibitem[Rec97]{EPSLATEX}
  Keith Reckdahl.
  \newblock \emph{Using Imported Graphics in LaTeX2e}.
  \newblock December, 1997.
  \newblock (Available from CTAN as
             \url{/info/epspatex.ps} or \url{/info/epslatex.pdf})
\end{comment}

%%% keep
\bibitem[Reh72]{REHE72}
  Rolf Rehe.
  \newblock `Type and how to make it most legible'.
  \newblock \emph{Design Research International}, 1972.

\begin{comment}
\bibitem[Rei07]{REINGOLD07}
  Edward M. Reingold.
  \newblock `Writing numbers in words in TeX'.
  \newblock TUGboat, 28, 2 pp 256--259, 2007.
\end{comment}

%%% keep
\bibitem[RAE71]{ROBINSON71}
  D.~O.~Robinson, M.~Abbamonte and S.~H.~Evans.
  \newblock `Why serifs are important: The perception of small print'.
  \newblock \emph{Visible Language}, pp 353--359, 4, 1971.

%%% keep
\bibitem[Rog43]{ROGERS43}
  Bruce Rogers.
  \newblock \emph{Paragraphs on Printing}.
  \newblock William E. Rudge's Sons, Inc., 1943.
  \newblock (Reissued by Dover, 1979, ISBN 0--486--23817--2)

%%% keep
\bibitem[Rog49]{ROGERS49}
  Bruce Rogers.
  \newblock \emph{Centaur Types}.
  \newblock October House, 1949.

\begin{comment}
\bibitem[RBC74]{ROUSEBALL}
  W. W. Rouse Ball and H. S. M. Coxeter. 
  \newblock \emph{Mathematical Recreations and Essays}.
  \newblock University of Toronto Press, twelfth edition, 1974.
\end{comment}

%%% keep
\bibitem[Ryd76]{RYDER}
  John Ryder. 
  \newblock \emph{Printing for Pleasure}. Revised edition.
  \newblock The Bodley Head, 1976.
  \newblock ISBN 0--370--10443--9.
  \newblock (In the USA published by Henry Regenery Co., Michigan, 1977.
             ISBN 0--8092--78103--3)

\begin{comment}
\bibitem[SW94]{EBOOK}
  Douglas Schenck and Peter Wilson.
  \newblock \emph{Information Modeling the EXPRESS Way}.
  \newblock Oxford University Press, 1994.
  \newblock ISBN 0--19--508714--3.

\bibitem[SRR99]{VERBATIM}
  Rainer Sch\"{o}pf, Bernd Raichle and Chris Rowley.
  \newblock \emph{A New Implementation of LaTeX's verbatim
                  and verbatim* Environments}.
  \newblock December, 1999.
  \newblock (Available from CTAN in
            \url{/macros/latex/required/tools})
\end{comment}

%%% keep
\bibitem[Sch97]{SCHRIVER97}
  Karen A.~Schriver.
  \newblock \emph{Dynamics in Document Design}.
  \newblock Wiley \& Sons, 1997.

%%% keep
\bibitem[Sme96]{SMEIJERS96}
  Fred Smeijers.
  \newblock \emph{Counterpunch: making type in the sixteenth century,
                  designing typefaces now}.
  \newblock Hyphen Press, London, 1996.
  \newblock ISBN 0--907259--06--5.

\begin{comment}
\bibitem[Sne04]{SNEEP04}
  Maarten Sneep.
  \newblock \emph{The atmosphere in the laboratory: cavity ring-down
                  measurements on scattering and absorption}.
  \newblock Phd thesis,
  \newblock Vrijie Universiteit, Amsterdam, 2004. 

\bibitem[Tal06]{DATETIME}
  Nicola L. C. Talbot.
  \newblock \emph{datetime.sty: Formatting Current Date and Time}.
  \newblock December, 2006.
  \newblock (Available from CTAN in
            \url{/macros/latex/contrib/datetime})

\bibitem[Thi98]{ORNAMENTAL}
  Christina Thiele.
  \newblock `Hey --- it works: Ornamental rules'. 
  \newblock \emph{TUGboat}, 
  \newblock vol. 19, no. 4, p 427, December 1998.

\bibitem[Thi99]{TTC199}
  Christina Thiele.
  \newblock `The Treasure Chest: Package tours from CTAN', 
  \newblock \emph{TUGboat}, 
  \newblock vol. 20, no. 1, pp 53--58, March 1999.

\bibitem[TJ05]{CALC}
  Kresten Krab Thorup, Frank Jensen (and Chris Rowley).
  \newblock \emph{The calc package --- Infix notation arithmetic in LaTeX}.
  \newblock August, 2005.
  \newblock (Available from CTAN in
            \url{/macros/latex/required})
\end{comment}

%%% keep
\bibitem[Tin63]{TINKER63}
  Miles A.~Tinker.
  \newblock \emph{Legibility of Print}.
  \newblock Books on Demand (University Microfilms International), 1963.

\begin{comment}
\bibitem[Tob00]{SETSPACE}
  Geoffrey Tobin.
  \newblock \emph{setspace.sty}.
  \newblock December, 2000.
  \newblock (Available from CTAN in
            \url{/macros/latex/contrib/setspace})
\end{comment}

%%% keep
\bibitem[Tsc91]{TSCHICHOLD91}
  Jan Tschichold.
  \newblock \emph{The Form of the Book}.
  \newblock Lund Humphries, 1991.
  \newblock ISBN 0--85331--623--6.

%%% keep
\bibitem[Tuf83]{TUFTE83}
  Edward R. Tufte.
  \newblock \emph{The Visual Display of Quantative Information}.
  \newblock Graphics Press, 1983.


\begin{comment}
\bibitem[Ume99]{GEOMETRY}
  Hideo Umeki.
  \newblock \emph{The geometry package}.
  \newblock November, 1999.
  \newblock (Available from CTAN in
            \url{/macros/latex/contrib/geometry})
\end{comment}

%%% keep
\bibitem[Whe95]{WHEILDON95}
  Colin Wheildon.
  \newblock \emph{Type \& Layout}.
  \newblock Strathmore Press, 1995.
  \newblock ISBN 0--9624891--5--8.

\begin{comment}
\bibitem[Wil00]{CATALOGUE}
  Graham Williams.
  \newblock \emph{The TeX Catalogue}.
  \newblock (Latest version on CTAN as \url{/help/Catalogue/catalogue.html})
\end{comment}

%%% keep
\bibitem[Wil93]{ADRIANWILSON93}
  Adrian Wilson.
  \newblock \emph{The Design of Books}.
  \newblock Chronicle Books, 1993.
  \newblock ISBN 0--8118--0304--X.



\begin{comment}
\bibitem[Wil99b]{TOCVSEC2}
  Peter Wilson.
  \newblock \emph{The tocvsec2 package}.
  \newblock January, 1999.
  \newblock (Available from CTAN in 
            \url{/macros/latex/contrib/tocvsec2})

\bibitem[Wil00a]{EPIGRAPH}
  Peter Wilson.
  \newblock \emph{The epigraph package}.
  \newblock February, 2000.
  \newblock (Available from CTAN in 
            \url{/macros/latex/contrib/epigraph})

\bibitem[Wil00b]{ISOCLASS}
  Peter Wilson.
  \newblock \emph{LaTeX files for typesetting ISO standards}.
  \newblock February, 2000.
  \newblock (Available from CTAN in 
            \url{/macros/latex/contrib/isostds/iso})

\bibitem[Wil00c]{NEXTPAGE}
  Peter Wilson.
  \newblock \emph{The nextpage package}.
  \newblock February, 2000.
  \newblock (Available from CTAN as 
            \url{/macros/latex/contrib/misc/nextpage.sty})

\bibitem[Wil00d]{NEEDSPACE}
  Peter Wilson.
  \newblock \emph{The needspace package}.
  \newblock March, 2000.
  \newblock (Available from CTAN as 
            \url{/macros/latex/contrib/misc/needspace.sty})

\bibitem[Wil00e]{XTAB}
  Peter Wilson.
  \newblock \emph{The xtab package}.
  \newblock April 2000.
  \newblock (Available from CTAN in 
             \texttt{macros/latex/contrib/xtab})

\bibitem[Wil01a]{ABSTRACT}
  Peter Wilson.
  \newblock \emph{The abstract package}.
  \newblock February, 2001.
  \newblock (Available from CTAN in 
            \url{/macros/latex/contrib/abstract})

\bibitem[Wil01b]{CHNGPAGE}
  Peter Wilson.
  \newblock \emph{The chngpage package}.
  \newblock February, 2001.
  \newblock (Available from CTAN as 
            \url{/macros/latex/contrib/misc/chngpage.sty})

\bibitem[Wil01c]{APPENDIX}
  Peter Wilson.
  \newblock \emph{The appendix package}.
  \newblock March, 2001.
  \newblock (Available from CTAN in 
            \url{/macros/latex/contrib/appendix})

\bibitem[Wil01d]{CCAPTION}
  Peter Wilson.
  \newblock \emph{The ccaption package}.
  \newblock March, 2001.
  \newblock (Available from CTAN in 
            \url{/macros/latex/contrib/ccaption})

\bibitem[Wil01e]{CHNGCNTR}
  Peter Wilson.
  \newblock \emph{The chngcntr package}.
  \newblock April, 2001.
  \newblock (Available from CTAN as 
            \url{/macros/latex/contrib/misc/chngcntr.sty})

\bibitem[Wil01f]{HANGING}
  Peter Wilson.
  \newblock \emph{The hanging package}.
  \newblock March, 2001.
  \newblock (Available from CTAN in 
            \url{/macros/latex/contrib/hanging})

\bibitem[Wil01g]{TITLING}
  Peter Wilson.
  \newblock \emph{The titling package}.
  \newblock March, 2001.
  \newblock (Available from CTAN in 
            \url{/macros/latex/contrib/titling})

\bibitem[Wil01h]{TOCBIBIND}
  Peter Wilson.
  \newblock \emph{The tocbibind package}.
  \newblock April, 2001.
  \newblock (Available from CTAN in 
            \url{/macros/latex/contrib/tocbibind})


\bibitem[Wil01i]{TOCLOFT}
  Peter Wilson.
  \newblock \emph{The tocloft package}.
  \newblock April, 2001.
  \newblock (Available from CTAN in 
            \url{/macros/latex/contrib/tocloft})


\bibitem[Wil01j]{MEMOIR}
  Peter Wilson.
  \newblock \emph{The LaTeX memoir class for configurable book 
                  typesetting: Source code}.
  \newblock July, 2001.
  \newblock (Available from CTAN in 
            \url{/macros/latex/contrib/memoir})

\bibitem[Wil01k]{VERSE}
  Peter Wilson.
  \newblock \emph{Typesetting simple verse with LaTeX}
  \newblock July, 2001.
  \newblock (Available from CTAN in 
            \url{/macros/latex/contrib/verse})

\bibitem[Wil01l]{BOOKLET}
  Peter Wilson.
  \newblock \emph{Printing booklets with LaTeX}
  \newblock August, 2001.
  \newblock (Available from CTAN in 
            \url{/macros/latex/contrib/booklet})

\bibitem[Wil03a]{LAYOUTS}
  Peter Wilson.
  \newblock \emph{The layouts package}
  \newblock November, 2003.
  \newblock (Available from CTAN in 
            \url{/macros/latex/contrib/layouts})

\bibitem[Wil03b]{LEDMAC}
  Peter Wilson.
  \newblock \emph{ledmac: A presumptuous attempt to port EDMAC and TABMAC
                  to LaTeX}
  \newblock November, 2003.
  \newblock (Available from CTAN in 
            \url{/macros/latex/contrib/ledmac})

\bibitem[Wil04a]{GLISTER3}
  Peter Wilson.
  \newblock `Glisterings'.
  \newblock TUGboat, 25, 2 pp 201--202, 2004.

\bibitem[Wil04b]{PAGENOTE}
  Peter Wilson.
  \newblock \emph{The pagenote package}
  \newblock September, 2004.
  \newblock (Available from CTAN in 
            \url{/macros/latex/contrib/pagenote})

\bibitem[Wil07a]{TITLEPAGES}
  Peter Wilson.
  \newblock \emph{Some Examples of Title Pages}.
  \newblock Herries Press, 2007.
  \newblock (Available from CTAN as \url{info/latex-samples/titlepages.pdf})

\bibitem[Wil07b]{MEMCODE}
Peter Wilson.
  \newblock \emph{The \ltx\ memoir class for configurable book typesetting: source code}
  \newblock November, 2007.
  \newblock (Available from CTAN in 
            \url{/macros/latex/contrib/memoir})

\bibitem[Wil07c]{MEMMAN}
Peter Wilson.
  \newblock \emph{The Memoir Class for Configurable Typesetting --- User Guide}
  \newblock November, 2007.
  \newblock (Available from CTAN in 
            \url{/macros/latex/contrib/memoir})

\bibitem[Wil07d]{MEMADD}
Peter Wilson.
  \newblock \emph{ADDENDUM: The Memoir Class for Configurable Typesetting --- User Guide}
  \newblock November, 2007.
  \newblock (Available from CTAN in 
            \url{/macros/latex/contrib/memoir})

\bibitem[Wil07e]{GLISTER07}
Peter Wilson.
  \newblock `Glisterings', \emph{TUGboat}, 28(2):229--232, 2007.
\end{comment}

%%% keep
\bibitem[Wil07f]{TUGKEYNOTE07}
Peter Wilson.
  \newblock `Between then and now --- A meandering memoir', 
  \newblock \emph{TUGboat}, 28(3):280--298, 2007.


\begin{comment}
\bibitem[Wil08a]{CHANGEPAGE}
  Peter Wilson.
  \newblock \emph{The changepage package}.
  \newblock March, 2008.
  \newblock (Available from CTAN as 
            \url{/macros/latex/contrib/misc/changepage.sty})

\bibitem[Wil08b]{GLISTER08}
Peter Wilson.
  \newblock `Glisterings', \emph{TUGboat}, 29(2):324--327, 2008.

\bibitem[Wil09a]{FONTTABLE}
  Peter Wilson.
  \newblock \emph{The fonttable package}
  \newblock April, 2009.
  \newblock (Available from CTAN in 
            \url{/macros/latex/contrib/fonttable})
\end{comment}

%%% keep
\bibitem[Wil09]{MEMMAN}
Peter Wilson.
  \newblock \emph{The Memoir Class for Configurable Typesetting --- User Guide},
            Seventh edition, 
  \newblock July, 2009.
  \newblock (Available from CTAN in 
            \url{/macros/latex/contrib/memoir})
\begin{comment}

\bibitem[Wil??]{RUMOUR}
Peter Wilson.
\newblock \emph{A Rumour of Humour: A scientist's commonplace book}.
\newblock To be published?
\end{comment}

%%% keep
\bibitem[Wis03]{WISHART03}
David Wishart.
\newblock \emph{The Printing of Mathematics}
\newblock in \emph{Type \& Typography: Highlights from \emph{Matrix}, the
       review for printers \& bibliophiles}.
\newblock Mark Batty Publisher, 2003.
\newblock ISBN 0--9715687--6--6.
\newblock (Originally published in \emph{Matrix 8}, 1988)

%%% keep
\bibitem[Wul53]{WULLING-FOOTNOTES}
  Emerson G. Wulling.
  \newblock \emph{A Comp's-Eye View of Footnotes}.
  \newblock Sumac Press, 1953.

%%% keep
\bibitem[Zac69]{ZACHRISSOM69}
  B.~Zachrissom.
  \newblock \emph{Studies in the Legibility of Printed Text}.
  \newblock Almqvist \& Wiksell, Stockholm, 1969.

\begin{comment}
\bibitem[Zan98]{FANCYBOX}
  Timothy Van Zandt.
  \newblock \emph{Documentation for fancybox.sty: Box tips and tricks for 
             LaTeX},
  \newblock November, 1998.
  \newblock (Available from CTAN in 
            \url{/macros/latex/contrib/fancybox})
\end{comment}

%%% keep
\bibitem[Zap00]{ZAPF00}
  Hermann Zapf.
  \newblock \emph{The Fine Art of Letters}.
  \newblock The Grolier Club, 2000.
  \newblock ISBN 0--910672--35--0.

\end{thebibliography}

\end{comment}
